\documentclass{article}
\usepackage{fullpage}
\usepackage{commath}
\addtolength{\parskip}{\baselineskip}

\begin{document}
\begin{center}
\Large
Documentation for PYRATS:
Python Routines for Analyzing Transport Simulations
\end{center}
\normalsize

\noindent \textbf{Introduction: }PYRATS is a data processing tool designed to make creating plots of output data from GYRO, TGYRO, or NEO easy.  It makes use of the python programming language, and the associated packages numpy and matplotlib.  With some knowledge of the python language and matplotlib, the typical user should be able to create plots quickly and efficiently from the python command line interpreter.

\noindent To begin, type the following command into the terminal:
\fontfamily{\ttdefault}\selectfont

\$ python

\fontfamily{\rmdefault}\selectfont
\noindent This will bring up the python interpreter, which is the interface you will use to interact with PYRATS.  As a first example, we will look at the layout of TGYROData.  Now execute the following commands:
\fontfamily{\ttdefault}\selectfont

>>> from pyrats.tgyro.data import TGYROData
\addtolength{\parskip}{-\baselineskip}

>>> help(TGYROData)

\addtolength{\parskip}{\baselineskip}
\fontfamily{\rmdefault}\selectfont
\noindent This will bring up the built-in documentation for the class TGYROData.  It lists all of the methods and attributes of the class, and the function of each one.  This information is also contained on page x of this manual.

\noindent TGYROData can be used to load TGYRO data with the command:
\fontfamily{\ttdefault}\selectfont

>>> sim1 = TGYROData('example\_directory')

\fontfamily{\rmdefault}\selectfont
\noindent where 'example\_directory' is a directory containing TGYRO output files.  The data is loaded into objects corresponding to the output file which the data came from.  To create a basic plot of the particle densities and temperatures as a function of radius, type the following commands:
\fontfamily{\ttdefault}\selectfont

\addtolength{\parskip}{-\baselineskip}

\begin{verbatim}
>>> import matplotlib.pyplot as plt
>>> fig = plt.figure(1)
>>> ax1 = fig.add_subplot(221)
>>> ax2 = fig.add_subplot(222)
>>> ax3 = fig.add_subplot(223)
>>> ax4 = fig.add_subplot(224)
>>> ax1.plot(sim1.get_r(), sim1.get_Te)
>>> ax2.plot(sim1.get_r(), sim1.get_Ti)
>>> ax3.plot(sim1.get_r(), sim1.get_ne)
>>> ax4.plot(sim1.get_r(), sim1.get_ni)
>>> ax1.set_xlabel('r/a')
>>> ax1.set_ylabel('Electron Temperature (keV)')
>>> ax2.set_xlabel('r/a')
>>> ax2.set_ylabel('Ion Temperature (keV)')
>>> ax3.set_xlabel('r/a')
>>> ax3.set_ylabel('Electron Particle Density (e^19/m^3)')
>>> ax4.set_xlabel('r/a')
>>> ax4.set_ylabel('Ion Particle Density (e^19/m^3)')
>>> plt.show()
\end{verbatim}

\addtolength{\parskip}{\baselineskip}
\fontfamily{\rmdefault}\selectfont
\newpage
profiles\_genData contents:
\begin{itemize}
\item Data:
\begin{itemize}
\item \textbf{data} \emph{dictionary of numpy arrays} - Contains all of the data read in by profiles\_genData.  It is organized in a dictionary with the keys being the different column headers in the input.profiles file that is read.  The keys for the dictionary can be requested with the following commands:
\fontfamily{\ttdefault}\selectfont

\$ python

>>> from pyrats.profiles\_gen.data import profiles\_genData

>>> sim1 = profiles\_genData('example\_directory')

>>> sim1.data.keys()

\fontfamily{\rmdefault}\selectfont
\noindent where, again, 'example\_directory' is a directory containing the file input.profiles.
\item \textbf{n\_exp} \emph{int} - The number of timesteps in the simulation.  It is also the length of each array in data.
\item \textbf{hlen} \emph{int} - The length of the header of file input.profiles in rows.
\item \textbf{fignum} \emph{int} - The number of the current active matplotlib figure.
\item \textbf{plotcounter} \emph{int} - The number of the current active axes on the current active matplotlib figure.
\item \textbf{ar} \emph{nested list of floats} - The sine coefficients for the r-component of the fourier series representation of the flux surfaces.  Read from input.profiles.geo.
\item \textbf{br} \emph{nested list of floats} - The cosine coefficients for the r-component of the fourier series representation of the flux surfaces.  Read from input.profiles.geo.
\item \textbf{az} \emph{nested list of floats} - The sine coefficients for the z-component of the fourier series representation of the flux surfaces.  Read from input.profiles.geo.
\item \textbf{bz} \emph{nested list of floats} - The cosine coefficients for the z-component of the fourier series representation of the flux surfaces.  Read from input.profiles.geo.
\item \textbf{directory\_name} \emph{str} - The name of the directory containing file input.profiles and possibly input.profiles.geo.
\end{itemize}
\item Methods:
\begin{itemize}
\item \textbf{\_\_init\_\_(}\emph{str}\textbf{ directory='.')} - This is the constructor for the class.  When a new instance of the class is created, the constructor is called and executed.  The only argument is the name of the directory from which to build the class, and it defaults to the current directory.  It calls these methods in the following order:
\begin{enumerate}
\item set\_directory(\emph{str} directory)
\item init\_data()
\item store\_data()
\end{enumerate}
\item \textbf{compplot(}\emph{float}\textbf{ inner, }\emph{float}\textbf{ outer, }\emph{int}\textbf{ n, }\emph{bool}\textbf{ verbose)} - This method creates overlaid flux surface plots using both the fourier series decomposition method and the shaped Grad-Shafranov Miller-type equilibrium for the flux surfaces.
\begin{itemize}
\item Inner specifies the innermost flux surface to be plotted
\item outer specifies the outermost flux surface to be plotted
\item n specifies the number of flux surfaces to plot in between inner and outer
\item if verbose is True, the legend will display the location of each flux surface
\end{itemize}
\item \textbf{compute\_fouriereq(}\emph{float}\textbf{ r)} - This method calculates the flux surface at radius r according to the general Grad-Shafranov Fourier-series equilibrium.
\item \textbf{compute\_mtypeeq(}\emph{float}\textbf{ r)} - This method calculates the flux surface at radius r according to the shaped Grad-Shafranov Miller-type equilibrium.
\item \textbf{fourierplot(}\emph{float}\textbf{ inner, }\emph{float}\textbf{ outer, }\emph{int}\textbf{ n, }\emph{bool}\textbf{ verbose)} - This method creates flux surface plots using only the fourier series decomposition method for the flux surfaces.
\begin{itemize}
\item inner specifies the innermost flux surface to be plotted
\item outer specifies the outermost flux surface to be plotted
\item n specifies the number of flux surfaces to plot in between inner and outer
\item if verbose is True, the legend will display the location of each flux surface
\end{itemize}
\item \textbf{get(}\emph{str}\textbf{ var)} - This method returns the numpy array corresponding to var.
\item \textbf{init\_data()} - This method initializes all of the data objects.
\item \textbf{match(}\emph{float}\textbf{ val, }\emph{list}\textbf{ vec)} - This method finds the closest match to val in a \emph{list} of values (vec) and returns the index of that value.
\item \textbf{millerplot(}\emph{float}\textbf{ inner, }\emph{float}\textbf{ outer, }\emph{int}\textbf{ n, }\emph{bool}\textbf{ verbose)} - This method creates flux surface plots using only the shaped Grad-Shafranov Miller-type equilibrium for the flux surfaces.
\begin{itemize}
\item inner specifies the innermost flux surface to be plotted
\item outer specifies the outermost flux surface to be plotted
\item n specifies the number of flux surfaces to plot in between inner and outer
\item if verbose is True, the legend will display the location of each flux surface.
\end{itemize}
\item \textbf{plot(}\emph{str}\textbf{ var, }\emph{int}\textbf{ n1=2, }\emph{int}\textbf{ n2=2, }\emph{int}\textbf{ plotcounter=0, }\emph{int}\textbf{ fignum=0)} - This method creates plots of the requested data (var) using matplotlib.
\begin{itemize}
\item n1 is the horizontal number of plots in one window
\item n2 is the vertical number of plots in one window
\item plotcounter is the position on which the new graph is to be placed
\item fignum is the number of the matplotlib figure on which to place the new graph
\end{itemize}
\item \textbf{read\_data()} - This method reads in data from input.profiles.  It returns a dictionary containing the data that was read in.
\item \textbf{read\_fourier()} - This method reads in data from input.profiles.geo, and stores that data in the class objects ar, br, az, and bz.
\item \textbf{set\_directory(}\emph{str}\textbf{ directory)} - This method sets the class attribute directory\_name to directory.
\item \textbf{store\_data()} - This method reads data and renames it appropriately.  It is necessary because the names of the differetn parameters are not uniformly formatted.  store\_data cleans them up by inserting spaces where necessary, and by deleting \#-signs when necessary.
\end{itemize}
\end{itemize}

\newpage

NEOData contents:
\begin{itemize}
\item Data:
\begin{itemize}
\item \textbf{master} \emph{str} - Holds the name of the master directory containing NEO output files.
\item \textbf{directory\_name} \emph{str} - Holds the name of the currently open directory.
\item \textbf{fignum} \emph{int} - The number of the current active matplotlib figure.
\item \textbf{plotcounter} \emph{int} - The number of the current active axes on the current active matplotlib figure.
\item \textbf{toplot} \emph{list} - The list of variables to plot when a plot command is called.
\item \textbf{transport} \emph{dictionary of dictionaries of NEOObjects} - Contains data read in from out.neo.transport.  The keys are the transport variables, each of which correspond to a NEOObject.
\item \textbf{HH\_theory} \emph{dictionary of dictionaries of NEOObjects} - Contains data read in from out.neo.theory.  The keys are the flows and fluxes predicted by the Hinton-Hazeltine model, each of which correspond to a NEOObject.
\item \textbf{CH\_theory} \emph{dictionary of dictionaries of NEOObjects} - Contains data read in from out.neo.theory.  The keys are the ion heat fluxes predicted by the Chang-Hinton model, each of which correspond to a NEOObject.
\item \textbf{TG\_theory} \emph{dictionary of dictionaries of NEOObjects} - Contains data read in from out.neo.theory.  The keys are the ion heat fluxes predicted by the Taguchi model, each of which correspond to a NEOObject.
\item \textbf{S\_theory} \emph{dictionary of dictionaries of NEOObjects} - Contains data read in from out.neo.theory. The keys are the bootstrap currents predicted by the Sauter model, each of which correspond to a NEOObject.
\item \textbf{HS\_theory} \emph{dictionary of dictionaries of NEOObjects} - Contains data read in from out.neo.theory.  The keys are the fluxes predicted by the Hirshman-Sigmar model, each of which  correspond to a NEOObject.
\item \textbf{control} \emph{dictionary of dictionaries of NEOObjects} - Contains data read in from out.neo.control.
\end{itemize}
\item Methods:
\begin{itemize}
\item \textbf{\_\_init\_\_(}\emph{str} \textbf{sim\_directory)} - Constructor that is executed when a new NEOData object is created.  It takes a directory name as its argument, and then executes a top down walk down that directory.  It then calls read\_data() whenever it is in a subdirectory with NEO output files.  Finally, it executes store\_data()
\item \textbf{get\_CH\_theory(}\emph{str} \textbf{var)} - Returns a NEOObject object that corresponds to the Chang-Hinton theory prediction of var.  If requested variable does not exist or is zero everywhere, returns None.
\item \textbf{get\_HH\_theory(}\emph{str} \textbf{var)} - Returns a NEOObject object that corresponds to the Hinton-Hazeltine theory prediction of var.  If requested variable does not exist or is zero everywhere, returns None.
\item \textbf{get\_HR\_theory(}\emph{str} \textbf{var)} - Returns a NEOObject object that corresponds to the Hinton-Rosenbluth theory prediction of var.  If requested variable does not exist or is zero everywhere, returns None.
\item \textbf{get\_HS\_theory(}\emph{str} \textbf{var)} - Returns a NEOObject object that corresponds to the Hirshman-Sigmar theory prediction of var.  If requested variable does not exist or is zero everywhere, returns None.
\item \textbf{get\_S\_theory(}\emph{str} \textbf{var)} - Returns a NEOObject object that corresponds to the Sauter theory prediction of var.  If requested variable does not exist or is zero everywhere, returns None.
\item \textbf{get\_TG\_theory(}\emph{str} \textbf{var)} - Returns a NEOObject object that corresponds to the Taguchi theory prediction of var.  If requested variable does not exist or is zero everywhere, returns None.
\item \textbf{get\_control(}\emph{str} \textbf{var)} - Returns a NEOObject object that corresponds to the control parameter associated with var.  If requested variable does not exist or is zero everywhere, returns None.
\item \textbf{get\_input(}\emph{str} \textbf{input\_name)} - Returns requested variable from input.neo.gen.
\item \textbf{init\_data()} - Initializes object data.
\item \textbf{plot(}\emph{str} \textbf{var, }\emph{int} \textbf{n1=2, }\emph{int} \textbf{n2=2,} \emph{int} \textbf{plotcounter=0,} \emph{int} \textbf{fignum=0,} \emph{bool} \textbf{legend=True,} \emph{bool} \textbf{verbose=False,} \emph{str} \textbf{cols='bgkcmyrw',} \emph{list of strs} \textbf{styles=['-', '--', '-.', ':'])} - Plots var as a matplotlib scatter plot with data from different directories coming in different colors, and different species coming in different line styles.  Automatically searches for both theoretical and simulated values, and plots everything that is available.
\item \textbf{print\_vars()} - Prints all available simulated variables.
\item \textbf{read\_data()} - Read in object data.  Calls read\_grid(), read\_equil(), read\_theory(), read\_transport(), read\_transport\_gv().
\item \textbf{read\_equil()} - Reads out.neo.equil.  The data is eventually stored by store\_data() in control.
\item \textbf{read\_file()} - Loads data from NEO output file into buffer for manipulation and storage.
\item \textbf{read\_grid()} - Reads out.neo.grid.  The data is eventually stored by store\_data() in control.
\item \textbf{read\_theory()} - Reads out.neo.theory.  The data is eventually stored by store\_data() in HH\_theory, TG\_theory, CH\_theory, S\_theory, HR\_theory, and HS\_theory.
\item \textbf{read\_transport()} - Reads out.neo.transport.  The data is eventually stored by store\_data() in transport.
\item \textbf{read\_transport\_gv()} - Reads out.neo.transport\_gv.  The data is eventually stored by store\_data() in transport.
\item \textbf{set\_directory(}\emph{str} \textbf{path)} - Sets the current directory to path.
\item \textbf{split(}\emph{list} \textbf{array)} - Takes a 2-D list which has entries with multiple elements and converts it to a 2-D list with only one element per entry.
\item \textbf{store\_data()} - Stores data into data dictionaries by variable name and directory.  Data can be accessed with two dictionary keys, like so:

sim1.transport[parameter][directory].
\end{itemize}
\end{itemize}

\newpage
TGYROData contents:
\begin{itemize}
\item Data:
\begin{itemize}
\item \textbf{n\_iterations} \emph{int} - Number of TGYRO iterations in simulation.
\item \textbf{n\_fields} \emph{int} - Number of fields in simulation.
\item \textbf{n\_radial} \emph{int} - Number of radial gridpoints in simulation.
\item \textbf{directory\_name} \emph{str} - Name of loaded directory.
\item \textbf{chi\_e} \emph{dictionary of lists of numpy arrays} - Contains the data contained in the file chi\_e.out.  The column headers form the keys of the dictionary, with the iterations corresponding to the indicies of the lists.  The entries of each list are numpy arrays containing the data.
\item \textbf{chi\_i} \emph{dictionary of lists of numpy arrays} - Contains the data contained in the file chi\_i.out.  The column headers form the keys of the dictionary, with the iterations corresponding to the indicies of the lists.  The entries of each list are numpy arrays containing the data.
\item \textbf{chi\_i2} \emph{dictionary of lists of numpy arrays} - Contains the data contained in the file chi\_i2.out.  The column headers form the keys of the dictionary, with the iterations corresponding to the indicies of the lists.  The entries of each list are numpy arrays containing the data.
\item \textbf{chi\_i3} \emph{dictionary of lists of numpy arrays} - Contains the data contained in the file chi\_i3.out.  The column headers form the keys of the dictionary, with the iterations corresponding to the indicies of the lists.  The entries of each list are numpy arrays containing the data.
\item \textbf{chi\_i4} \emph{dictionary of lists of numpy arrays} - Contains the data contained in the file chi\_i4.out.  The column headers form the keys of the dictionary, with the iterations corresponding to the indicies of the lists.  The entries of each list are numpy arrays containing the data.
\item \textbf{chi\_i5} \emph{dictionary of lists of numpy arrays} - Contains the data contained in the file chi\_i5.out.  The column headers form the keys of the dictionary, with the iterations corresponding to the indicies of the lists.  The entries of each list are numpy arrays containing the data.
\item \textbf{gyro\_bohm\_unit} \emph{dictionary of lists of numpy arrays} - Contains the data contained in the file gyrobohm.out.  The column headers form the keys of the dictionary, with the iterations corresponding to the indicies of the lists.  The entries of each list are numpy arrays containing the data.
\item \textbf{profile} \emph{dictionary of lists of numpy arrays} - Contains the data contained in the file profile.out.  The column headers form the keys of the dictionary, with the iterations corresponding to the indicies of the lists.  The entries of each list are numpy arrays containing the data.
\item \textbf{geometry} \emph{dictionary of lists of numpy arrays} - Contains the data contained in the file geometry.out.  The column headers form the keys of the dictionary, with the iterations corresponding to the indicies of the lists.  The entries of each list are numpy arrays containing the data.
\item \textbf{flux\_e} \emph{dictionary of lists of numpy arrays} - Contains the data contained in the file flux\_e.out.  The column headers form the keys of the dictionary, with the iterations corresponding to the indicies of the lists.  The entries of each list are numpy arrays containing the data.
\item \textbf{flux\_i} \emph{dictionary of lists of numpy arrays} - Contains the data contained in the file flux\_i.out.  The column headers form the keys of the dictionary, with the iterations corresponding to the indicies of the lists.  The entries of each list are numpy arrays containing the data.
\item \textbf{flux\_i2} \emph{dictionary of lists of numpy arrays} - Contains the data contained in the file flux\_i2.out.  The column headers form the keys of the dictionary, with the iterations corresponding to the indicies of the lists.  The entries of each list are numpy arrays containing the data.
\item \textbf{flux\_i3} \emph{dictionary of lists of numpy arrays} - Contains the data contained in the file flux\_i3.out.  The column headers form the keys of the dictionary, with the iterations corresponding to the indicies of the lists.  The entries of each list are numpy arrays containing the data.
\item \textbf{flux\_i4} \emph{dictionary of lists of numpy arrays} - Contains the data contained in the file flux\_i4.out.  The column headers form the keys of the dictionary, with the iterations corresponding to the indicies of the lists.  The entries of each list are numpy arrays containing the data.
\item \textbf{flux\_i5} \emph{dictionary of lists of numpy arrays} - Contains the data contained in the file flux\_i5.out.  The column headers form the keys of the dictionary, with the iterations corresponding to the indicies of the lists.  The entries of each list are numpy arrays containing the data.
\item \textbf{flux\_target} \emph{dictionary of lists of numpy arrays} - Contains the data contained in the file flux\_target.out.  The column headers form the keys of the dictionary, with the iterations corresponding to the indicies of the lists.  The entries of each list are numpy arrays containing the data.
\item \textbf{mflux\_e} \emph{dictionary of lists of numpy arrays} - Contains the data contained in the file mflux\_e.out.  The column headers form the keys of the dictionary, with the iterations corresponding to the indicies of the lists.  The entries of each list are numpy arrays containing the data.
\item \textbf{mflux\_i} \emph{dictionary of lists of numpy arrays} - Contains the data contained in the file mflux\_i.out.  The column headers form the keys of the dictionary, with the iterations corresponding to the indicies of the lists.  The entries of each list are numpy arrays containing the data.
\item \textbf{mflux\_i2} \emph{dictionary of lists of numpy arrays} - Contains the data contained in the file mflux\_i2.out.  The column headers form the keys of the dictionary, with the iterations corresponding to the indicies of the lists.  The entries of each list are numpy arrays containing the data.
\item \textbf{mflux\_i3} \emph{dictionary of lists of numpy arrays} - Contains the data contained in the file mflux\_i3.out.  The column headers form the keys of the dictionary, with the iterations corresponding to the indicies of the lists.  The entries of each list are numpy arrays containing the data.
\item \textbf{mflux\_i4} \emph{dictionary of lists of numpy arrays} - Contains the data contained in the file mflux\_i4.out.  The column headers form the keys of the dictionary, with the iterations corresponding to the indicies of the lists.  The entries of each list are numpy arrays containing the data.
\item \textbf{mflux\_i5} \emph{dictionary of lists of numpy arrays} - Contains the data contained in the file mflux\_i5.out.  The column headers form the keys of the dictionary, with the iterations corresponding to the indicies of the lists.  The entries of each list are numpy arrays containing the data.
\item \textbf{mflux\_target} \emph{dictionary of lists of numpy arrays} - Contains the data contained in the file mflux\_target.out.  The column headers form the keys of the dictionary, with the iterations corresponding to the indicies of the lists.  The entries of each list are numpy arrays containing the data.
\item \textbf{gradient} \emph{dictionary of lists of numpy arrays} - Contains the data contained in the file gradient.out.  The column headers form the keys of the dictionary, with the iterations corresponding to the indicies of the lists.  The entries of each list are numpy arrays containing the data.
\item \textbf{local\_res} \emph{dictionary of lists of numpy arrays} - Contains the data contained in the file local\_res.out.  The column headers form the keys of the dictionary, with the iterations corresponding to the indicies of the lists.  The entries of each list are numpy arrays containing the data.
\item \textbf{global\_res} \emph{dictionary of lists of numpy arrays} - Contains the data contained in the file global\_res.out.  The column headers form the keys of the dictionary, with the iterations corresponding to the indicies of the lists.  The entries of each list are numpy arrays containing the data.
\item \textbf{wr\_ion} \emph{dictionary of lists of numpy arrays} - Contains the data contained in the file wr\_ion.out.  The column headers form the keys of the dictionary, with the iterations corresponding to the indicies of the lists.  The entries of each list are numpy arrays containing the data.
\item \textbf{wi\_ion} \emph{dictionary of lists of numpy arrays} - Contains the data contained in the file wi\_ion.out.  The column headers form the keys of the dictionary, with the iterations corresponding to the indicies of the lists.  The entries of each list are numpy arrays containing the data.
\item \textbf{wr\_elec} \emph{dictionary of lists of numpy arrays} - Contains the data contained in the file wr\_elec.out.  The column headers form the keys of the dictionary, with the iterations corresponding to the indicies of the lists.  The entries of each list are numpy arrays containing the data.
\item \textbf{wr\_ion} \emph{dictionary of lists of numpy arrays} - Contains the data contained in the file wr\_ion.out.  The column headers form the keys of the dictionary, with the iterations corresponding to the indicies of the lists.  The entries of each list are numpy arrays containing the data.
\item \textbf{r} \emph{numpy array} - Contains the normalized radial gridpoints in a 1-D numpy array.
\end{itemize}
\item Methods:
\begin{itemize}
\item \textbf{\_\_init\_\_(}\emph{str} \textbf{ sim\_directory)} - Constructor which is called when a new TRYGOData object is created.  It calls the following methods:
\begin{enumerate}
\item set\_directory(\emph{str} sim\_directory)
\item init\_data()
\item read\_data()
\end{enumerate}
\item \textbf{get\_Te(}\emph{int}\textbf{ iteration=-1)} - Returns a numpy array of the electron temperature for the specified TGYRO iteration.  The default is the last iteration.
\item \textbf{get\_Ti(}\emph{int}\textbf{ iteration=-1)} - Returns a numpy array of the species 1 ion temperature for the specified TGYRO iteration.  The default is the last iteration.
\item \textbf{get\_Ti2(}\emph{int}\textbf{ iteration=-1)} - Returns a numpy array of the species 2 ion temperature for the specified TGYRO iteration.  The default is the last iteration.
\item \textbf{get\_Ti3(}\emph{int}\textbf{ iteration=-1)} - Returns a numpy array of the species 3 ion temperature for the specified TGYRO iteration.  The default is the last iteration.
\item \textbf{get\_Ti4(}\emph{int}\textbf{ iteration=-1)} - Returns a numpy array of the species 4 ion temperature for the specified TGYRO iteration.  The default is the last iteration.
\item \textbf{get\_Ti5(}\emph{int}\textbf{ iteration=-1)} - Returns a numpy array of the species 5 ion temperature for the specified TGYRO iteration.  The default is the last iteration.
\item \textbf{get\_chi\_e\_neo(}\emph{int}\textbf{ iteration=-1,}\emph{ bool}\textbf{ mks=False)} - Returns a numpy array of the neoclassical chi\_e from the specified iteration.  The default is the last iteration.  If mks is set to True, the results will be given in mks units.
\item \textbf{get\_chi\_e\_turb(}\emph{int}\textbf{ iteration=-1,}\emph{ bool}\textbf{ mks=False)} - Returns a numpy array of the turbulent chi\_e from the specified iteration.  The default is the last iteration.  If mks is set to True, the results will be given in mks units.
\item \textbf{get\_chi\_i\_neo(}\emph{int}\textbf{ iteration=-1,}\emph{ bool}\textbf{ mks=False)} - Returns a numpy array of the neoclassical chi\_i from the specified iteration.  The default is the last iteration.  If mks is set to True, the results will be given in mks units.
\item \textbf{get\_chi\_i\_turb(}\emph{int}\textbf{ iteration=-1,}\emph{ bool}\textbf{ mks=False)} - Returns a numpy array of the turbulent chi\_i from the specified iteration.  The default is the last iteration.  If mks is set to True, the results will be given in mks units.
\item \textbf{get\_chi\_i2\_neo(}\emph{int}\textbf{ iteration=-1,}\emph{ bool}\textbf{ mks=False)} - Returns a numpy array of the neoclassical chi\_i2 from the specified iteration.  The default is the last iteration.  If mks is set to True, the results will be given in mks units.
\item \textbf{get\_chi\_i2\_turb(}\emph{int}\textbf{ iteration=-1,}\emph{ bool}\textbf{ mks=False)} - Returns a numpy array of the turbulent chi\_i2 from the specified iteration.  The default is the last iteration.  If mks is set to True, the results will be given in mks units.
\item \textbf{get\_chi\_i3\_neo(}\emph{int}\textbf{ iteration=-1,}\emph{ bool}\textbf{ mks=False)} - Returns a numpy array of the neoclassical chi\_i3 from the specified iteration.  The default is the last iteration.  If mks is set to True, the results will be given in mks units.
\item \textbf{get\_chi\_i3\_turb(}\emph{int}\textbf{ iteration=-1,}\emph{ bool}\textbf{ mks=False)} - Returns a numpy array of the turbulent chi\_i3 from the specified iteration.  The default is the last iteration.  If mks is set to True, the results will be given in mks units.
\item \textbf{get\_chi\_i4\_neo(}\emph{int}\textbf{ iteration=-1,}\emph{ bool}\textbf{ mks=False)} - Returns a numpy array of the neoclassical chi\_i4 from the specified iteration.  The default is the last iteration.  If mks is set to True, the results will be given in mks units.
\item \textbf{get\_chi\_i4\_turb(}\emph{int}\textbf{ iteration=-1,}\emph{ bool}\textbf{ mks=False)} - Returns a numpy array of the turbulent chi\_i4 from the specified iteration.  The default is the last iteration.  If mks is set to True, the results will be given in mks units.
\item \textbf{get\_chi\_i5\_neo(}\emph{int}\textbf{ iteration=-1,}\emph{ bool}\textbf{ mks=False)} - Returns a numpy array of the neoclassical chi\_i5 from the specified iteration.  The default is the last iteration.  If mks is set to True, the results will be given in mks units.
\item \textbf{get\_chi\_i5\_turb(}\emph{int}\textbf{ iteration=-1,}\emph{ bool}\textbf{ mks=False)} - Returns a numpy array of the turbulent chi\_i5 from the specified iteration.  The default is the last iteration.  If mks is set to True, the results will be given in mks units.
\item \textbf{get\_flux\_count(}\emph{int}\textbf{ iteration=-1)} - Returns the total number of calls to the flux driver up to the given iteration.
\item \textbf{get\_flux\_e\_neo(}\emph{int}\textbf{ iteration=-1,}\emph{ bool}\textbf{ mks=False)} - Returns a numpy array of the neoclassical electron heat flux from the specified iteration.  The default is the last iteration.  If mks is set to True, the results will be given in mks units.
\item \textbf{get\_flux\_e\_target(}\emph{int}\textbf{ iteration=-1,}\emph{ bool}\textbf{ mks=False)} - Returns a numpy array of the target electron heat flux from the specified iteration.  The default is the last iteration.  If mks is set to True, the results will be given in mks units.
\item \textbf{get\_flux\_e\_turb(}\emph{int}\textbf{ iteration=-1,}\emph{ bool}\textbf{ mks=False)} - Returns a numpy array of the turbulent electron heat flux from the specified iteration.  The default is the last iteration.  If mks is set to True, the results will be given in mks units.
\item \textbf{get\_flux\_i\_neo(}\emph{int}\textbf{ iteration=-1,}\emph{ bool}\textbf{ mks=False)} - Returns a numpy array of the neoclassical ion heat flux from the specified iteration.  The default is the last iteration.  If mks is set to True, the results will be given in mks units.
\item \textbf{get\_flux\_i\_target(}\emph{int}\textbf{ iteration=-1,}\emph{ bool}\textbf{ mks=False)} - Returns a numpy array of the target ion heat flux from the specified iteration.  The default is the last iteration.  If mks is set to True, the results will be given in mks units.
\item \textbf{get\_flux\_i\_turb(}\emph{int}\textbf{ iteration=-1,}\emph{ bool}\textbf{ mks=False)} - Returns a numpy array of the turbulent ion heat flux from the specified iteration.  The default is the last iteration.  If mks is set to True, the results will be given in mks units.
\item \textbf{get\_flux\_i2\_neo(}\emph{int}\textbf{ iteration=-1,}\emph{ bool}\textbf{ mks=False)} - Returns a numpy array of the neoclassical ion species 2 heat flux from the specified iteration.  The default is the last iteration.  If mks is set to True, the results will be given in mks units.
\item \textbf{get\_flux\_i2\_target(}\emph{int}\textbf{ iteration=-1,}\emph{ bool}\textbf{ mks=False)} - Returns a numpy array of the target ion species 2 heat flux from the specified iteration.  The default is the last iteration.  If mks is set to True, the results will be given in mks units.
\item \textbf{get\_flux\_i2\_turb(}\emph{int}\textbf{ iteration=-1,}\emph{ bool}\textbf{ mks=False)} - Returns a numpy array of the turbulent ion species 2 heat flux from the specified iteration.  The default is the last iteration.  If mks is set to True, the results will be given in mks units.
\item \textbf{get\_flux\_i3\_neo(}\emph{int}\textbf{ iteration=-1,}\emph{ bool}\textbf{ mks=False)} - Returns a numpy array of the neoclassical ion species 3 heat flux from the specified iteration.  The default is the last iteration.  If mks is set to True, the results will be given in mks units.
\item \textbf{get\_flux\_i3\_target(}\emph{int}\textbf{ iteration=-1,}\emph{ bool}\textbf{ mks=False)} - Returns a numpy array of the target ion species 4 heat flux from the specified iteration.  The default is the last iteration.  If mks is set to True, the results will be given in mks units.
\item \textbf{get\_flux\_i3\_turb(}\emph{int}\textbf{ iteration=-1,}\emph{ bool}\textbf{ mks=False)} - Returns a numpy array of the turbulent ion species 3 heat flux from the specified iteration.  The default is the last iteration.  If mks is set to True, the results will be given in mks units.
\item \textbf{get\_flux\_i4\_neo(}\emph{int}\textbf{ iteration=-1,}\emph{ bool}\textbf{ mks=False)} - Returns a numpy array of the neoclassical ion species 4 heat flux from the specified iteration.  The default is the last iteration.  If mks is set to True, the results will be given in mks units.
\item \textbf{get\_flux\_i4\_target(}\emph{int}\textbf{ iteration=-1,}\emph{ bool}\textbf{ mks=False)} - Returns a numpy array of the target ion species 4 heat flux from the specified iteration.  The default is the last iteration.  If mks is set to True, the results will be given in mks units.
\item \textbf{get\_flux\_i4\_turb(}\emph{int}\textbf{ iteration=-1,}\emph{ bool}\textbf{ mks=False)} - Returns a numpy array of the turbulent ion species 4 heat flux from the specified iteration.  The default is the last iteration.  If mks is set to True, the results will be given in mks units.
\item \textbf{get\_flux\_i5\_neo(}\emph{int}\textbf{ iteration=-1,}\emph{ bool}\textbf{ mks=False)} - Returns a numpy array of the neoclassical ion species 5 heat flux from the specified iteration.  The default is the last iteration.  If mks is set to True, the results will be given in mks units.
\item \textbf{get\_flux\_i5\_target(}\emph{int}\textbf{ iteration=-1,}\emph{ bool}\textbf{ mks=False)} - Returns a numpy array of the target ion species 5 heat flux from the specified iteration.  The default is the last iteration.  If mks is set to True, the results will be given in mks units.
\item \textbf{get\_flux\_i5\_turb(}\emph{int}\textbf{ iteration=-1,}\emph{ bool}\textbf{ mks=False)} - Returns a numpy array of the turbulent ion species 5 heat flux from the specified iteration.  The default is the last iteration.  If mks is set to True, the results will be given in mks units.
\item \textbf{get\_geometry\_factor(}\emph{str}\textbf{ factor,}\emph{ int}\textbf{ iteration=-1)} - Returns a numpy array of the specified geometric factor for the specified iteration.  The default is the last iteration.  Possible factors are:
\begin{itemize}
\item 'rho'
\item 'q'
\item 's'
\item 'kappa'
\item 's\_kappa'
\item 'delta'
\item 's\_delta'
\item 'shift'
\item 'rmaj/a'
\item 'b\_unit'
\end{itemize}
\item \textbf{get\_global\_res(}\emph{int}\textbf{ iteration=-1)} - Returns a numpy array of the global residual from the specified iteration.  The default is the last iteration.
\item \textbf{get\_gradient\_factor(}\emph{str}\textbf{ factor,}\emph{ int}\textbf{ iteration=-1)} - Returns a numpy array of the specified gradient factor for the given iteration.  The default iteration is the last.  Possible factors are:
\begin{itemize}
\item 'r/a'
\item 'a/Lni'
\item 'a/Lne'
\item 'a/LTi'
\item 'a/LTe'
\item 'a/Lp'
\item 'a*gamma\_e/cs'
\item 'a*gamma\_p/cs'
\end{itemize}
\item \textbf{get\_input(}\emph{str}\textbf{ input\_name)} - Returns the specified variable from input.tgyro.gen.

Ex: get\_input("TGYRO\_MODE")
\item \textbf{get\_local\_res(}\emph{int}\textbf{ field=1,}\emph{ int}\textbf{ iteration=-1)} - Returns a numpy array of the local field residual for the given field and iteration.  The default is the last iteration, and field 1.
\item \textbf{get\_mflux\_e\_neo(}\emph{int}\textbf{ iteration=-1,}\emph{ bool}\textbf{ mks=False)} - Returns a numpy array of the neoclassical electron momentum flux from the specified iteration.  The default iteration is the last.  If mks is set to True, the data will be returned in mks units.
\item \textbf{get\_mflux\_e\_turb(}\emph{int}\textbf{ iteration=-1,}\emph{ bool}\textbf{ mks=False)} - Returns a numpy array of the turbulent electron momentum flux from the specified iteration.  The default iteration is the last.  If mks is set to True, the data will be returned in mks units.
\item \textbf{get\_mflux\_i\_neo(}\emph{int}\textbf{ iteration=-1,}\emph{ bool}\textbf{ mks=False)} - Returns a numpy array of the neoclassical ion momentum flux from the specified iteration.  The default iteration is the last.  If mks is set to True, the data will be returned in mks units.
\item \textbf{get\_mflux\_i\_turb(}\emph{int}\textbf{ iteration=-1,}\emph{ bool}\textbf{ mks=False)} - Returns a numpy array of the turbulent ion momentum flux from the specified iteration.  The default iteration is the last.  If mks is set to True, the data will be returned in mks units.
\item \textbf{get\_mflux\_i2\_neo(}\emph{int}\textbf{ iteration=-1,}\emph{ bool}\textbf{ mks=False)} - Returns a numpy array of the neoclassical ion species 2 momentum flux from the specified iteration.  The default iteration is the last.  If mks is set to True, the data will be returned in mks units.
\item \textbf{get\_mflux\_i2\_turb(}\emph{int}\textbf{ iteration=-1,}\emph{ bool}\textbf{ mks=False)} - Returns a numpy array of the turbulent ion species 2 momentum flux from the specified iteration.  The default iteration is the last.  If mks is set to True, the data will be returned in mks units.
\item \textbf{get\_mflux\_i3\_neo(}\emph{int}\textbf{ iteration=-1,}\emph{ bool}\textbf{ mks=False)} - Returns a numpy array of the neoclassical ion species 3 momentum flux from the specified iteration.  The default iteration is the last.  If mks is set to True, the data will be returned in mks units.
\item \textbf{get\_mflux\_i3\_turb(}\emph{int}\textbf{ iteration=-1,}\emph{ bool}\textbf{ mks=False)} - Returns a numpy array of the turbulent ion species 3 momentum flux from the specified iteration.  The default iteration is the last.  If mks is set to True, the data will be returned in mks units.
\item \textbf{get\_mflux\_i4\_neo(}\emph{int}\textbf{ iteration=-1,}\emph{ bool}\textbf{ mks=False)} - Returns a numpy array of the neoclassical ion species 4 momentum flux from the specified iteration.  The default iteration is the last.  If mks is set to True, the data will be returned in mks units.
\item \textbf{get\_mflux\_i4\_turb(}\emph{int}\textbf{ iteration=-1,}\emph{ bool}\textbf{ mks=False)} - Returns a numpy array of the turbulent ion species 4 momentum flux from the specified iteration.  The default iteration is the last.  If mks is set to True, the data will be returned in mks units.
\item \textbf{get\_mflux\_i5\_neo(}\emph{int}\textbf{ iteration=-1,}\emph{ bool}\textbf{ mks=False)} - Returns a numpy array of the neoclassical ion species 5 momentum flux from the specified iteration.  The default iteration is the last.  If mks is set to True, the data will be returned in mks units.
\item \textbf{get\_mflux\_i5\_turb(}\emph{int}\textbf{ iteration=-1,}\emph{ bool}\textbf{ mks=False)} - Returns a numpy array of the turbulent ion species 5 momentum flux from the specified iteration.  The default iteration is the last.  If mks is set to True, the data will be returned in mks units.
\item \textbf{get\_mflux\_target(}\emph{int}\textbf{ iteration=-1,}\emph{ bool}\textbf{ mks=False)} - Returns a numpy array of the target momentum flux from the specified iteration.  The default iteration is the last.  If mks is set to True, the data will be returned in mks units.
\item \textbf{get\_most\_unstable\_at\_radius(}\emph{int}\textbf{ radius=0,}\emph{ str}\textbf{ direction='ion')} - Returns the most unstable mode at specified radius in the specified direction.  The parameter radius is the requested radial index; it is an integer between 0 and n\_r-1.  The parameter direction can be either 'ion' for the most unstable ion mode, or 'electron' for the most unstable electron mode.
\item \textbf{get\_ne(}\emph{int}\textbf{ iteration=-1)} - Returns a numpy array of ne for the specified iteration.  The default iteration is the last.
\item \textbf{get\_ni(}\emph{int}\textbf{ iteration=-1)} - Returns a numpy array of ni for the specified iteration.  The default iteration is the last.
\item \textbf{get\_ni2(}\emph{int}\textbf{ iteration=-1)} - Returns a numpy array of ni2 for the specified iteration.  The default iteration is the last.
\item \textbf{get\_ni3(}\emph{int}\textbf{ iteration=-1)} - Returns a numpy array of ni3 for the specified iteration.  The default iteration is the last.
\item \textbf{get\_ni4(}\emph{int}\textbf{ iteration=-1)} - Returns a numpy array of ni4 for the specified iteration.  The default iteration is the last.
\item \textbf{get\_ni5(}\emph{int}\textbf{ iteration=-1)} - Returns a numpy array of ni5 for the specified iteration.  The default iteration is the last.
\item \textbf{get\_profile\_factor(}\emph{str}\textbf{ factor,}\emph{ int}\textbf{ iteration=-1)} - Returns a numpy array of the specified profile factor for the given iteration.  The default iteration is the last.  The possible factors are:
\begin{itemize}
\item 'r/a'
\item 'ni'
\item 'a/Lni'
\item 'Ti'
\item 'a/LTi'
\end{itemize}
\item \textbf{get\_profile2\_factor(}\emph{str}\textbf{ factor,}\emph{ int}\textbf{ iteration=-1)} - Returns a numpy array of the specified profile2 factor for the given iteration.  The default iteration is the last.  The possible factors are the same as get\_profile\_factor.
\item \textbf{get\_profile3\_factor(}\emph{str}\textbf{ factor,}\emph{ int}\textbf{ iteration=-1)} - Returns a numpy array of the specified profile3 factor for the given iteration.  The default iteration is the last.  The possible factors are the same as get\_profile\_factor.
\item \textbf{get\_profile4\_factor(}\emph{str}\textbf{ factor,}\emph{ int}\textbf{ iteration=-1)} - Returns a numpy array of the specified profile4 factor for the given iteration.  The default iteration is the last.  The possible factors are the same as get\_profile\_factor.
\item \textbf{get\_profile5\_factor(}\emph{str}\textbf{ factor,}\emph{ int}\textbf{ iteration=-1)} - Returns a numpy array of the specified profile5 factor for the given iteration.  The default iteration is the last.  The possible factors are the same as get\_profile\_factor.
\item \textbf{get\_r()} - Returns a numpy array containing the normalized radial gridpoints, r/a.
\item \textbf{get\_stability\_at\_radius(}\emph{int}\textbf{ radius=0,}\emph{ str}\textbf{ frequency='r',}\emph{ str}\textbf{ direction='ion')} - Returns a list of frequency vs. ky at specified radius from stability analysis.  The parameter radius is the requested radial index; it is an integer between 0 and n\_r-1.  The parameter frequency can be either the real ('r') or the imaginary ('i') spectrum.  The parameter direction specifies the direction of the spectrum.  It can be either the ion direction ('ion') or the electron direction ('elec').
\item \textbf{init\_data()} - Initializes object data.
\item \textbf{make\_gradient\_vs\_field\_plot(}\emph{int}\textbf{ r=1,}\emph{ int}\textbf{ evolve\_field=1,}\emph{ str}\textbf{ grad='a/LTi',}\emph{ str}\textbf{ profile='ti',}\emph{ bool}\textbf{ arrows=False)} - Return a matplotlib.pyplot figure of gradient\_vs\_field\_space.  The parameter r is the radial index; it is an integer between 0 and n\_r-1.  The parameters are passed to make\_gradient\_vs\_field\_space, and are described there.
\item \textbf{make\_gradient\_vs\_field\_space(}\emph{int}\textbf{ r=1,}\emph{ int}\textbf{ evolve\_field=1,}\emph{ str}\textbf{ grad='a/LTi',}\emph{ str}\textbf{ profile='ti')} - Returns a matrix of Space[iteration][gradient,profile,residual] at givel radial point.  eeg: Space[[4.0, 1.7, 2.5], [3.6, 1.5, 0.1]] if grad 4.0 $\rightarrow$ 3.6 while Ti $\rightarrow$ 1.5 and local residual went from 2.5 $\rightarrow$ 0.1.  The parameter evolve\_field specifies the field type to pass to get\_local\_res.  The parameter grad is the gradient factor to pass to get\_gradient\_factor.  The parameter profile is the profile factor to pass to get\_profile\_factor.
\item \textbf{read\_chi\_e()} - Internal method.  This method reads in chi\_e.out and stores it in self.chi\_e.
\item \textbf{read\_chi\_i()} - Internal method.  This method reads in chi\_i.out and stores it in self.chi\_i.
\item \textbf{read\_control()} - Internal method.  This method reads in control.out to set resolutions.
\item \textbf{read\_data()} - Internal method.  This method reads in object data and calls the following other methods:
\begin{enumerate}
\item read\_control()
\item read\_chi\_e()
\item read\_chi\_i(loc\_n\_ion)
\item read\_gyrobohm()
\item read\_profile()
\item read\_geometry()
\item read\_flux(loc\_n\_ion)
\item read\_mflux(loc\_n\_ion)
\item read\_gradient()
\item read\_residual()
\end{enumerate}
\item \textbf{read\_file(}\emph{str}\textbf{ file\_name)} - Reads TGYRO output file.  Output is data['column\_header'][iteration]
\item \textbf{read\_flux(}\emph{int}\textbf{ num\_ions=1)} - Reads flux\_e.out, flux\_i(2-5).out, and flux\_target.out.  Num\_ions determines how many ion files to read.
\item \textbf{read\_geometry()} - Reads and stores geometry.out in self.geometry.
\item \textbf{read\_gradient()} - Reads and stores gradient.out in self.gradient.
\item \textbf{read\_gyrobohm()} - Reads and stores gyrobohm.out in self.gyro\_bohm\_unit.
\item \textbf{read\_mflux()} - Reads and stores mflux\_e.out, mflux\_i(2-5).out, and mflux\_target.out.
\item \textbf{read\_num\_ions()} - Reads LOC\_N\_ION and stores it as self.loc\_n\_ion.
\item \textbf{read\_profile()} - Reads and stores profile.out in self.profile.
\item \textbf{read\_profile2()} - Reads and stores profile2.out in self.profile2.
\item \textbf{read\_profile3()} - Reads and stores profile3.out in self.profile3.
\item \textbf{read\_profile4()} - Reads and stores profile4.out in self.profile4.
\item \textbf{read\_profile5()} - Reads and stores profile5.out in self.profile5.
\item \textbf{read\_residual()} - Reads residual.out.
\item \textbf{read\_stab\_file(}\emph{str}\textbf{ file\_name)} - Reads files generated with stability analysis mode.  The parameter file\_name is the name of the requested stability file, sich as "wi\_elec.out".  It returns a list of [r, ks, freq].
\item \textbf{read\_stabilities()} - Reads output files from TGYRO\_METHOD=2, and stores them into variables, as follows:
\begin{itemize}
\item wr\_ion.out $\rightarrow$ self.wr\_ion
\item wi\_ion.out $\rightarrow$ self.wi\_ion
\item wr\_elec.out $\rightarrow$ self.wr\_elec
\item wi\_elec.out $\rightarrow$ self.wi\_elec
\end{itemize}
\item \textbf{read\_tgyro\_mode()} - Reads the TGYRO\_MODE and stores it as self.tgyro\_mode.
\item \textbf{set\_directory(}\emph{str}\textbf{ sim\_directory)} - Sets the simulation directory.  Stores sim\_directory in

self.directory\_name.
\end{itemize}
\end{itemize}

\newpage
GYROData Contents:
\begin{itemize}
\item Data
\begin{itemize}
\item \textbf{directory\_name}\emph{ str} - The name of the simulation directory.
\item \textbf{profile}\emph{ dictionary of numpy arrays} - Dictionary where control information from out.gyro.profile is stored.  The keys can be obtained by typing: sim1.profile.keys()
\item \textbf{geometry}\emph{ dictionary of numpy arrays} - Dictionary where geometry data from out.gyro.geometry is stored.  The keys can be obtained by typing: sim1.geometry.keys()
\item \textbf{t}\emph{ dictionary of numpy arrays} - Dictionary where time data from t.out is stored.  The keys are: 't/deltat', '(cbar\_s/a)t', and 'n\_time' which is an int, not a numpy array.
\item \textbf{diff}\emph{ numpy array} - Numpy array with dimensions of (n\_kinetic, n\_field, moments=2, n\_time).  It contains the Gyrobohm-normalized diffusivities averaged over radius and summer over mode number.  The moments are:
\begin{enumerate}
\item $D_\sigma/\chi_{GB}$ (particle diffusivity)
\item $\chi_\sigma/\chi_{GB}$ (energy diffusivity)
\end{enumerate}
\item \textbf{diff\_i}\emph{ numpy array} - Numpy array with dimensions of (n\_kinetic, n\_field, moments=2, n\_x, n\_time).  It contains the Gyrobohm-normalized diffusivities as a function of radius, summed over mode number.  The moments are:
\begin{enumerate}
\item $D_\sigma(r)/\chi_{GB}$ (particle diffusivity)
\item $\chi_\sigma(r)/\chi_{GB}$ (energy diffusivity)
\end{enumerate}
\item \textbf{diff\_n}\emph{ numpy array} - Numpy array with dimensions of (n\_kinetic, n\_field, moments=2, n\_n, n\_time).  It contains the Gyrobohm-normalized diffusivities averaged over radius for each mode number.  The moments are:
\begin{enumerate}
\item $D_{\sigma,n}/\chi_{GB}$ (particle diffusivity)
\item $\chi_{\sigma,n}/\chi_{GB}$ (energy diffusivity)
\end{enumerate}
\item \textbf{gbflux}\emph{ numpy array} - Numpy array with dimensions of (n\_kinetic, n\_field, moments=4, n\_time).  It contains the Gyrobohm-normalized fluxes averaged over radius and summed over mode number.  The moments are:
\begin{enumerate}
\item $\Gamma_\sigma/\Gamma_{GB}$ (particle flux)
\item $Q_\sigma/Q_{GB}$ (energy flux)
\item $\Pi_\sigma/\Pi_{GB}$ (momentum flux)
\item $S^{tur}_{W,\sigma}/S_{GB}$ (exchange power density)
\end{enumerate}
\item \textbf{gbflux\_i}\emph{ numpy array} - Numpy array with dimensions of (n\_kinetic, n\_field, moments=4, n\_x, n\_time).  It contains the Gyrobohm-normalized fluxes as a function of mode number and averaged over radius.  The moments are:
\begin{enumerate}
\item $\Gamma_\sigma(r)/\Gamma_{GB}$ (particle flux)
\item $Q_\sigma(r)/Q_{GB}$ (energy flux)
\item $\Pi_\sigma(r)/\Pi_{GB}$ (momentum flux)
\item $S^{tur}_{W,\sigma}(r)/S_{GB}$ (exchange power density)
\end{enumerate}
\item \textbf{gbflux\_n}\emph{ numpy array} - Numpy array with dimensions of (n\_kinetic, n\_field, moments=4, n\_n, n\_time).  It contains the Gyrobohm-normalized fluxes as a function of mode number and averaged over radius.  The moments are:
\begin{enumerate}
\item $\Gamma_\sigma/\Gamma_{GB}$ (particle flux)
\item $Q_\sigma/Q_{GB}$ (energy flux)
\item $\Pi_\sigma/\Pi_{GB}$ (momentum flux)
\item $S^{tur}_{W,\sigma}/S_{GB}$ (exchange power density)
\end{enumerate}
\item \textbf{moment\_u}\emph{ numpy array} - Complex numpy array with dimensions of (n\_theta\_plot, n\_x, i\_field=n\_field, n\_n, n\_time).  It contains the potential expansion coefficients.  More information is available online at https://fusion.gat.com/theory/Gyrousermanual\#Computed\_Quantities.  The fields are described below.  There can be up to three different fields:
\begin{enumerate}
\item $\frac{e\delta\phi_n}{\bar{T}_e}$ (electrostatic potential)
\item $\frac{\bar{c}_s}{c}\frac{e\delta A_{\parallel n}}{\bar{T}_e}$ (electromagnetic potential)
\item $\frac{\delta B_\parallel}{B_{\mathrm{unit}}(r)}$ (compressional perturbation)
\end{enumerate}
\item \textbf{moment\_n}\emph{ numpy array} - Complex numpy array with dimensions of (n\_theta\_plot, n\_x, n\_kinetic, n\_n, n\_time).  It contains the density moment expansion coefficients: $\frac{\delta n_{\sigma,n}}{\bar{n}_e}$.  More information is available online at https://fusion.gat.com/theory/Gyrousermanual\#Computed\_Quantities.
\item \textbf{moment\_e}\emph{ numpy array} - Complex numpy array with dimensions of (n\_theta\_plot, n\_x, n\_kinetic, n\_n, n\_time).  It contains the energy moment expansion coefficients: $\frac{\delta E_{\sigma,n}}{\bar{n}_e\bar{T}_e}$.  More information is available online at https://fusion.gat.com/theory/Gyrousermanual\#Computed\_Quantities.
\item \textbf{moment\_v}\emph{ numpy array} - Complex numpy array with dimensions of (n\_theta\_plot, n\_x, n\_kinetic, n\_n, n\_time).  It contains the parallel velocity moment expansion coefficients: $\frac{\delta V_{\sigma,n}}{\bar{n}_e\bar{c}_s}$.  More information is available online at https://fusion.gat.com/theory/Gyrousermanual\#Computed\_Quantities.
\item \textbf{moment\_zero}\emph{ numpy array} - Numpy array with dimensions of (n\_x, n\_kinetic, moments=n\_moment, n\_time).  It contains the flux-surface average of the $n=0$ component of the density and energy moments.  The moments are described below.  There can be up to two different moments:
\begin{enumerate}
\item $\langle\frac{\delta n_{\sigma,0}}{\bar{n}_e}\rangle$
\item $\langle\frac{\delta E_{\sigma,0}}{\bar{n}_e\bar{T}_e}\rangle$
\end{enumerate}
\item \textbf{flux\_velocity}\emph{ numpy array} - Numpy array with dimensions of (n\_energy, n\_lambda, n\_kinetic, i\_field=n\_field, moments=2, n\_n, n\_time).  It contains the velocity-space flux densities: $$\Gamma=\int \! \dif \varepsilon \int \! \dif \lambda \, \Gamma(\varepsilon, \lambda), \, \, \, \, Q=\int \! \dif \varepsilon \int \! \dif \lambda \, Q(\varepsilon, \lambda)$$
The moments are:
\begin{enumerate}
\item $\Gamma{\sigma,n}(\varepsilon,\lambda)$ (particle flux)
\item $Q_{\sigma,n}(\varepsilon,\lambda)$ (energy flux)
\end{enumerate}
The possible fields are:
\begin{enumerate}
\item electrostatic component
\item electromagnetic component
\end{enumerate}
\item \textbf{k\_perp\_squared}\emph{ numpy array} - Numpy array with dimensions of (n\_n, n\_time).  It contains the flux-surface and radial average of $k^2_\perp$:
$$\frac{\langle\langle(k_\perp\bar{\rho}_{s,\mathrm{unit}})^2|\delta\phi_n|^2\rangle\rangle_r}{\langle\langle|\delta\phi_n|^2\rangle\rangle_r}$$
\end{itemize}
\item Methods:
\begin{itemize}
\item \textbf{\_\_init\_\_(}\emph{str}\textbf{ sim\_directory)} - This is the class constructor.  It is called when a new object of that class is created, and it reads in data from sim\_directory to create that object.  It sets the variable directory\_name to sim\_directory, and appends the pyrats folder to the python search path.  It also calls the following methods:
\begin{enumerate}
\item init\_data()
\item read\_data()
\item equil\_time()
\end{enumerate}
\item \textbf{equil\_time()} - This method counts the number of time steps present in the currently loaded arrays, and sets them all equal (to the length of the array with the fewest number of time steps).
\item \textbf{get\_input(}\emph{str}\textbf{ input\_name)} - This method returns the specified variable from input.gyro.gen.  Ex: get\_input("TIME\_STEP")
\item \textbf{init\_data()} - This method initializes object data.
\item \textbf{make\_diff()} - This method creates self.diff by computing it from self.gbflux.
\item \textbf{make\_diff\_i()} - This method creates self.diff\_i by computing it from self.gbflux\_i.
\item \textbf{make\_gbflux()} - This method creates self.gbflux by averaging over the radial component of self.gbflux\_i.
\item \textbf{plot(}\emph{list}\textbf{ x,}\emph{ list}\textbf{ y,}\emph{ tuple}\textbf{ dim=(1, 1))} - This method creates a matplotlib plot of the requested data.
\item \textbf{read\_data()} - This method reads in object data.  It executes:
\begin{enumerate}
\item read\_profile()
\item read\_geometry()
\item read\_t()
\item read\_gbflux\_i()
\item read\_gbflux\_n()
\end{enumerate}
\item \textbf{read\_file(}\emph{str}\textbf{ fname,}\emph{ int}\textbf{ dSize)} - This method reads the GYRO data file named out.gyro.fname.  It must be given the length in number of characters of the data elements in the file as a second argument.
\item \textbf{read\_flux\_velocity()} - This method reads in flux\_velocity data, and stores it in self.flux\_velocity.
\item \textbf{read\_freq()} - This method reads in frequency data and stores it in self.freq.
\item \textbf{read\_gbflux\_i()} - This method reads in gbflux\_i data and stores it in self.gbflux\_i.
\item \textbf{read\_gbflux\_n()} - This method reads in gbflux\_n data and stores it in self.gbflux\_n.
\item \textbf{read\_geometry()} - This method reads in geometry\_array data and stores it in self.geometry.
\item \textbf{read\_k\_perp\_squared()} - This method reads in k\_perp\_squared data and stores it in self.k\_perp\_squared.
\item \textbf{read\_moment\_e()} - This method reads in moment\_e data and stores it in self.moment\_e.
\item \textbf{read\_moment\_n()} - This method reads in moment\_n data and stores it in self.moment\_n.
\item \textbf{read\_moment\_u()} - This method reads in moment\_u data and stores it in self.moment\_u.
\item \textbf{read\_moment\_v()} - This method reads in moment\_v data and stores it in self.moment\_v.
\item \textbf{read\_moment\_zero()} - This method reads in moment\_zero data and stores it in self.moment\_zero.
\item \textbf{read\_profile()} - This method reads out.gyro.profile to get control data and stores it in self.profile.
\item \textbf{read\_t()} - This method reads t.out to get time data and stores it in self.t.
\item \textbf{set\_directory(}\emph{str}\textbf{ path)} - This method sets the variable self.directory\_name to path.
\end{itemize}
\end{itemize}
\end{document}
