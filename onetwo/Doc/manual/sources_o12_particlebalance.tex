\section{Transport Equations}\label{sec:transport}

As will be seen below the grad rho factors that enter into the equations used in
Onetwo (as a result of flux surface averaging the two dimensional transport
equations) are treated in a  particular way in the code. Thus one has to be
careful in defining the fluxes and  diffusion coefficients that are to be used.
A detailed discussion is given in appendix \ref{appendix1}.

\subsection{Particle Balance Equations}

The equation that governs the evolution of  the density of primary ion species,
i, is
\beq
 \pdiff{n_i}{t}+\frac{1}{H\rho}\ddiff{\rho} \big(H\rho\Gamma_i
 \big) =
 S_i+S_i^{2D} \label{eq:ni}.
\eeq
The second term in this and other equations appearing below represents the
azimuthally symmetric flux surface averaged divergence of the flux. Here
$\Gamma_i $ is the particle flux ($\frac{\#}{cm^2sec} $) of ion species i. In
simulation mode a pinch term can be added to  this flux. The source terms
appearing on the RHS of this and  subsequent equations below are explained in
section \ref{source}. 

\subsection{Electron Energy Equation}

The equation for describing the evolution of the electron thermal energy is 
\begin{multline} 
 \frac{3}{2}\bigg (T_e\sum_{i=1}^{nion}\big(n_i\pdiff{Z_i}{T_e}\big) + n_e
  \bigg)\pdiff{T_e}{t} 
 +\frac{3}{2}T_e \sum_{i=1}^{nion}Z_i\pdiff{n_i}{t} \\
 +\frac{1}{H\rho}\ddiff{\rho}\bigg(H\rho (
  q_e+\frac{5}{2}\Gamma_e T_e) \bigg)
 =Q_e  -\omega L_e + S^{2D}_{T_e} \label{eq:eenergy}.
\end{multline}
The source term $\omega L_e $ represents the (beam) energy that goes into
spinning up the electron fluid. This energy is consequently not available for
heating the electron distribution. The rotational kinetic energy of the
electrons is  assumed to be given to the ions (see below, this is a crude first
approximation).  

\subsection{Ion Energy Equation}

The equation for describing the evolution of the ion thermal  and rotational
kinetic energy is
\begin{multline} 
 \sum_{i=1}^{nion}\left\{\frac{3}{2}n_i\pdiff{T}{t} + 
  \pdiff{n_i}{t}\bigg(\frac{3}{2}T+
  \frac{1}{2}m_i\omega^2\langle R^2\rangle \bigg)\right\} 
 + \sum_{i=1}^{nion} m_in_i\omega \langle R^2\rangle\pdiff{\omega}{t} \\
 +\frac{1}{H\rho}\ddiff{\rho}\left\{H\rho\bigg(\sum_{i=1}^{nion}(
  q_i+\frac{5}{2}\Gamma_i T) +\Gamma_T^\omega
  +\Pi\omega\bigg)\right\}\\
 = Q +S_T^\omega+S^{2D}_{T}+S^{2D\omega}_{T}
 \label{eq:ionenergy}.
\end{multline}
All ion species are assumed to have a common temperature $ T $. The flux surface
average kinetic energy per unit volume in rotation is 
\beq
 \sum_{i=1}^{nion}\frac{1}{2}m_in_i\omega^2\langle R^2\rangle .
\eeq
The  kinetic energy is assumed to diffuse with the primary ions so we have a
flux
\beq
 \Gamma_T^\omega=\sum_{i=1}^{nion}\frac{1}{2}m_i
  \langle R^2\rangle\omega^2\Gamma_i
\label{eq:cvct}
\eeq
and a term associated with  viscous heating of the plasma $\Pi \omega$ described
further below.

\subsection{Faraday's Law} \label{fday}

The evolution of the poloidal B field is given by Faraday's Law. In \ot this
equation takes the form
\begin{multline}
 \frac {1}{ FG(H\rho)^2\alpha}\pdiff {(FGH\rho B_{p0})}{t} 
 -\frac{1}{ H\rho}\pdiff { }{\rho}
 \Bigg [H\rho \bigg (d_{4,1}\pdiff {n_i}{\rho} +d_{4,2}\pdiff {T_e}{\rho} 
  +d_{4,3}\pdiff {T}{\rho} \\  
  +d_{4,4}\pdiff {FGH\rho B_{p0}}{\rho} \bigg)\Bigg]
 -\frac{1}{H\rho}\pdiff {(dB_{p0})}{\rho}   
 =-\frac{1}{ H\rho}\pdiff { }{\rho} 
  \bigg (\eta_{\parallel} c H
   \langle {\vec {J_{aux}}} \cdotp \frac{\vec {B}}{ B_{t0}}\rangle \bigg)\\
 +\frac{1}{ H\rho} \pdiff { }{\rho} 
  \bigg [H\rho(D_f^e +D_f^b)\pdiff{ n_f}{\rho}  \bigg]
 +\frac{B_{p0}}{ H\rho } \pdiff { }{t} \bigg( lnFGH\rho \bigg)
-\frac{B_{p0}}{ H\rho } \pdiff {d}{\rho} \label{eq:fday1_1}.
\end{multline}                                 
Here the terms in the gradient of $n_i,\ T_e,\ T,$ and $D_f^e,\  D_f^b $ are
related to the bootstrap current.  $B_{p0}$\ is the  poloidal magnetic field
defined as
\mbox{$B_{p0} = \frac{1}{ R_0}\pdiff {\psi }{\rho} $.} 
d is the speed of a flux surface moving relative to the magnetic axis. For non
evolving equilibria we have $ d \equiv 0 $ and the profiles F,G,H are time
independent. The factor $\alpha$ appearing  in this equation is used to speed up
evolvement toward the steady state in certain instances ($\alpha$ is set to a
large number when the input switch \texttt{iffar}=1 is set in inone). The
bootstrap selection is done with the $d_{4,j}$ parameters . A number of
different bootstrap models are available in \ot. Each model consists of a
subroutine which provides the fourth row of the matrix d as indicated above, 
plus an explicit source term  for fast ions (since, at present, a fast ion
equation is not part of our coupled set of diffusion equations). The total
bootstrap current density takes the particular form
\beq
 \left \langle \frac{\vec{J_{boot}} \cdotp \vec{B}}{B_{T0}} 
 \right \rangle =
 -\frac{\rho}{c\eta_\parallel}\left (
 \sum_{i=1}^{3}d_{4,i}\pdiff{u^i}{\rho}
 + (D_f^e+D_f^b)\pdiff{n_f}{\rho} \right ) \label{eq:bootdef}.
\eeq
The terms in $D_f^e $ and $D_f^b$ represent an  ad-hoc approximation to account
for the fast ions and may be included or excluded as the user desires.

The total parallel current density  consists of ohmic, bootstrap, and auxiliary
contributions:
\beq
 \bigg \langle \frac{\vec{J_t} \cdotp \vec{B}}{B_{T0}}
  \bigg \rangle =
 \left \langle \frac{\vec{J_{ohm}} \cdotp \vec {B} } {\vec{B_{T0}}}
  \right \rangle +
 \left \langle \frac{\vec{J_{aux}} \cdotp \vec {B} } {\vec{B_{T0}}}
  \right \rangle +
 \left \langle \frac{\vec{J_{boot}} \cdotp \vec {B} } {\vec{B_{T0}}}
  \right \rangle    \label{eq:jtot}.
\eeq
Here $\vec{J_{aux}}$ is the current driven by neutral beam injection and rf
current drive, $\vec{J_{boot}}$ is the bootstrap current, and the total magnetic
field is $\vec{B}=\vec{B_T}+\vec{B_P}$ (i.e. the sum of toroidal and poloidal
components). $B_{T0}$ is a reference magnetic field in the absence of any 
plasma current at the particular location $R_0 =169.5\ cm$ (for DIIID, these
must be defined consistently with $f(\psi)$; see MHD section below).

The parallel component of Ampere's Law is given by
\beq
 \left \langle \frac{\vec{J_t}\cdotp \vec{B}}{B_{T0}} \right \rangle
 =\frac{\rho}{c\eta_\parallel}d_{4,4}\pdiff{\left (FGH\rho B_{p0}\right)}{\rho}
 \label{eq:jtpar}
\eeq
where the ``diffusion'' coefficient is
\beq
d_{4,4}=\frac{c^2\eta_\parallel}{4\pi F^2 H\rho^2}.
\eeq
The form of \Eqref{eq:fday1_1} is a consequence of the assumed relationship
between the parallel inductive electric field and the ohmic current:
\beq
 E_0\equiv H\left \langle\frac{\vec {E} \cdotp \vec{B}}{B_{T_0}}\right \rangle
 =\eta_{\parallel}H\left \langle\frac{\vec{J_{ohm}}\cdotp \vec{ B}}{B_{T_0}}
  \right \rangle   \label{eq:efield}.
\eeq
$\eta_\parallel$ is the parallel resistivity (corrected for trapped particles)
and $\vec{J_{ohm}}$ is the inductively driven  current. At any time, $t$, the
loop voltage calculated by the code is given in terms of this electric field as
\beq
V_{loop}=2\pi R_0E_0(\rho=a) \label{eq:vloop}
\eeq
where $a$ is the plasma radius.
Combining these results we find that the equation for the parallel
ohmic electric field is given by
\beq
 -\frac{cE_0}{H\rho}=-\left ( \sum_{i=1}^4d_{4,i}\pdiff{u^i}{\rho}
  +\left (D_f^e+D_f^b\right )\pdiff{n_f}{\rho}\right )
 +\frac{\eta_\parallel c}{\rho}
  \left \langle\frac{\vec{J_{aux}}\cdotp\vec{B}}{B_{T0}} \right \rangle
 \label{eq:e0}
\eeq
Note that when i=4 in the above equation we are in fact referring to the total
current. Hence the RHS of \Eqref{eq:e0} is just the ohmic current expressed
using \Eqrefs{eq:bootdef}{eq:jtpar}.

The Grad-Shafranov equation is flux surface averaged to yield an expression for
the toroidal current density $ J_\phi $:
\begin{eqnarray}
 \frac{1}{H\rho}\ddiff{\rho} \left ( \frac{u^4}{F}\right) & = & \frac{4\pi}{c}
 \left \langle \frac{J_\phi R_0}{R} \right \rangle \\
 &  = & -\frac{1}{B_{P0}}\left ( 4\pi\pdiff{P}{\rho}
  +\frac{1}{2} \left \langle \frac{1}{R^2} \right \rangle
  \pdiff{f^2}{\rho} \right )       
\end{eqnarray}

The calculations are done  as follows. At any time, $t$, the dependent variable
$u^4 \equiv FGH \rho B_{P0}$ is known either from the initial condition or from
having solved (possibly some subset of) the set of diffusion equations. The
total parallel current density, \Eqref{eq:jtpar}, is thus determined. The
bootstrap current is evaluated using \Eqref{eq:bootdef} and independent
models are used to determine the auxiliary beam and rf driven currents. We are
thus able to solve \Eqref{eq:jtot} for the ohmic term. The original version
of \ot  substitutes $\left \langle \frac{J_\phi R_0} {R} \right \rangle$ for 
$\left \langle \frac{\vec{J_t}\cdotp \vec{B}}{B_{T0}} \right \rangle$ in this
calculation. The  relationship between these quantities is
\beq
 \left \langle \frac{J_\phi R_0}{R} \right \rangle = 
 F\left \langle \frac{\vec{J_t} \cdotp \vec{B} }{B_{T0}} \right \rangle
 -\frac{c}{4\pi}\frac{GB_{p0}}{F}\pdiff {F}{\rho}
\eeq
and hence the replacement is equivalent to assuming that
$F\equiv 1$. The new version of \ot allows the user to 
eliminate this approximation!

%------------------------------------------------------------------
\subsubsection{Faraday's Law In TDEM Mode}\label{fday2}

The time dependent eqdsk mode of operation (see section \ref{tdem} for a
description), introduces a new way of treating Faraday's Law in \ot.  By writing
\Eqref{eq:fday1_1} in the form 
\begin{multline}
 c\ddiff{\rho} \bigg( H\eta_{\parallel} \left \langle
  \frac{\vec{J_{Ohmic}} \cdotp \vec {B}}{B_{T0}} \right \rangle \bigg )
 = \frac{1}{FGH\rho } \pdiff{(FGH\rho B_{P0})}{t}-\ddiff{\rho}{dB_{P0}}\\  
 - B_{P0}\ddiff{t} (\ln{FGH\rho})+ B_{P0}\pdiff{d}{\rho} 
 \label{eq:fday2_1}
\end{multline}
we obtain a simple differential equation for the product of the parallel
resistivity and the ohmic current. Internal definititons used in \ot which are
relevant to \Eqref{eq:fday2_1} are
\begin{gather}
 -B_{P0}\ddiff{t} (\ln{FGH\rho}) = -H\rho\emph{fday2d1} \\
 B_{P0}\pdiff{d}{\rho} = -H\rho\emph{fday2d2} \\
 -\ddiff{\rho}{dB_{P0}} = -H\rho \emph{fday2d3} .
\end{gather}
Note that \Eqref{eq:fday2_1} can be reduced to the simpler form 
\beq
 \ddiff{t}\Psi = H R_0 \eta_{\parallel}  \left \langle 
  \frac{\vec{J_{Ohmic}} \cdotp \vec {B}}{B_{T0}} \right \rangle
 \label{eq:fday2_4}
\eeq
using the relationship
\beq
 \pdiff{\Psi}{t}\pmb{\bigg \vert_\rho} = 
 \pdiff{\Psi}{t} \pmb{\bigg \vert_\zeta} 
 -\bigg (\frac{\zeta}{\rho_a} \bigg )
 \bigg (  \pdiff{\Psi}{\zeta}\pmb{\bigg \vert_t}\bigg ) 
 \bigg(\pdiff{\rho_a}{t}\pmb{\bigg \vert_\zeta} \bigg ) 
\eeq
and the definitions
\begin{gather}
 B_{P0}\equiv\frac{1}{R_0}\pdiff{\Psi}{\rho}\pmb{\bigg \vert_t} \\
 d \equiv -\pdiff{\zeta}{t}\pmb{\bigg \vert_{\rho}}\pdiff{\rho}{\zeta}
  \pmb{\bigg \vert_t} = \pdiff{\rho}{t}\pmb {\bigg \vert_\zeta}\\
 \zeta \equiv \frac{\rho}{\rho_{a(t)}}.
\end{gather}

In TDEM mode the space and time dependent behavior of the RHS of
\Eqref{eq:fday2_1} is approximated a priori using time dependent solutions of
the Grad-Shafranov equation. The constant of integration in \Eqref{eq:fday2_1}
is related to the plasma surface voltage, 
$V_s \equiv 2\pi\ddiff{t}\Psi_{lim} $,
\beq
 \frac{V_s}{2\pi R_0} = \eta_{parallel} H 
 \left \langle \frac{\vec{J_{Ohmic}} \cdotp \vec {B}}{B_{T0}} \right \rangle
 \pmb{\bigg \vert_{\rho_a}}.
\eeq

It is thus possible to  determine the resistivity and/or the noninductive
current. In the latter case we can determine either the bootstrap or the
auxiliary current provided we are willing to approximate the other current
profiles with an appropriate model. 

%------------------------------------------------------------------
\subsection{Toroidal Momentum Equation}

The equation for toroidal momentum and rotation used in Onetwo assumes that all
of the momentum and energy is carried by the ions. All ions have the same
temperature and rotation speed, the associated momentum of each ion fluid
depends on the mass of the ion however. The actual equation solved by Onetwo is
\begin{multline}
 \sum_{i=1}^{nprim}m_i n_i\langle R^2\rangle \pdiff{\omega}{t}
 +\omega\sum_{i=1}^{nprim}m_i\langle R^2\rangle\pdiff{n_i}{t} 
 +\frac{1}{ H \rho}\ddiff{\rho}\left(H\rho
 \Gamma_\omega\right) 
 = S_\omega  +S_\omega ^{2D}  \label{eq:omega}.
\end{multline}
The flux surface average toroidal angular momentum density for ion species i  is
given by
\beq
 m_in_i\omega\langle R^2\rangle.
\eeq
The total momentum flux ($\frac{g}{sec^2}$) is made up of two parts as usual
\beq
 \Gamma_\omega=\Gamma_\omega^{cond}+\Gamma_\omega^{conv}
\label{eq:omgam}.
\eeq
The ``convective'' flux is
\beq
 \Gamma_\omega^{conv}=\sum_{i=1}^{nprim}m_i\langle R^2\rangle\omega\Gamma_i.
\eeq
The ``conductive'' momentum flux is 
\beq
 \Gamma_\omega^{cond} \equiv \Pi =\sum_{k=1} \Pi_k
\label{eq:gwcond}.
\eeq
With each $\Pi_k $ defined as
\beq
 \Pi_k=-d_k(\rho)\ddiff{\rho} u_k \label{eq:dcoef}.
\eeq
At the present time $d_k (\rho)\frac{g cm}{sec}$ is assumed to be diagonal with
a contribution only form $k=\omega $. Several simple models for $d_\omega $
exist in the code.

All of the above equations can be run in analysis or simulation mode. Analysis
mode means that the appropriate dependent variable is known a priori as a
function of space (and possibly time).  Consequently the associated diffusion
equation can be inverted to yield information on the transport coefficients. The
success of  this inverse method depends on how well the sources are known. For 
example if the momentum equation is run in analysis mode, \Eqref{eq:omega}
is first solved for $\Gamma_\omega $ (assuming all other terms in the equation
are known a priori):
\begin{multline}
 \Gamma_\omega =\frac{1}{H\rho}\int_0^\rho H\rho\bigg(S_\omega +S_\omega^{2D}
  -\omega\sum_{i=1}^{nprim}m_i\langle R^2\rangle\pdiff{n_i}{t} \\
 - \pdiff{\omega}{t}\sum_{i=1}^{nprim}m_i n_i\langle R^2\rangle \bigg)d\rho
\end{multline}
and then $\Gamma_\omega^{cond} $ is obtained from  \Eqref{eq:omgam} (the
convective part is known at this point from particle balance,
\Eqref{eq:ni}). The diffusion  coefficient can then be obtained using
\Eqref{eq:dcoef}.

The equations that are run in simulation mode must have  transport coefficients
and appropriate initial and boundary conditions specified a priori. The profiles
are then evolved from the initial condition profile as usual. 

\subsection{Neutral Density Equation}

The neutral density in Onetwo is determined by instantaneous global rate balance
conditions. Given the electron and fast and thermal ion densities we can
calculate the volume source of neutrals from the recombination and beam neutral
charge exchange processes. Using this volumetric source we can determine the
resulting neutral density by solving \Eqref{eq:neut} below.
\beq 
 n_{neut}=?\label{eq:neut}
\eeq 
This neutral density then allows us to determine the total volume integrated
rate at which neutrals are consumed due to electron impact ionization.  At any
given time the total volume integrated neutral loss rate must equal the  source
rate.  This  determines the neutral density due to volumetric sources and sinks
(subroutine \texttt{neuden} carries out these calculations).

To complete the problem we must also consider wall sources of neutrals. In
analysis mode additional information, in the form of the particle confinement
time, $taupin $, is required. Given $taupin $ we can determine the total rate 
of ions leaving the plasma through outward diffusion (by the definition of
particle confinement time):
\beq
 \Gamma_i S_A = \frac{\int n_i dv}{taupin}
\eeq
The inward neutral flux due to this ion species is the above value plus any
additional specified gas puff. This inward flux of neutrals yields, by way of
\Eqref{eq:neut}?, an associated density in the plasma. Again, this neutral
density has a volume integrated loss rate due to electron impact ionization, 
which must be balanced with the global source  rate,  and  hence the neutral
density due to wall sources is determined. This part of the neutral density is
thus inversely proportional to $taupin$. The total neutral density is just the
sum of the wall and volume  source problems (\Eqref{eq:neut}? is linear). 

\ot  may determine  fast ion charge exchange losses that are too large when
$taupin $ is set to a  value that is too small (e.g. much less than the usual
200-400 ms) which results in a large neutral density. Because $taupin $ controls
the neutral density and the neutral density in turn controls the fast ion
density, a small value of $taupin $ is often used to enforce charge balance.
Particularly in high performance, low density cases, the fast ion density  may
locally (near the magnetic axis) become greater than or equal to the measured
electron density. Decreasing $taupin$ would decrease this fast ion density so
that charge balance could be ensured. However a large fast ion charge exchange
loss is thus incurred.

Assuming that $taupin $ is set to approximately the correct value and assuming
that the measured input profiles of $n_e$ and $z_{eff}$ are reasonably
accurate,  one possible explanation of this local lack of charge balance is an
incorrect (i.e. too peaked ) fast ion density profile. Improper prompt orbit
averaging and possibly (lack of) fast ion diffusion are thus likely to be the
cause. An option in \ot allows spreading of the fast ion density to achieve
charge balance with the specified $z_{eff}$ and electron density profiles (see
the input switch \texttt{hdepsmth}). This is only a temporary  stop-gap measure
until more appropriate action can be taken. 

NOTE: As discussed above $taupin $ controls the neutral density from (part of )
the wall source. It may happen that the neutral density due to the volume source
of neutrals is much larger (high density cases) so that $taupin $  becomes
irrelevant.

The neutral density is determined by the Boltzmann equation specialized to 
circular cylindrical geometry, as developed by Burrell\cite{Burrell:78}.

\begin{center}
 Put neutral density equation and defns here!
\end{center}


