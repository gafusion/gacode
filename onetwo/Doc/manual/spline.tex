\section{ Cubic Spline Review }

With n knots, $[a_1,a_n]$, there are $n-1$ cubics to be determined. Each cubic
has 4 unknowns so we need to find $4(n-1)$ unknowns altogether. A spline is
defined by its function values at the knots, the continuity of the first and
second derivatives at the interior knots and two boundary conditions.  This
arises as follows:

The continuity of the value of the function for each of the $n-1$ cubics at the
left and right ends of the interval in which the cubic is defined results in
$2(n-1)$ equations; the continuity of the first derivative at the $n-2$ interior
knots yields $n-2$ equations and we get  another $n-2$ equations for the
continuity of the second derivatve. Hence we have $2(n-1) +2(n-2) = 4n-6$
equations . An additional equation is given at the first knot and at the last
knot to get a total of $4(n-1)$ as required.

In practice the above procedure to find the equations for the spline
coefficients is greatly simplified by taking advantage of the following general
representtion of a cubic in an arbitrary interval. This representation already
incorporates the function values at the knots and the continuity of the second
derivative at the interior knots. For    $ x \in [a_{i-1},a_i] $ a general cubic
can be written in the form 
\begin{multline}
 c(x) = M_{i-1} \left [ \frac{(a_i-x)^3}{6h_i} -\frac{h_i}{6}(a_i-x) \right ]\\
 + M_i \left [\frac{(x-a_{i-1})^3}{6 h_i} -\frac{h_i}{6}(x-a_{i-1}) \right ]  
 +c_{i-1}\frac{a_i - x}{h_i}
 +c_i\frac{x-a_{i-1}}{h_i} \label{eqa}.
\end{multline}
The four constants that define the cubic have been written in terms of the
function values $(c_{i-1},c_i)$ and the second derivatives, $(M_{i-1}, M_i)$  at
the end points of the interval. The knot spacing is given by $h_i = a_i -
a_{i-1} $ and the index $i$ takes on values from 2 to $n$ (so there are $n-1$
intervals). In order to turn this representation of a cubic into a cubic spline
representatstion of a function we have to consider $n-1$  contiguous intervals,
with internal boundaries $a_2,\ldots, a_{n-1} $ and edge (e.g. boundary) values
at either  end, $a_1$ and $a_n$. The above prescription already satisfies the
condition that the second  derivative is continuous across the knots since, at
each interior knot, the right end of interval $i$ is the same as the left end of
inerval $i+1$. Hence we need only force the first derivative to be continuus at
the interior knots.  Applying this conditon at $x = a_2, \ldots,a_{n-1} $ yields
$n-2$  equations  for the $n$ unknowns, $M_1,\ldots,M_n$,
\begin{eqnarray}
 \frac{h_i}{6}M_{i-1} + \frac{h_i+h_{i+1}}{3} M_i +
 \frac{h_{i+1}}{6} M_{i+1} = \frac{c_{i-1}}{h_i}
 -c_i(\frac{1}{h_i} +\frac{1}{h_{i+1}})
 +\frac{c_{i+1}}{h_{i+1}}
 \label{eqbb}.
\end{eqnarray}
Note that \Eqref{eqbb} is valid only at the interior knots, $i=2\ldots n-1 $.
Two additional equations are thus required to solve for the $ M_i $ uniquely.
The general form of these two additional equations is given by \Eqref{eqa}
specialized to the first and last intervals ($i =1$ and $i = n$, respectively).
To cleanly blend with the IMSL splines routines these two equations are taken in
the form
\begin{eqnarray}
 2M_1 +bpar(1)M_2&  = &  bpar(2)              \label{eqc}\\
 bpar(3)M_{n-1}+2M_n & =& bpar(4)             \label{eqd}.
\end{eqnarray}
To make sense out of these two equations we look at the  cases that typically
arise: a) the first derivative is given  or b) the second derivative is
constant. Usually one of (a) or (b) is applied at $a_1 $ and, independently, one
of (a) or (b)  is also applied at $a_n$. (The energy code allows for an
additional alternative specification as a point of inflection. - This is not
discussed here). Differentiating \Eqref{eqa} and applying the results at $a_1 $
and $a_n$ we have:
\begin{eqnarray}
 2M_1 +M_2 & = & \frac{6}{h_2}\left(\frac{c_2-c_1}{h_2} - d \right) 
 \label{eqe}\\ 
 M_{n-1} + 2M_n & = & \frac{6}{h_n}\left(e - \frac{c_n
 -c_{n-1}}{h_n} \right) \label{eqf}.
\end{eqnarray}
Here $d$ is the desired first derivative at $x = a_1 $ and $e$ is the desired
first derivative at $x= a_n $. Comparing Eqs.~\eqref{eqe} and \eqref{eqf} with
Eqs.~\ref{eqc} and \ref{eqd},  we can read off the definition of bpar required
to set the first derivatives to the values $d$, $e$ . This confirms  the IMSL
definition of $bpar$. 

The more mundane case of constant second derivative at the
left and right ends is simply given by the fact that $ M_1 = M_2 $ and
$  M_{n-1} = M_n $. These equations must be written in the
IMSL conformable way as:
\begin{eqnarray}
 2M_1 -  2 M_2 & = &  0  \label{eqg} \\
 - 2 M_{n-1} + 2M_n & = & 0   \label{eqh}
\end{eqnarray}
Hence we see that $bpar(1) = -2$, $bpar(2) =0.0$ and
$bpar(3) = -2$, $bpar(4) =0$ will achieve this in
Eqs.~\eqref{eqc} and \eqref{eqd}.

Obviously there are other boundary conditions that could be fit
into this scheme. To summarize, the cubic spline is defined by
a set of $n$ equations for the $M_i $. The first equation is
taken as \Eqref{eqc}, equations 2 to $n-1$ are of the form
\Eqref{eqbb}, and the last ($n$'th) equation is \Eqref{eqd}.
$bpar$ suitably specialized yields the desired end point
conditions.

