\section{Neutron Rate Calculations}

\subsection{Thermonuclear Rate}

\begin{figure}[hbt] %note: figure environment not available in slides
 \centering
 \mbox{\epsfig{file=ddnhe3.fig1,height=10cm,width=12cm}}
 \caption{Neutron rate coefficients, $\frac{cm^3}{sec}$, for thermal 
 $D(D,n)He^3 $. Solid line is the new Bosch-Hale rate, dashed line is the
 ('94,'83) NRL rate.}
 \label{Figps2}
\end{figure}

The NRL formulary parameterization of the reaction rate plotted
in Fig. \ref{Figps2} is
\begin{equation}
 \langle\sigma v\rangle =\frac{2.33\cdot 10^{-14}T^{-(\frac{2}{3})}exp[-18.76
 T^{-(\frac{1}{3})}]}{2}.
\end{equation}
\emph{THIS PARAMETERIZATION IS NOT ACCURATE AND SHOULD NOT BE USED.}  (It does
not reproduce the rate tables in NRL with sufficient  accuracy).
%\vspace{10 mm}

The Bosch and Hale parameterization is
\begin{eqnarray}
 \langle\sigma v\rangle &  = &
  C1\cdot\Theta\sqrt{\frac{\xi}{m_rc^2T^3}}exp(-3\xi) \\
 \xi & = & {\biggl(\frac{B_G^2}{4\cdot\Theta}\biggr)}^{\frac{1}{3}}\\
 \Theta & = & \frac{T}{1.-\frac{T(C2+T(C6))}{1+T(C3+T(C5+TC7))}}
\end{eqnarray}
This expression precisely  matches the rate tables in NRL \emph{and} in Bosch
and Hale.

\subsection{Beam-Thermal Rate}

The fast ion distribution function is assumed to be the uniform magnetic field,
azimuthally symmetric solution of the Fokker-Planck equation, which
asymptotically approaches 
\begin{eqnarray}  %use eeqnarray for multiple line formulas
 f_b(v,\zeta)= \frac{\dot S \tau_s}{v^3+v_c^3} P_{cx}(v)
 \sum_{l=0}^{\infty}\frac{2l+1}{2}P_l(\zeta)P_l(\zeta_b)
 \left[\frac{v^3}{v_b^3} \left(\frac{v_b^3+v_c^3}{v^3+v_c^3}
 \right)\right]^{\frac{1}{6}l(l+1)Z_2} \label{eq:1},
\end{eqnarray}
where the probability against charge exchange is given by 
\beq
 P_{cx}(v)=  exp\left(-\tau_s \int_{v}^{v_b}\frac{v^2dv}
 {(v^3+v_c^3)\tau_{cx}}\right) \label{eq:2}.
\eeq
For the neutron rate calculations in \ot , either the above  integral for
$P_{cx}$ is evaluated numerically with the mean time against charge exchange
given by
\beq
 \tau_{cx}=\frac{1}{n_n\sigma_{cx}(v)v} \label{eq:3}
\eeq
or it is assumed that 
$\tau_{cx}$ is constant, see \Eqref{eq:taucx}, in which case we obtain
\beq
 P_{cx}(v)=\left(\frac{v_b^3+v_c^3}{v^3+v_c^3}\right)
 ^{-\frac{\tau_s}{3\tau_{cx}}} \label{eq:4}
\eeq
For both cases the charge exchange cross section is the original Freeman and
Jones \cite{Freeman:1974} expression for charge exchange with hydrogen (not
deuterium).

The fast ion distribution function, integrated over polar and azimuthal angles
is 
\beq
 f_b(v)v^2dv=\frac{\dot S \tau_s P_{cx}(v) v^2dv}{ v^3+v_c^3} \label{eq:5}.
\eeq
The density of fast ions is by definition
\begin{eqnarray}
 n_f=\dot S \tau_s N \label{eq:6},\\
 N=\int \frac{P_{cx}(v)v^2dv}{v^3+v_c^3}\label{eq:7}.
\end{eqnarray}
In \ot an effective time dependent source rate is defined by calculating 
$P_{cx}$ and $N$ analytically and then setting
\beq
 \dot S \tau_s = \frac{n_f}{N}\label{eq:8}
\eeq
at subsequent times. The fast ion density, $n_f$, is assumed to to build up or
decay away with a time constant given by $\tau_f$. Initially $\dot S \tau_s $ is
obtained from \emph{NFREYA}. The speed dependent part of the fast ion
distribution function is then taken as independent of $P_{cx}$. Using energy as
the independent variable the transformed fast ion distribution becomes
\beq
 f_b(E)dE=\frac{ \frac{n_f}{N} dE}{2E(1+{\left(
 \frac{E_c}{E}\right)}^{\frac{3}{2}})} \label{eq:9}
\eeq

The neutron rate density $\left[\frac{1}{cm^3sec}\right]$ is given by
\beq
 R=\int \vec{dv_t}\vec{dv_b}f_t(v_t)f_b(v_b)v_{rel}\sigma({v_{rel}}) .
 \label{eq:10}
\eeq

One approximate formulation available in \ot assumes that the thermal ion speed,
$v_t=0$. The neutron rate density thus reduces to
\beq
 R=\frac{n_d \frac{n_f}{N}}{\sqrt{2m_b}}
 \int\frac{dE\sigma_{DD}(E)}{E^{\frac{1}{2}}
 \left[1+{\left(\frac{E_c}{E}\right)}^{\frac{3}{2}}\right]} \label{eq:11}
\eeq

Using the approximate (\emph{NRL Formulary}) cross section
\beq
 \sigma_{DD} =\frac{ c \epsilon^{\frac{-b}{\sqrt{E}}}}{E} \label{eq:12b}
\eeq
an analytic expression for $R$ is obtained by assuming that
$\frac{E_c}{E}=\frac{E_c}{E_b}$ in \Eqref{eq:11}. 

The neutron rate density becomes \Eqref{eq:11a}, which was originally obtained
from \cite{Scott}
\beq
 R=\frac{ \frac{n_f}{N} c k n_D e^{\frac{-b}{\sqrt{E_b}}}}
 {\left[1+\left({\frac{E_c}{E_b}}\right)^{\frac{3}{2}}\right]b} \label{eq:11a}.
\eeq
This is the (old) form used in \ot. To account for bulk  rotation the beam
energy is modified to reflect its value in the rotating frame. In the rotating
plasma frame the beam energy, $E_b^R$, is
\beq
 E_b^R=E_b+\frac{m_b}{m_{th}}E_{bulk}-|v_fv_{bulk}|m_f\cos(\theta)\label{eq:12},
\eeq
where $\cos(\theta) $ is the angle between the velocity vector of the
injected fast ion (which has an energy of $E_b$ in the lab frame)
and the (zero temperature) thermal ion (which has only bulk motion 
with velocity $\vec{v}_{bulk}$ and energy $E_{bulk}=
\frac{1}{2}m_{th}v_{bulk}^2$.) The beam-thermal neutron rate
determined from \Eqref{eq:11a}, with $E_b$ replaced by $E_b^R $ of \Eqref{eq:12} is used in \ot when the input  \texttt{iddfusb=0} is given.

A more accurate option available in \ot uses the Bosch and Hale\cite{Bosch}
cross section, accounts for the Maxwellian nature of the thermal ion
distribution, and does the integrals in \Eqref{eq:12c}  for the neutron rate
density numerically: 
\begin{eqnarray}
 R=2\pi\alpha\dot S \tau_s \int dv_{th}v_{th}^2e^{-\beta v_{th}^2}
 \int dv_f\frac{v_f^2P_{cx}(v)F(v_f,v_{th})}{v_f^3+v_c^3} \label{eq:12c}\\
 F(v_f,v_{th})=\int d\zeta_f\sigma(E_{com})(v_f^2+v_{th}^2
 -2v_fv_{th}\zeta_f)^{\frac{1}{2}} \nonumber
\end{eqnarray}
\Eqref{eq:12c} is selected by setting \texttt{iddfusb=1}.  The option of using
the effective source $\frac{n_f}{N} $ together with  neglecting $P_{cx}(v) $ may
be used in this expression as well to speed up the computations (set
\texttt{icalc\_cxrate=0}). Otherwise, with \texttt{icalc\_cxrate=1} the
computations are somewhat (but not significantly)  slower due to the numerical
evaluation of $P_{cx}(v) $, \Eqref{eq:2}. The time dependence of the fast ion
distribution function is accounted for by adjusting the lower and upper limits
of integration over the fast ion speed to reflect the fact that after beam turn
on no fast ions exist below speed $v_f^{lim}$ given by
\beq
 v_f^{lim}=[(v_b^3+v_c^3)exp(-3\frac{t}{\tau_s})-v_c^3]^{\frac{1}{3}}
 \label{eq:13}
\eeq
for times $t<\frac{1}{3}\tau_s \ln\left(\frac{v_b^3+v_c^3}{v_c^3}\right)$.
Similarly, after beam turnoff, no fast ions exist above speed $v_f^{lim}$ where
t is  measured from beam turnoff. Only a single beam turn on and off is
accounted for in the code. Time dependent beam power models are currently not
included. The calculations are done in the plasma frame with the initial beam
energy given by \Eqref{eq:12}. To eliminate rotation from the neutron rate
calculations of \Eqref{eq:12c} (by using $E_b$ instead of $E_b^R $) set
\texttt{iddfusb\_bulk=0}.


\subsection{Beam-Beam Rate}
calculations are in place in the code but documentation is not yet done.
