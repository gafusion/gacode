

  \documentclass[12pt]{article} 
   %% LaTeX2e file `abbrev.tex'
%% generated by the `filecontents' environment
%% from source `sources_o12' on 2010/05/19.
%%

   \newcommand{\vp}{ V^{\prime}}
   \newcommand{\ot}{Onetwo\xspace}
   \newcommand{\ct}{Curray\xspace}
   \newcommand{\pdiff}[2]{\frac{\partial{ #1}}{\partial{ #2}}}
   %useage \pdiff{#1}{#2}
   \newcommand{\ddiff}[1]{\frac{\partial}{\partial #1}}
   \newcommand{\strt}{\begin{eqnarray}}
   \newcommand{\trts}{\end{eqnarray}}
   \newcommand{\beq}{\begin{eqnarray}}
   \newcommand{\eeq}{\end{eqnarray}}
   \newcommand{\pdiffn}[3]{\frac{\partial^{#3}{#1}}{\partial{ #2}^{#3}}}
   \newcommand{\pdiffz}[3]{\frac{\partial{#1}}{\partial{ #2}}\bigg \vert_{#3}}
    \newcommand{\myint}{\int\limits_{\rho_{j-\frac{1}{2}}}^{\rho_{j+\frac{1}{2}}
}d\rho H \rho}
    \newcommand{\myind}{\int\limits_{\rho_{j-\frac{1}{2}}}^{\rho_{j+\frac{1}{2}}
}d\rho}
    \newcommand{\onehalf}{\frac{1}{2}}
    %\newcommand{\eqref}[1]{(\ref{#1})}
    \newcommand{\Eqref}[1]{Eq.~\eqref{#1}}
    \newcommand{\Eqrefs}[2]{Eqs.~\eqref{#1}-\eqref{#2}}

 \textheight8.5in
\newenvironment{narrow}[2]{%
   \begin{list}{}{%
   \setlength{\topsep}{0pt}%
   \setlength{\leftmargin}{#1}%
   \setlength{\rightmargin}{#2}%
   \setlength{\listparindent}{\parindent}%
   \setlength{\itemindent}{\parindent}%
   \setlength{\parsep}{\parskip}}%
\item[]}{\end{list}}

 % can keep my definitions is file abbrev.tex
   \usepackage{epsfig}
    \begin{document} 
      %\title{  (title gets put on separate page)
\begin{center} \bfseries
  Onetwo  Regression Testing Task Description \\
  H. St.John, 04/21/05 \\
\end{center}
    %  }
        %\section*{Summary}
        %\Large \boldmath 
        
\normalsize \mdseries

 
As per our conversation on April 21, here is the agreed upon project,
to ensure that Onetwo is not compromised as we add new features.
The project consists of two phases. In the  first phase several
scripts will be developed that automate the process described below.
In the second phase the scripts will be applied on an as per needed 
basis to verify the integrity of Onetwo. We will carry out the 
tests only on Linux based platforms. The choice of which machine to
use depends on the availability of compilers and software libraries
on each platform. We test Onetwo using both pgf90 and lf95
compilers. All Linux platforms except Delphi and Zeus currently
support Onetwo. It is my understanding that Delphi and Zeus are to be 
upgraded to the latest version of Linux in the near future so that
these two platforms should also be available eventually. To do
regression tests across (Linux based) platforms we would like to
partition the set of test cases among the available platforms.
The partitioning should be done in such a way that each platform runs
different sets of test cases each time the regression tests are
performed. The exception is that if a test fails then it should be
rerun on the same platform until it passes. (This implies that some
information is written to log files that can be used as input
to the regression script)

The basic approach consists of running Onetwo for all or possibly
a subset of directories stored under \\
\begin{center} /p/linux/onetwo/regression  \end{center}
There will be a number of versions of Onetwo that should be tested.
These version are defined by a)compiler type (pgf90,lf95) and b) grid
size used. The regression script should run all versions of Onetwo
in the directories
\begin{center}  /p/linux/onetwo/test\_version/pgf90 \\
  /p/linux/onetwo/test\_version/lf95  \end{center}
Typically we will concentrate on pgf90 code so the test versions in
the ../lf95 directory will be limited.
The regression script will know what subset of directories to run
by reading a file,''tests\_to\_run.txt'', that specifies which directories to run. It is
expected that after the initial creation this file will be updated
automatically by the regression script. It should be plain text so
that the user can edit it. A simple directive such  as ``all'' on the
first line of this file should be used to indicate that all
directories found under /p/linux/onetwo/regression are to be run.

The general procedure is as follows:
\begin{itemize}
  \item For each code in /p/linux/onetwo/test\_version/pgf90 
    \begin{itemize}
      \item decode the Onetwo executable name to get the
            grid size. For example a name in
            /p/linux/onetwo/test\_version/pgf90 
          might be onetwov3.6\_129\_51 . This implies that all
          directories below the directory 129\_51/  in the regression
          directory hierarchy are to be entered. Conversely it 
         means that directories under 65\_51/ are to be ignored.
      \item For each selected directory in ``tests\_to\_run'':
  \item  Copy the files to a scratch area
  \item  run the code
  \item run summarize (this creates the file summary)
    \item compare scratch/summary with the base summary file found the
          original test directory. Comparison will be based on
          comparing numbers in the summary files. 
    \item for cases that passes 
   \begin{itemize}
     \item write Onetwo version used, platform name , case description
     (which is just the full name of the test directory) 
     to a file log\_pass.txt
   \end{itemize}
     \item    for cases that fail, 
      \begin{itemize}
     \item write the above information to a file
        log\_fail.txt . Also write failed information to a file
        ``tests\_to\_run\_again.txt'' . This file will be copied
        to file ``tests\_to\_run.txt'' and used as input in subsequent
        regression tests. Save the failed case summary file in 
        the current regression directory as summary\_failed.
    \end{itemize}
    \end{itemize}
    \item repeat the above sequence of steps  using the test codes in
      /p/linux /onetwo/test\_version/lf95 . 

\end{itemize}

 \noindent Notes:  \\

  None of the file or path names should be ``hard coded''. Instead the
 regression script should, each time it is run, read a
 file,''regress\_directives.txt'' that contains the filenames and
 paths. For example contents of regress\_directives.txt might be : \\
  \begin{verbatim}
      /p/linux/onetwo/regression         
      /p/linux/onetwo/test_version/pgf90                           
      /p/linux/onetwo/test_version/lf95  
      tests_to_run.txt 
      log_pass.txt 
      log_fail.txt 
      tests_to_run_again.txt 
      summary_failed  
\end{verbatim} 

  The above entries are in order: \\
  1)root of regression tree       \\
  2)location of  pgf90 compiled  Onetwo executable  to test  \\
  3)location of  lf95  compiled  Onetwo executable  to test  \\
  4)name of file that indicates which directories are to be 
    tested. This file initially will consist of the single line
    ``all'' meaning that all directories are to be tested.
    After such an initialization this file will be overwritten by
    file test\_to\_run\_again.txt or similar. \\
  5)name of log file for runs that passed  \\
  6)name of log file for runs that failed  \\
  7)name of file that contains information on cases that failed.
    Note that this file will be used as input instead of
    tests\_to\_run.txt in some circumstances.
  8) for cases that failed the summary file should be renamed
    summary\_fail and saved in the current regression directory.


   If a directory in the regression tree,/p/linux /onetwo/regression ,
   is empty or if there are no test codes in
   /p/linux /onetwo/test\_version/pgf90   or       
   /p/linux /onetwo/test\_version/lf95   then it should be
   reported as an error in file log\_fail.txt.

  All scripts developed in association with this project should be 
  CVS based. Hence create a directory called ``regression'' in the Onetwo
  CVS repository /c/cvsroot/onetwo. Then keep all developments current
  in this repository.

 To compare summary files a script can be created that reads in each
 line of each file and extracts the numbers to be compared.
 Alternatively a Fortran code can be built by exchanging the 
 write statements with read statements. See the script
 /usr/local/bin/summarize and the Fortran subroutine 
  /p/linux/onetwo/source/onetwo/cray309.f 

 The Python language is well suited for this task. See \\
 http://www.ifi.uio.no/in228/lecsplit/ \\
 for a great tutorial on how to do things in Python,
 including walking directory trees and regression testing, etc.


  \end{document}