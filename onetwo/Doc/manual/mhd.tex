\renewcommand{\textfraction}{0.15}
\renewcommand{\topfraction}{0.85}
\renewcommand{\bottomfraction}{0.65}
\renewcommand{\floatpagefraction}{0.60}

\section{MHD}

The flux surface average geometry dependent factors appearing in the above
equations are defined as 
$$  
 F=R_0 B_{t0}/f(\psi(\rho)),\qquad  
 G=\langle(\nabla\rho)^2R_0^2/R^2\rangle,\qquad   
 H = F/\langle R_0^2/R^2\rangle,
$$
where the angle brackets denote the usual flux surface averaging, $ R_0$ is the
major radius at which the vacuum magnetic toroidal field is given as $B_{t0}$,
$R$ is the major radius coordinate to a a point on a flux surface (in the output
of \ot these quantities are labeled fcap, gcap, and hcap, respectively). The
independent radial coordinate used in the diffusion equations is defined as
\beq
 \rho=\sqrt{\frac{\Phi}{\pi B_{t0}}}.
\eeq
Here $\Phi $ is the toroidal flux inside a given flux surface.

There are currently three methods of running mhd calculations in \ot: single
cyclic reduction (used primarily in the free boundary equilibrium cases),
successive over relaxation (used primarily in the fixed boundary cases), and the
Green's function solution  (which is too slow for routine usage but provides a
check on  other methods). Two other methods, the time dependent eqdsk method,
and the static equilibrium method, do not solve the Grad-Shafranov equation at
all.

\subsection{CYCRED}

CYCRED refers to the single cyclic reduction scheme used in \ot to solve the
Grad-Shafranov equation. I designed the cycred subroutine especially with the
idea that the FACR(l) method (i.e. Fourier Analysis combined with cyclic
reduction to level l, which is the fastest known method if the grid is fine
enough) would eventually be implemented. The method is usually used to determine
the contribution to the poloidal flux due to the plasma current only.
Contributions from external coils are added in using the Greens function method.
Note that using single (as opposed to double) cyclic reduction eliminates the
restriction on the grid size in the non reduced dimension (in this case R). 
Hence 90 by 129 eqdsks have been generated on occasion. Additionally the
solution is somewhat faster than in the double cyclic reduction case. This is
due to the fact that after a reduction in the Z direction the resulting set of
equations is tridiagonal and can be solved very efficiently with a standard
(pivoting) tridiagonal solver.

\subsection{SORPICRD}
SORPICRD refers to successive over relaxation and Picard iteration. I developed
this method to handle cases where the plasma boundary is known and fixed for all
times. The method uses variable finite difference expressions to accommodate the
plasma boundary. We may add the faster (algebraic) multi-grid method at some
point in the future since subroutines are now available. Inverse methods could
also be used advantageously here.

\subsection{GREEN}
GREEN refers to the Greens function solution method.  A fundamental solution of
the Grad-Shafranov equation is obtained  by solving
\beq
 \nabla ^* G =\frac{\delta (R-R_0) \delta(Z-Z_0)} {2\Pi R} \label{eeq}
\eeq
\begin{equation*}
 G(R=0,Z)=0,\ \ \  G(R,Z) @>>{R,Z \Rightarrow \infty}> 0.
\end{equation*}
The substitution $ G=RA(R,Z)$ yields the equation
\[  
 \overbrace{\frac{\partial^2 A}{\partial{ R^2}}+\frac{1}{R}\pdiff{A}{R} 
  -\frac{A}{R^2}  }^{B_1}
 +\frac{\partial^2 A}{\partial{ Z^2}}
 =\frac{\delta (R-R_0) \delta(Z-Z_0)}{2\Pi R^2} .
\]
Here $B_1$ is Bessel's operator which can be Hankel transformed 
$(R\Rightarrow \zeta)$. A subsequent Fourier transform $(Z \Rightarrow s)$ 
leads to the expression
\[
 A(\zeta,s) = -\frac{J_1(R_0\zeta)}{R_0 (2\pi)^{\frac{3}{2}}} 
  \frac{e^{isZ_0}}{(s^2+\zeta^2)}.
\]
Inversion of the Fourier and Hankel transforms then yields the fundamental
solution
\beq
 G^*=\frac{1}{\pi}\sqrt{\frac{R}{R_0}}\frac{1}{k}
 \left\{ E(k)-(1-\frac{1}{2}k^2)K(k) \right\}
\eeq
\[
 x=\frac{(Z-Z_0)^2+R_0^2+R^2}{2R_0R}, \qquad
 k^2=\frac{2}{x+1},\qquad G^*=2\pi R_0G.
\]
Here $E(k) $ and $K(k)$ are the incomplete  elliptic functions which are
evaluated numerically using the rational fits in Abramowitz\cite{Abramowitz}.
For any toroidal current density the poloidal flux (divided by $ 2\pi $) is then
found by evaluating  the integral
\beq
 \Psi(R,Z)=-\mu_0\int R_0J_\phi(R_0,Z_0)G^*(R,Z;R_0,Z_0)dR_0dZ_0
\eeq
The finite cross sectional area of  each toroidal field coil cannot be
neglected. Consequently the  current density $J_\phi $ is modeled as a sum of
filaments, each with its own $(R_0,Z_0)$. For any given machine the
contribution of the coils can be calculated a priori and this information is
stored in the Greens Table.  The method can be used with the plasma current as
well although in practice it is too slow. Instead we use a finite difference
form of the Grad Shafranov equation to determine that part of the poloidal flux
which is due to the plasma current.  

\subsection{Current Hole Equilibria}

The Greens function solution method, \Eqref{eeq}, lends some insight into the
possibility of generating a current hole equilibrium. Any solution for
$\Psi(R,Z)$, must satisfy the boundary condition that $\Psi =0 $ at the symmetry
axis and at infinity. This condition is built into the integral equation,
\Eqref{eeq}. The integration over $(R,Z)$ in \Eqref{eeq} can be broken up
into three regions: 1) $\Omega_1$, the force free region, 2) $\Omega_2$, the
region inside the plasma but outside the current hole, and 3) $\Omega_3$, the
region outside the plasma that extends to infinity. The integral representation
of $\Psi$ can then be written as  
\begin{multline}
 \Psi(R,Z)=-\mu_0\int_{\Omega_1} R_0J_\phi(R_0,Z_0)G^*(R,Z;R_0,Z_0)dR_0dZ_0 \\
           -\mu_0\int_{\Omega_2} R_0J_\phi(R_0,Z_0)G^*(R,Z;R_0,Z_0)
\end{multline}
The integral over region 1 vanishes since J is zero there. The integral over
region 3 consists of currents flowing in external coils and hence can be
considered to be known a priori. The interesting part is the region 2 integral
which we view as a nonlinear integral equation for $\Psi$. In this region 
$J_\phi $ must have the well known form :
\beq
 J_\phi(R_0,Z_0) = R\pdiff{P}{\Psi} -\frac{F\pdiff{F}{\Psi}}{\mu_0 R}.
\eeq
The pressure P and toroidal flux function $F\pdiff{F}{\Psi} $ can be
parameterized in tems of n unknown constants. For example suppose we model P and
$F\pdiff{F}{\Psi} $ as splines with n and m knots respectively. The objext is
then to find the value of the pressure and $F\pdiff{F}{\Psi} $ at each of the n
and m knot locations respectively. This can be done by supllying n+m equations
to solve for these n+m unknowns.

One way to get these n+m equations is to pick n+m points on the contour of
$\Psi$ that separates reqions $\Omega_1$ and $\Omega_2$. On this contour $ \Psi
= \Psi_c $ and hence we have n+m equations of the form
\begin{multline}
 \Psi_c =-\mu_0\int_{\Omega_1} R_0J_\phi(R_0,Z_0)G^*(R_i,Z_i;R_0,Z_0)dR_0dZ_0\\ 
         -\mu_0\int_{\Omega_2} R_0J_\phi(R_0,Z_0)G^*(R_i,Z_i;R_0,Z_0)dR_0dZ_0
\end{multline}
where i ranges from 1 to n+m.
Because the spline parametrization is done  in terms of
normalized $\Psi $ these equations are in fact nonlinear such that 
an iterative solution is called for. But this can be done
using existing tools available, for example, in \ot.

\subsection{TDEM}\label{tdem}
The idea behind  TDEM (i.e. Time Dependent Eqdsk Mode) is that  often a
significant amount of effort has ben made in generating equilibria for
confinement analysis. These equilibria are  subsequently not used in any
transport studies. Rather, if time dependent MHD information is to be included
in the transport model, then it is generated anew inside the transport code.
This is a non interactive process and generally will not be able to match the
finely tuned equilibria already available without a significant amount of
additional effort. The TDEM mode allows us to circumvent this process by using
the previously available  eqdsk to generate the space and time dependent
coefficients required in the diffusion equations a priori.

To use TDEM in \ot a file, \emph{shotno.tdem}, in netcdf format, containing the
required space and time dependent information must first be created. Netcdf is
used so that this file can be easily  read by other codes (e.g. IDL) as well as
archived.  This file is generated outside of \ot using the FORTRAN code MEPC and
a graphical front end, MEPC.tcl. To generate the file start up MEPC.tcl and
click the help button which will provide all necessary information. Briefly, you
use MEPC.tcl to generate a file, \emph{??} that contains a list of eqdsks to be
processed. MEPC.tcl then passes this list to MEPC which does the actual
calculations and writes the file \emph{??}. A partial third namelist for \ot is
also created, in the file \emph{??} . You must copy this information  into the
third namelist of your \ot input file, \emph{inone}. The TDEM mode is invoked by
setting  \texttt{mhdmethod = `tdem', ieqdsk = 0,ifixshap = 0,} and
\texttt{mhdmode = `no\_coils'}.

\subsection{Non Uniform Grid and Fixed Boundary Calculations}

Solving the Grad-Shafranov equation in cylindrical coordinates can be done
efficiently if the edge of the plasma is approximated by a stair step type
boundary. However this approximation is not always satisfactory and we would
like to  have a more accurate representation of the boundary. Inverse solvers
don not have this problem but typically need a direct solver to supply
information to other codes  (e.g. the ray tracing codes curray and Toray and the
neutral beam code Freya).

In \ot the plasma boundary is accommodated in a direct solver by using variable
spacing in the $(R,Z) $ grid for points adjacent to the plasma boundary. The
boundary thus becomes part of the computational grid in much the same way as it
does for the rectangular boundary case. In this section we document how this is
accomplished computationally and describe some routines that are useful when
dealing with equilibrium files (eqdsks) that were generated using this method.
Because the number of radial grid points inside the plasma will vary as a
function of height $Z$ (and vice versa) we lose the  regular structure that
allows application of fast Poisson solvers (e.g. cyclic reduction). This can be
compensated by the relative ease with which  applicable solvers, such as
successive over relaxation, can be run in parallel however.  

Consider the standard $(R,Z)$ grid with nw radial and nh vertical grid points.
The object of a fixed boundary direct solver is to find $\psi(R,z) $ by solving
the G.S. equation with the plasma boundary fixed and known a priori. Hence this
becomes a simple, elliptic, boundary value problem. For computational purposes
we define an index, $k= (i-1)*nh + j $, that will identify any grid point
$(R_i,Z_j),\  i \in 1..nw, j \in 1,,nh $ within the computational domain. The
first step is to define an indicator array, called \texttt{wzero} in \ot, such
that \texttt{wzero(k)=1} if point k is inside the plasma and 
\texttt{wzero(k)=0} if point k is outside the plasma or on the plasma
boundary. The determination  of wzero is a standard application of concepts from
computational geometry and is not discussed here.

All points in the computational domain, but not on the rectangular boundary of
the $(R,Z)$ grid, will be surrounded by four neighboring points. For grid point
l these neighboring points are designated by lb, ll, lt, and lr for bottom,
left, top, and right respectively.
\begin{figure}[hbt] %note: figure environment not available in slides
 \centering 
 \begin{narrow}{-.50in}{0in}   
  \mbox{\epsfig{file=grdi.eps,height=3in,width=6in}}
  \caption{Irregular grid near plasma boundary. $\lambda_{ll}\Delta R$
  is the distance that boundary point a is to the left of grid point l. Similar
  definitions hold for the remaining $\lambda$}
 \end{narrow} 
 \label{mhd:f1}
\end{figure}

In terms of our single indexing scheme point ll has
index l-nh, point lt has index l+1,point lr has index l+nh and point lb
has index l-1. The computation thus proceeds as follows. For each
point k with \texttt{wzero(k)==1} we check the four neighboring points. Any of
the neighboring points that are outside(or on)  the plasma will have
\texttt{wzero ==0}. This indicates that the grid spacing for that point,
relative to the central point l, will have to be adjusted. There are
in fact 16 possibilities (the number of combinations of 4 points
taken  1, 2, 3, and 4 at a time) that need to be recognized as shown in
the following code segment:

\tiny
\begin{verbatim}
c  there are 16 cases to be accounted for:
c  ll  lt  lr  lb
c   1   1   1   1       case 15 all points inside
c   1   1   1   0            14 lb outside
c   1   1   0   1            13 lr outside
c   1   1   0   0            12 lr,lb outside
c   1   0   1   1            11 lt outside
c   1   0   1   0            10 lt,lb outside
c   1   0   0   1             9 lt,lr outside
c   1   0   0   0             8 lt,lr,lb outside
c   0   1   1   1             7 ll outside
c   0   1   1   0             6 ll,lb outside
c   0   1   0   1             5 ll,lr outside
c   0   1   0   0             4 ll,lr,lb ouside
c   0   0   1   1             3 ll,lt outside
c   0   0   1   0             2 ll,lt,lb outside
c   0   0   0   1             1 ll,lt,lr outside
c   0   0   0   0             0 ll,lt,lr,lb outside
c
c  case 10 has no vertical extent and is pathological
c        5        horizontal
c        0        vertical and no horizontal 
\end{verbatim}
\normalsize
%\enlargethispage{50pt}
We find the intersection of the plasma boundary contour with the $ R,Z $ grid
lines using computational geometry methods . From these intersections we define
the four parameters
${\lambda_{lb},\lambda_{ll},\lambda_{lt},\lambda_{lr}}$
associated with each grid point l. These four parameters represent the fraction
of the uniform $ \Delta R $ and $ \Delta Z $ grid spacing associated with each
of the points surrounding point l. For example, if all four neighboring points
of point l are inside the plasma then all the $\lambda $ are 1.0. If point ll is
outside the plasma (l itself is inside of course) then the boundary contour
crosses the radial line segment between point ll and point l and hence  the
actual distance to the boundary will be less than $ \Delta R $ . $\lambda_{ll}$
then gives the actual fraction of $\Delta R$ that the boundary lies from point l
in the direction of point ll. Similar definitions hold for the other three
values of $\lambda $. Given the distances that the neighboring grid points are
from point l we can define a consistent set of first and second derivatives of
$\Psi $ by expanding $\Psi $ in a Taylor series about the point l:
\begin{multline}
 \Psi(R_i+r,Z_j+z) = \Psi(R_i,Z_j) + \pdiff{\psi}{R}r+
 \pdiff{\psi}{Z}z  + 0.5 \pdiffn{\Psi}{R}{2}r^2\\
 +  0.5 \pdiffn{\Psi}{Z}{2}z^2 +\ddiff{R}\ddiff{Z}\Psi \ rz\label{e1}
\end{multline}

The coordinates $(r,z)$ are to be taken as corresponding to the four neighboring
points relative to the central point l:
\beq
\Psi_{ll}(-\lambda_{ll}\Delta R,0)  \\
\Psi_{lt}(0,\lambda_{lt}\Delta Z)         \\
\Psi_{lr}(\lambda_{lr}\Delta R,0)  \\ 
\Psi_{lb}(0,-\lambda_{lb}\Delta z) .
\eeq
Applied at the four neighboring points ${lb,ll,lt,lr}$ the mixed second
derivative contribution  in the Taylor series  is thus  zero since either $r$ or
$z$  is zero. The remaining derivatives in the Taylor series expansion (all
evaluated at grid point l) are  the (4) unknowns. Hence we need to solve the 4
by 4 system
\beq
 -\pdiff{\Psi}{R} \lambda_{ll} \Delta R 
 +\frac{1}{2}\pdiffn{\Psi}{R}{2}(\lambda_{ll}\Delta R)^2 = \Psi_{ll} - \Psi_l \\
 \pdiff{\Psi}{Z} \lambda_{lt} \Delta Z 
 +\frac{1}{2}\pdiffn{\Psi}{Z}{2}(\lambda_{lt}\Delta Z)^2 = \Psi_{lt} - \Psi_l \\
 \pdiff{\Psi}{R} \lambda_{lr} \Delta R 
 +\frac{1}{2}\pdiffn{\Psi}{R}{2}(\lambda_{lr}\Delta R)^2 = \Psi_{lr} - \Psi_l \\
 -\pdiff{\Psi}{Z} \lambda_{lb} \Delta Z 
 +\frac{1}{2}\pdiffn{\Psi}{Z}{2}(\lambda_{lb}\Delta Z)^2 = \Psi_{lb} - \Psi_l
\eeq
for the derivatives
${\pdiff{\Psi}{R},\ \pdiff{\Psi}{Z},\ 
  \pdiffn{\Psi}{R}{2},\ \pdiffn{\Psi}{Z}{2}} $
for each grid point l that is in the vicinity of the plasma boundary. For points
where all $\lambda = 1$ this is not necessary since the standard diamond
difference scheme then supplies the values for the first and second derivatives.
The required solution is
\beq
 \pdiff{\Psi}{R}\pmb{\bigg \vert_l}&  = &
 -\frac{-\Psi_{lr}\lambda_{ll}^2 + \Psi_l\left ( \lambda_{ll}^2
  -\lambda_{lr}^2\right )+\lambda_{lr}^2\Psi_{ll}}{\lambda_R\Delta R} \\
 \pdiff{\Psi}{Z}\pmb{\bigg \vert_l} &  = &
 \frac{\Psi_{lt}\lambda_{lb}^2 + \Psi_l\left ( \lambda_{lt}^2
  -\lambda_{lb}^2\right )-\lambda_{lt}^2\Psi_{lb}}{\lambda_Z \Delta z}\\
 \pdiffn{\Psi}{Z}{2}\pmb{\bigg \vert_l}&  = & -2
 \frac{-\Psi_{lt}\lambda_{lb} + \Psi_l\left (\lambda_{lt}+\lambda{lb}
  \right ) -\lambda{lt}\Psi_{lb}}{\lambda_Z (\Delta Z)^2} \\
 \pdiffn{\Psi}{R}{2}\pmb{\bigg \vert_l}&  = & -2
 \frac{-\Psi_{lr}\lambda_{ll} + \Psi_l\left (\lambda_{ll}+\lambda{lr}
  \right ) -\lambda{lr}\Psi_{ll}}{\lambda_R (\Delta R)^2} \\
 \lambda_R &  = & \lambda_{ll}\lambda{_lr}(\lambda_{ll}+\lambda{lr}) \\
 \lambda_Z &  = & \lambda_{lb}\lambda{_lt}(\lambda_{lb}+\lambda{lt}).
\eeq
Note that these derivatives reduce to the standard, second order, finite diamond
difference scheme when  all the $ \lambda $ are equal to 1.  Hence we have a
simple method for incorporating the boundary  but retaining the cylindrical
$(R,Z)$ grid. 

One additional complication is that the points inside the plasma are renumbered
so that we generate equations only for those points that we actually will  find
a value of $\Psi$ for. This array, called \texttt{map(m)}, m =1,..nw*nh, in \ot
is defined so that \texttt{map(k)=j} means that the grid point numbered k in the
rectangular computational domain is actually grid point number j when we solve
the discretized GS equation. The solution of the resulting set of equations can
be done most easily with an iterative technique such as SOR.

Since most codes that \ot uses (e.g. Freya, Toray, Curray, etc.) routinely
require the solution $\Psi$ in the entire rectangular domain we also need to
find the solution outside the plasma. Thus in the region outside the plasma, but
inside the rectangular domain, we solve the equation ${\nabla}^{*} \Psi =0.0 $
with the known value of $\nabla\Psi$ on the plasma boundary and some known
values of $\Psi $ on the rectangular $(R,Z)$ boundary. Typically these later
boundary points are assumed to be time independent and given by the solution of
the complete free boundary value problem (as determined by EFIT for example).

The user can verify that the discretized GS equation now takes the
form
\begin{subequations}\label{mhd:eq}
 \begin{gather}
  F_l \equiv \Psi_{ll}\alpha_{ll} +\Psi_l\alpha_l +\Psi_{lr} \alpha_{lr} +
   \Psi_{lt} \alpha_{lt} + \Psi_{lb}\alpha_{lb} -S_l(\Delta R)^2 = 0  
   \label{mhd:eq1}\\
  S_l = -\mu_0R^2\pdiff{P}{\Psi}\pmb{\bigg \vert_l}
   -f\pdiff{f}{\Psi}\pmb{\bigg \vert_l}
   \label{mhd:eq2}     \\
  \alpha_{ll} =2\frac{\lambda_{lr}}{\lambda_r} + \Delta R
   \frac{\lambda_{lr}^2}{\lambda_r R_l}
   \label{mhd:eq3}\\
  \alpha_{lr} = 2\frac{\lambda_{ll}}{\lambda_r} - \Delta R
   \frac{\lambda_{ll}^2}{\lambda_r R_l} 
   \label{mhd:eq4}\\
  \alpha_{lt} = 2\frac{\lambda_{lb}}{\lambda_z}
   \left (\frac{\Delta R}{\Delta z} \right)^2
   \label{mhd:eq5}\\
  \alpha_{lb} = 2\frac{\lambda_{lt}}{\lambda_z}
   \left (\frac{\Delta R  }{\Delta z} \right)^2
   \label{mhd:eq6}\\
  \alpha_l = -2\frac{\lambda_{ll}+\lambda_{lr}}{\lambda_r} + \Delta R
   \frac{\lambda_{ll}^2-\lambda_{lr}^2}{\lambda_r R_l} 
   -2\frac{\lambda_{lt}+\lambda_{lb}}{\lambda_Z}
   \left( \frac{\Delta R}{\Delta Z} \right)^2
   \label{mhd:eq7}
 \end{gather}
\end{subequations}
Here $F_l$ labels the GS equation for grid point (i,j) where $l=(i-1)*nh+ j $
and we assume that there are nw radial and nh vertical grid points. The non
linearity in the source term, $S_l$, in \Eqrefs{mhd:eq1}{mhd:eq2} can be handled
using Picard (also called successive substitution) iteration and this option
exists in \ot. However it is possible to apply a globally convergent Newton
based  approach to these equations as follows.

Let us suppose that for the usual nw (in R) by nh (in Z) grid there are n grid
points interior to the plasma for which we wish to find a solution. Typically $
n = \zeta*nw*nh $ where $\zeta \approx 0.8$ . Hence we have to solve an n by n
system of equations that is both large and sparse. Each of these equations has
the form given by \Eqref{mhd:eq1}. The n by n Jacobian of this system will have
entries of the form
\begin{subequations}\label{mhd:J}
 \begin{gather}
  J(l,l) \equiv \pdiff{F_l}{\Psi_l} = 1.-\frac{(\Delta R )^2}{\alpha_l}
   \pdiff{S_l}{\Psi_l}              \label{mhd:J1}      \\
  J(l,l-nh) \equiv \pdiff{F_l}{\Psi_{ll}} =
   \frac{\alpha_{ll}}{\alpha_l}
   -\frac{(\Delta R )^2}{\alpha_l} \pdiff{S_l}{\Psi_{ll}} \label{mhd:J2} \\
  J(l,l+nh) \equiv \pdiff{F_l}{\Psi_{lr}} =
   \frac{\alpha_{lr}}{\alpha_l}
   -\frac{(\Delta R )^2}{\alpha_l} \pdiff{S_l}{\Psi_{lr}} \label{mhd:J3} \\
  J(l,l+1) \equiv \pdiff{F_l}{\Psi_{lt}} =
   \frac{\alpha_{lt}}{\alpha_l}
   -\frac{(\Delta R )^2}{\alpha_l} \pdiff{S_l}{\Psi_{lt}} \label{mhd:J4} \\
  J(l,l-1) \equiv \pdiff{F_l}{\Psi_{lb}} =
   \frac{\alpha_{lb}}{\alpha_l}
   -\frac{(\Delta R )^2}{\alpha_l} \pdiff{S_l}{\Psi_{lb}}  \label{mhd:J45}
 \end{gather}
\end{subequations}
Because the source term $S$ is evaluated at grid point l all derivatives
of S with respect to the other grid points vanish and hence this term
survives only in \Eqref{mhd:J1}.  Typically the terms
$\pdiff{P}{\Psi}$ and $f\pdiff{f}{\Psi} $ that S contains are
parameterized in terms of a normalized $ \bar{\Psi}$ and the
dependence of $\bar{\Psi} $ on the values of $\Psi$ distributed over
the grid must be included in \Eqref{mhd:J1}. Hence we have
\beq
 \pdiffz{S_l}{\Psi_l}{total} = \pdiff{S_l}{\Psi_l} + \pdiff{S_l}{\bar{\Psi}}
 \pdiff{\bar{\Psi}}{\Psi_l}
\eeq

Typically the magnetic axis is found by fitting a bicubic spline to the current
estimate of $\Psi $ over the (R,Z) grid and then searching for the location
where $\nabla \Psi =0$. This creates an analytically intractable dependence of
$\bar{\Psi}$ on $\Psi_l $ however. Furthermore resolving this dependency
numerically would be too costly since it leads to filling in many of the
elements  in the otherwise sparse Jacobian.  It is in fact possible to create an
approximation to  the true Jacobian by simply neglecting the term
$\pdiff{S_l}{\bar{\Psi}}$. This does not destroy the positive definiteness of
the Jacobian so that we still obtain a direction that leads to decreasing the
residuals $F_l$.  But in practice this slows down the convergence rate of the
outer iterations to a degree where use of the Newton method is no longer
attractive.

The alternative used in \ot is to create a 12 point stencil centered
on the location of the magnetic axis as shown in Fig. \ref{eq:f1}.
\begin{figure}%[hbt] %note: figure environment not available in slides
 \centering 
 %\begin{narrow}{-.50in}{0in}   
 \mbox{\epsfig{file=magax.eps,height=3in,width=4in}}
 %\end{narrow}
 \caption{Coordinates involved in determination of magnetic axis} 
 \label{eq:f1}
\end{figure}
The fiducial point $k_a \equiv (i_a-1)*nh +ja $ is found as the lower
left point of the rectangle \mbox{\{ka, ka+1, ka+1+nh, ka+nh\}}. This rectangle is
is found by searching for two successive R grid points where
$\pdiff{\Psi}{R} $ changes sign and two successive z points where
$\pdiff{\Psi}{Z} $ changes sign. These derivatives are given by \Eqref{??} and hence, in terms of $\Psi$ the twelve grid points
{ka-nh,....ka+1+2*nh} become involved. Given this information  it is easy to find the line
$\overline{de}$  which represents the locus of points on which
$\pdiff{\Psi}{Z} =0 $ . Similarly the line $\overline{bc}$ represents the set of
points where $\pdiff{\Psi}{R} =0 $. (Although shown as tilted in the figure,
most likely these lines will turn out to be vertical and horizontal for most
tokamak equilibria.) The intersection point \textsf{a} represents the estimated
location of the magnetic axis. Since $\Psi $ at the magnetic axis is thus shown
to depend on the 12 given points, it is straightforward to incorporate that
dependency into the Jacobian. Using this approach it is possible to give an
analytic form for the elements $\pdiff{S_l}{\bar{\Psi}}$ and
$\pdiff{\bar{\Psi}}{\Psi_l}$, but since interpolation of the normalized
$\bar{\Psi}$ is required anyway we use a numerical perturbative approach for
this term.

Hence the Jacobian, J, is known and determined semi analytically. To continue we
need to solve the n by n set of equations
\beq \label{mhd:c}
 \underline{\underline{J}} * \underline{\psi} = - \underline{F}.
\eeq
Here $\underline{F} $ represents the column of residuals with elements $F_l$
given by \Eqref{mhd:eq1}, and $\psi$ represents the incremental corrections to
the $\Psi$ values to be made if the full Newton step is accepted. Typically a
line search along the Newton step is required in order to avoid divergence of
the iterations.  The near optimal method for this search used in \ot is given in
Ref. \ref{?}. The n by n system given in \Eqref{mhd:c} is large (for a 129 by
129 grid we will have about 13000 values of $\Psi $ to determine) but sparse.
The bandwidth is at least 2* nh and the matrix J does not have a regular
structure due to the fact that the number of grid points in the radial and
vertical directions is not uniform. In \ot we currently use a sparse matrix LU
factorization technique (y12m)  to solve the equations. Since there is some 
fill-in encountered during the solution  we are limited by hardware memory
requirements however. A better approach is to use an iterative solver such as a
bi-conjugate gradient method but such software was not available to me. (A
sparse multigrid algorithm does  not exist to my knowledge). At this time, for
grid sizes larger than 129 by 129 the Newton method is not used. Instead the
Picard iteration method is relied on for the outer iterations . In fact high
resolution eqdsk are often created by running a 65 by 65 EFIT type  eqdsk
through the code to produce eqdsk as large  as 513 by 513 using Picard
iteration.

An example of an ITER type equilibirum calculated using the SOR method
on a 513 by 513 grid is given in Fig. \ref{fig:iter}.
\begin{figure}[hbtp] %note: figure environment not available in slides
 \centering 
 %\begin{narrow}{-.50in}{0in}   
 \mbox{\epsfig{file=resid_example.eps,height=4in,width=5in}}
 \epsfig{file=fixbd_err.eps,height=4in,width=5in}
 %\end{narrow}
 \caption{(a):Absolute value of  residuals calculated using the variable grid
 method. As is to be expected the residual is zero near the plasma boundary
 (since that condition is built into the solver ) - (b):ignoring the fact that a
 non uniform grid was used to generate the solution would lead one to conclude
 that large errors exist near the plasma boundary} 
 \label{fig:iter}
\end{figure}

\subsubsection{The Exterior Problem}

The region exterior to the plasma but inside the rectangular $(R,Z)$ domain must
also be considered in order to generate an eqdsk. In this exterior region we
solve the GS equation with zero current density, subject to the boundary
conditons that grad psi is given on the plasma surface and psi is given on the
edges of the rectangular domain. The gradient of psi on the plasma surface is
computed from the interior solution. The values of psi on the rectanglar
boundary are assumed given. The finite difference representation of the GS
equation for grid points outside the plasma but near the plasma boundary is
slightly more complicated than the interior case. Consider Fig. \ref{f1e} where
we assume that grid points lb and lr are inside the plasma. Three gradients of
$\Psi$ at the points a, b, and c respectively are used to generate the equations
for the required derivatives at exterior grid point l. In the figure points a
and c are intersections of the plasma boundary with the grid lines. Point b was
not used in the interior calculations. It is defined by the fact that the
direction of the gradient at point b passes through grid point l. Depending on
the particular geometry it is possible that point b coincides either with point
a or point c. This leads to degeneracy in the equations described below.

\begin{figure}[hbt] %note: figure environment not available in slides
 \centering 
 %\begin{narrow}{-.50in}{0in}   
 \mbox{\epsfig{file=exterior.eps,height=4in,width=5in}}
 %\end{narrow}
 \caption{An example of an exterior grid point l near the plasma boundary. Three
 gradients of psi at the points a, b, and c  are used to generate the equations
 for the required derivatives at grid point l.} 
 \label{f1e}
\end{figure}

We expand $\Psi$ in a Taylor series about the exterior grid point l,
\Eqref{e1}, and differentiate the result to get
\beq \label{e2}
 \pdiff{\Psi(r,z)}{R} =\pdiff{\Psi}{R} \Biggr \vert_l
  +\pdiffn{\Psi}{R}{2} \Biggr \vert_lr
  + \ddiff{R}\ddiff{Z}\Psi  \Biggr \vert_l z
\eeq
A similar expression holds for the derivative in $Z$. Applied at the
points a, b, and c in Fig. \ref{f1e} we get the five equations in the five
unknowns $\pdiff{\Psi}{R}\vert_l$,$\pdiff{\Psi}{Z}\vert_l$,
$\pdiffn{\Psi}{R}{2}\vert_l$, $\pdiffn{\Psi}{Z}{2}\vert_l$,
$\ddiff{R}\ddiff{Z}\Psi \vert_l $ .

The RHS of \Eqref{?} represents the assumed known gradient of $\Psi$ on the
plasma boundary. The last two equations are obtained by applying  the Taylor
series at the grid points ll and lt. There are 16 different cases that can
arise, as previously dicussed for the interior solution. Equations similar to
the above set have to be generated for each of the cases. It is always possible
to generate 5 equations however. The details are given in subroutine ??.

The remaining issue is how to generate the values of $\nabla \Psi $ on the
plasma boundary so they can be used in solving the above equations. In \ot this
is accomplished in an iterative manner. We first generate an exterior solution
by replacing the  $\nabla \Psi $ boundary condition on the plasma surface with
the condition that $\Psi = 0$ on the plasma surface. Combined with the interior
soltuion this gives us a soltuion over th entire rectangular (R,Z) domain. We
bicubic spline fit this solution and from the spline fit determine the value of
$\nabla \Psi $ at all required boundary points. The exterior problem is then
solved a second time with the  $\nabla \Psi $ boundary condition applied. The
interior soltuion is not changed in this step. This process could be iterated
but in practice this is not necessary.  (The boundary condition on the outer,
rectangular boundary  is an approximation, not based on coil currents, and hence
any refinement would be ludicrous).

\subsection{Transport-MHD interface}

Proper communication between the transport system of equations and the
equilibrium (GS) equation is very important in order to get an efficient, stable
transport/equilibrium iteration cycle. The time step used to evolve the
transport equations, $\Delta t_t$, and the time step between equilibrium
calculations,$\Delta t_{eq}$, are independent quantities in \ot (but $ \Delta
t_{eq} \ge \Delta t_t$ must be observed).  We typically do not call the
equilibrium solver at every transport time step. Instead, a number of transport
time steps are taken before a new equilibrium is calculated. The actual number
of transport time steps between equilibrium calculations is initially set by the
user and is then dynamically decreased if necessary to assure convergence.
(There are actually a number of  options for setting the intervals $t_t$ and
$t_{eq}$ in \ot; see the input instructions in cray102.f .)

Evolution of the transport quantities depends on space and time dependent
metrics obtained from the equilibrium solution. Initially the space dependence
is known from a start up equilibrium but the time dependence is not known at
all. The actual method used to overcome this difficulty in \ot is as follows.

Initially we have an equilibrium solution (in the form of an eqdsk) from a
previous fixed boundary MHD calculation. The reason we must start with a fixed
boundary eqdsk (meaning an eqdsk that was generated using the fixed boundary
code), is that the metrics (see below), do not converge as well at the plasma
edge  if a standard EFIT type eqdsk is used. The fixed boundary code actually
solves a problem which is different than the problem that is solved by EFIT. The
difference is due to the way that the plasma edge is handled. The fixed boundary
code solves two boundary value problems, the interior problem, with the plasma
boundary as the mathematical boundary and the exterior problem, with the plasma
boundary as one boundary and the boundary of the rectangular grid as the outer
boundary. For the interior problem we specify a value of psi at the plasma
boundary. For the exterior problem the boundary condition on the plasma boundary
is that grad psi is given. This value of grad psi is determined from the
interior solution near the plasma boundary. On the rectangular boundary of the
exterior problem the values of psi are specified. In \ot these values are
assumed to be equal to the values at the initial time and do not change in time.
Effectively this means that some pseudo  coils with appropriate, perhaps non
physical, currents are present. One could supply actual machine dependent coils
and use Greens function methods to refine the exterior  problem but that has not
been done at this time.



The  eqdsk is used to construct the initial geometry factors, current density
and transport grid:
\begin{subequations}\label{mhd:a}
 \begin{gather}
  \Phi\vert_{t_0}  = 2\pi\int q d\Psi \label{mhd:a1} \\
  \rho\vert_{t_0}  = \sqrt{\frac{\Phi}{\pi B_{t0}}}   \\
  F\vert_{t_0} = \frac{R_0 B_{t0}}{f(\Psi)}  \\
  G\vert_{t_0} =  \left\langle
   \vert\nabla\rho\vert^2\frac{R_0^2}{R^2}\right\rangle \\
  H\vert_{t_0} =  \left\langle \frac{F}{\frac{R_0^2}{R^2}}\right\rangle .
 \end{gather}
\end{subequations}
The profiles required in the RHS of Eqs.~\eqref{mhd:a} are  available from the
startup eqdsk on a uniform $\Psi$ grid of length nw. Note that these quantities
are functions of both space and time. Initially the time dependence is only
known at the single  time point as indicated. Thus to solve the transport
equations for the duration of the first equilibrium cycle we assume that the
metrics and other quantities are constant in time as given by
Eqs.~\eqref{mhd:a}.

The transport equations are evolved, using time steps $\Delta t_t$, from some
initial time $t_0$ to time $t_1 =  t_0 +\Delta t_{eq} $. At time $t_1$ the
equilibrium solver is called with the evolved $\pdiff{P}{\Psi} $ and
$\langle\frac{J_{\phi}R_0}{R}\rangle$. The total pressure profile normally
includes beam and fusion contributions (using $\frac{2}{3}$ of the stored energy
density for the fast ion contributions). These quantities are evolved on the
$\rho\vert_{t0} $ grid and need to be mapped into the appropriate $\Psi $ grid
for use in the equilibrium solver. The conversion factor, $\pdiff{\rho}{\Psi}$,
is obtained from the transport results at time $t_1$. As shown in section
\ref{sec:transport} \ot evolves the quantity $u^4 \equiv FGH  \rho B_{p0} $.
From the definition of $B_{p0}$ we then obtain the $\Psi$ to $ \rho $ mapping:
\begin{subequations}\label{mhd:k}
 \begin{gather}
  B_{p0}\vert_{t_1} = \frac{u^4\vert_{t_1}}{(FGH\rho)\vert_{t_0}} \\
  \pdiff{\rho}{\Psi}\vert_{t_1} = \frac{1}{R_0 B_{p0}\vert_{t_1}}.
   \label{mhd:k1}
 \end{gather}
\end{subequations}  
Since the time dependence of the parameters associated with the
evolution of $ u^4 $ during the time interval $ t_0 $ to $ t_0 + \Delta
t_{eq} $ was neglected, these quantities are only an approximation to
the desired results. \Eqref{mhd:k1} gives us an initial
approximation of the $\rho$ to $\Psi $ mapping required to evaluate
$\pdiff{P}{\Psi}$ for example. 

To get a form for $f\pdiff {f}{\Psi} $ driven by transport we use
$\left\langle\frac{J_\phi R_0}{R}\right\rangle$. This quantity is also obtained from $u^4$:
\begin{subequations}\label{mhd:l}
 \begin{gather}
  \left\langle\frac{J_\phi R_0}{R}\right\rangle\vert_{t_1}
  =  \left [ \frac{\mu_0}{H\rho} \ddiff{\rho}\left(\frac{u^4}{F}\right) \right ]
  \Biggr \vert_{t_1}  \label{mhd:l1}
 \end{gather}
\end{subequations}
We map  both  $\pdiff{P}{\Psi}$ and $\langle\frac{J_\phi R_0}{R}\rangle$ to the
$\Psi$ grid by integrating \Eqref{mhd:k1}:
\beq
 \left [ \Psi(\rho_i) -\Psi(0) \right ] \vert_{t_1} =\left (
 R_0\int_0^{\rho_i} B_{p0}\, d\rho\right ) \Biggr \vert_{t_1} \label{mhdm}
\eeq
where $\rho_i$ is $\rho $ at radial transport grid point i. \Eqref{mhdm} is
evaluated in subroutine \texttt{psirho} (cray209.f) and produces a psi grid that
corresponds to the transport rho grid (the radial transport rho grid extends
from the magnetic axis to the plasma edge and has nj grid points). The toroidal
current density, \Eqref{mhd:n1}, is now flux surface averaged  to produce an
expression for $f\pdiff{f}{\Psi}$:
\begin{subequations}\label{mhd:n}
 \begin{gather}
  J_\phi= -R\pdiff{P}{\Psi}-\frac{f\pdiff{f}{\Psi}}{u_0R} \label{mhd:n1}\\
  \left (f\pdiff{f}{\Psi}\right )\Biggr\vert_{t_1} = 
  -\frac{u_0}{\left\langle\frac{R_0}{R^2}\right\rangle}\left(R_0\pdiff{P}{\Psi} 
  +\left\langle\frac{J_\phi R_0}{R}\right\rangle \right )\Biggr \vert_{t_1}
  \label{mhd:n2}
 \end{gather}
\end{subequations}

Armed with $f\pdiff{f}{\Psi} $ from \Eqref{mhd:n2}, and 
$\pdiff{P}{\Psi} =\pdiff{P}{\rho}\pdiff{\rho}{\Psi} $ from \Eqref{mhd:k1} we are
finally able to solve the GS equation. We note again that both of these
quantities are known on a $\Psi$ grid which corresponds to the radial transport
$\rho $ grid with nj (typically 51) grid points. For the initial step the radial
transport grid and all metrics required are given by the values at time $t_0$,
Eqs.~\eqref{mhd:a}. Note that there are additional quantities that depend on
flux surface averages, for example the trapped particle fraction. These
quantities are correctly handled by the MHD/transport coupling of the code but
are not dealt with explicitly here.

Having obtained a new equilibrium solution at time $t_0 +\Delta t_{eq}$  with the
given $f\pdiff{f}{\Psi} $ and $\pdiff{P}{\Psi}$ (which still have
dependence on parameters defined at time $t_0$) we need to check for
convergence of the transport/equilibrium cycle. This is done by
evaluating Eqs.~\eqref{mhd:a} with the newly calculated equilibrium.
Comparison of the new parameter set at time $t_0 +\Delta t_{eq}$ with the
original set at time ${t_0} $ then yields information as to whether
or not our MHD/transport system is sufficiently converged to
continue on with the next time interval from $t_0 + t_{eq}$ to $t_0 +
\Delta t_{eq} + \Delta t_{eq_1}$ . Here $\Delta t_{eq_1}$ is a new equilibrium time
interval which may be determined by the code based on how rapidly
parameters are changing.
A minimum maximum error criteria is used to establish if our
assumptions
of time independent metrics was justified. At this point, converged or
not, we have estimates of the parameter set, Eqs.~\eqref{mhd:a}, 
at two times, $t_0$ and $t_0 + \Delta t_{eq} $. If the system is not
sufficiently converged then we go back to time $t_0$ and start over
again. But this time all the parameters,including $\rho$ are linearly
interpolated in time to give an estimate of the time
dependence.(Section \ref{sec:?} deals with the issues of a time dependent $\rho$
grid). Arriving in this way once again at time $t_0 + \Delta t_{eq} $ we back
average the new estimate of the parameter set with the old one to aid
convergence. This process is allowed to continue for a user specified
number of times. If the mhd/transport cycle is still not converged
then the code either quits or (default) prints a warning message and
continues on the the next equilibrium cycle anyway. New equilibrium
cycles are just repeats of the above process, with the profiles
linearly extrapolated on the first attempt to span the time interval
$t_i + \Delta t_{eq_i}$. 


