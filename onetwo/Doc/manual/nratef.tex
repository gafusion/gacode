
   %neutron rate calculation details for d(d,n)he3 fast ion calcs1/31/95
   %SETUP FOR LATEX2E


  % \documentclass[12pt]{report}
   \documentclass{slides}
   \newcommand{\beq}{\begin{eqnarray}}
   \newcommand{\eeq}{\end{eqnarray}}
%   \usepackage{amssymb,amstext} 
   \usepackage{amsmath} 
   \newcommand{\vp}{ V^{\prime}}
   \newcommand{\pdiff}[2]{\frac{\partial{ #1}}{\partial{ #2}}}
   %useage \pdiff{#1}{#2}
   \newcommand{\ddiff}[1]{\frac{\partial}{\partial #1}}


%  \setmargins{1in}{1in}{6.5in}{10in}{0pt}{0mm}{0pt}{0mm}
   \textwidth6.5in
   \textheight9.5in
  % \hoffset -1.0in
   \voffset -1.0in

    \begin{document}
      % \large    %type size ,must be given after begin{document} large not appropriate for slides
        \boldmath
        %\section{$D_f(D_t,n)He^3 $ Neutron Rates } %no section comman in slides
        \begin{center}Review of $ D_f(D_t,n)He^3 $ Neutron Rate Calculations \end{center}
        \bigskip
	\hrule height2pt
        \bigskip
	The  fast ion distribution function is azimuthally symmetric and
        asymptotically approaches 
          \begin{eqnarray}  %use eeqnarray for multiple line formulas
           f_b(v,\zeta)= \frac{\dot S \tau_s}{v^3+v_c^3}
              \left[exp\left(-\tau_s\int_{v}^{v_b}\frac{v^2dv}
               {(v^3+v_c^3)\tau_{cx}}\right)\right] \nonumber \\
               \sum_{l=0}^{\infty}\frac{2l+1}{2}P_l(\zeta)P_l(\zeta_b)
               \left[\frac{v^3}{v_b^3} \left(\frac{v_b^3+v_c^3}{v^3+v_c^3}
                \right)\right]^{\frac{1}{6}l(l+1)Z_2}
          \end{eqnarray}

	The probability against charge exchange  
	 \beq
           P_{cx}(v)=  exp\left(-\tau_s \int_{v}^{v_b}\frac{v^2dv}
              {(v^3+v_c^3)\tau_{cx}}\right) 
         \eeq
         is neglected $(\tau_{cx} \rightarrow \infty)$  or can be included
         in a number of ways:
         \begin{itemize}
          \item  constant $\tau_{cx}$ (Callen)
            \beq
               P_{cx}(v)=\left(\frac{v_b^3+v_c^3}{v^3+v_c^3}\right)
                      ^{-\frac{\tau_s}{3\tau_{cx}}}
	    \eeq
          \item analytically integrable fit (Gaffey)
             \begin{eqnarray*}
                \frac{1}{\tau_{cx}}=n_n\sigma_{cx}(v)v  \\
    \sigma_{cx}=\frac{ 21\cdot10^{-16} cm^2}{0.2(0.5Mv^2)+1} 
              \end{eqnarray*}
              \begin{eqnarray}
          P_{cx}=\left[\left(\frac{v^3+v_c^3}{v_b^3+v_c^3}
                     \right)^{\frac{1}{2}}   
                     \left(\frac{v_b+v_c}
                    {v+v_c}\right)^{\frac{3}{2}}\right]
               ^{\frac{\tau_s}{3\tau_{cx}}} \hspace{2.5in}  \nonumber \\
                   exp\left[\frac{-\tau_s}{3\tau_{cx}} \left( 
  \sqrt{3}\arctan{\left(\frac{2v_b-v_c}{\sqrt{3v_c}}\right)}
             \right.\right. \hspace{2.0in} \nonumber \\
   \left.\left. -\sqrt{3}\arctan{\left(\frac{2v-v_c}{\sqrt{3v_c}}\right)}
               \right)\right] \hspace{1.0in}
             \end{eqnarray}
           \item numerical quadrature with experimental $\sigma_{cx}$
                (currently using Freeman and Jones)
         \end{itemize}
	The fast ion distribution function,integrated over polar and
	azimutahl angles is 
          \beq
           f_b(v)v^2dv=\frac{\dot S \tau_s v^2dv}{ v^3+v_c^3}
          \eeq
	which is expressed in terms of energy as
	 \beq
             f_b(E)dE=\frac{ \dot S \tau_s dE}{2E(1+{\left(
                        \frac{E_c}{E}\right)}^{\frac{3}{2}})}
         \eeq
           The thermal $D_t $ distribution has the form of a shifted
 	   Maxwellian:
           \begin{eqnarray}
	    f_t=\left({\frac {\beta}{\pi}}\right)^{\frac{3}{2}} exp\left[-\beta
                          (\vec{v}-\vec{w})^2\right] \nonumber\\
            f_t \stackrel{\beta \rightarrow \infty}=
            \frac{\delta (v-w) \delta(\zeta - 1)}{2\pi v^2
                                 \sqrt{1-\zeta^2}}  \\
            \beta = \frac{m}{2kT} \nonumber
           \end{eqnarray}
	  The neutron rate density $[\frac{1}{cm^3sec}]$ is given by
           \beq
           R=\int \vec{dv_t}\vec{dv_b}f_t(v_t)f_b(v_b)v_{rel}\sigma({v_{rel}}) 
           \eeq

           Current approximate formulation used in ONETWO assumes $v_t=0,
		w=0$ which allows use of Eq(7) for $f_t$,and Eq(6) for $f_b$.

                The neutron rate density becomes
                 \beq
                         R=\frac{n_d \dot S \tau_s}{\sqrt(2m_b)}
                           \int\frac{dE\sigma_{DD}(E)}{E^{\frac{1}{2}}
                           (1+{(\frac{E_c}{E})}^{\frac{3}{2}})}
                 \eeq
                
               Using the approximate $\left( NRL\  Formulary \right)  $ 
               cross section
               \beq
                   \sigma_{DD} =\frac{ c
                    \epsilon^{\frac{-b}{\sqrt{E}}}}{E}
                \eeq
	       an analytic expression for $R$ is obtained
               by asumming that $\frac{E_c}{E}=\frac{E_c}{E_b}$ in
	       Eq(9). 
             

               The neutron rate density becomes
         \beq
            R=\frac{ \dot S \tau_s c k n_D e^{\frac{-b}{\sqrt{E_b}}}}
          {  (1+\left({\frac{E_c}{E_b}}\right)
                     ^{\frac{3}{2}})b}
         \eeq

	 This is the form used in ONETWO and OVER ESTIMATES R 
         (typically over compensates for zero ion temp assumption)

	A more accurate calculation uses the Bosh and Hale
        cross section and does the integral in Eq.(9)  numerically.
    
        The combination of larger NRL cross section coupled
        with the approximation of the integrand results in
        neutron rates that are too high!
    \beq
         \lmoustache\lgroup fdv \rgroup = \rmoustache fdv
    \eeq
    \end{document}
 The cross section $\left( NRL\  Formulary \right)  $ is
               \beq
                   \sigma_{DD} =\frac{ 2.18\cdotp10^{-22}\cdotp
                    \epsilon^{\frac{-47.88}{\sqrt{E}}}}{E}
                \eeq
           Where   rate of $D_f$ energy transfer to electrons is
                 \beq
                   { \nu_{\epsilon}^{\frac{d}{e}} =
                    7.59(\frac{n_e}{10^{13}cm^3})}
                    {{\left(\frac{T_e}{kev}\right)}^{-\frac{3}{2}}}
                 \eeq
                 Note that $ \nu_{\epsilon}^{\frac{d}{e}} = 
                     \frac{2}{\tau_s} . $