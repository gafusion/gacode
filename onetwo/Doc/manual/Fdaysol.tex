\renewcommand{\textfraction}{0.10}
\renewcommand{\topfraction}{0.90}  % not greater than 1- textfraction
\renewcommand{\bottomfraction}{0.65} % not greater than 1- textfraction
\renewcommand{\floatpagefraction}{0.60}

\section{Solution of Faraday's Law: Some observations}

The solution of Faraday's and Ohm's law places restrictions on the problem which
may preclude a steady state solution from existing in some of the restrictred
modeling that can be done with Onetwo.  The effect is most easily demonstrated
when the pressure profiles and the MHD equilibirum are kept fixed in time but
the current density is allowed to evolve to steady state. In Fig.~\ref{F1a} we
present an example where a steady state solution (e.g. a solution with a flat
parallel electric field)  could not be found  with the assumed  given total
toroidal current of 10 MA. A simple test, keeping the equilibrium constant, but
simply varying the total current, yields  the results shown in the figure.  Ssq
represents the residual of Faraday's  law. A true solution of the equation will
have ssq $ \approx 0 $. As is seen, the solution approaches a zero residual
value only if the total current is increased to about 13.5 MA. From the
smoothness of the curve it is concluded that the non-linear solver is in fact
finding the best solution in each case (otherwise we would expect the curve to
have a fluctuating behavior). For each value of the total current the solver is
finding the best approximate solution in the sense that the residual of
Faraday's law is minimized.

The reason why we can not obtain a valid solution until the total current is
increased to 13.5 MA is due to the unique nature of the bootstrap current.
Inside of $ \rho \approx 0.25 $ the ohmic current is required to be positive 
Near the magnetic axis the bootstrap current forces the\ldots 

\begin{figure}[hbt] %note: figure environment not available in slides
 \centering  
 %\begin{narrow}{-.50in}{0in}
 \centering   
 \mbox{\epsfig{file=ssq.eps,height=3in,width=3in}} \\
 \vspace{.5in}
 \begin{narrow}{-.50in}{0in}
 \mbox{\rotatebox{90.}{\epsfig{file=ohmic.eps,height=3.5in,width=3.5in}}}
 \mbox{\rotatebox{90.}{\epsfig{file=etor.eps,height=3.5in,width=3.5in}}}
 \end{narrow}
 \caption{(a)Increasing the total current causes the non linear solver to
 smoothly approach the solution with zero residual. This is indicative of the
 fact that the solver is finding the best, approximate ``solution'' in each
 instance. Part (b) and (c) show the corresponding ohmic current profiles and
 electric fields.}
 \label{F1a}
\end{figure}

The above results apply to the  \emph{ITER\_FEAT} case with 6keV pedestal
temperature. It is interesting to note that a similar case with 5keV pedestal
temperature does not display this behavior. In Fig.~\ref{F2a} we show the GLF23
evolved steady state temperatures associated with these two cases. Note the 
gradient in $T_e$ near $\rho = 0.75 $ in the 6keV case.

\begin{figure}[hbt] %note: figure environment not available in slides
 \centering  
 %\begin{narrow}{-.50in}{0in}
 \centering   
 \mbox{\rotatebox{90.}{
  \epsfig{file=te_iter_feat.eps,height=3.5in,width=3.5in}}} \\
 \vspace{.5in}
 \begin{narrow}{-.50in}{0in}
 \mbox{\rotatebox{90.}{
  \epsfig{file=etor_iter_feat.eps,height=3.5in,width=3.25in}}}
 \mbox{\rotatebox{90.}{
  \epsfig{file=curden_iter_feat.eps,height=3.5in,width=3.5in}}}
 \end{narrow}
 \caption{(a) The steady state temperatures obtained with GLF23 for the 5 and 6
 keV pedestal cases. (b) The corresponding electric field. The electric field
 for the 6 kev case is the one shown in Fig.~\ref{F1a} for the 10 MA case and is
 not a true solution as explained in the  text. (c) The corresponding  current
 density profiles. For simplicity the beam and rf  driven current are not shown.
 The 6 keV pedestal case is the same as the 10 MA case in Fig.~\ref{F1a}.}
 \label{F2a}
\end{figure}

\begin{table}
 \begin{centering}
  \begin{tabular}{ccccc}
   \multicolumn{5}{c}{\bfseries Shot 84293} \\
   \multicolumn{5}{c}{\bfseries (L mode) }  \\ 
   FW MHz& $P_e ,kW $ & $P_i, kW  $ & $ I_{CD}, kA  $ & $P_a $ \\ \hline
   60  & 501& 499 & 56& 41 \%  \\
   60  & 491& 509 & 51& 80 \%  \\
   %83  & 710/650 & 290/350&  87/80 & 0 & \\
   83  & 744/607& 256/398  &  80/71 & 31\%   \\
   83  & 728 & 272?   &  74 & 80\%   \\
   117  &950  &50 & 111 & 29\%  \\
   117  &895  &105 & 99 & 82\%  \\
   \\ \hline 
   & \\
   \multicolumn{5}{c}{\bfseries Shot 111221} \\
   \multicolumn{5}{c}{\bfseries (H
   mode)} \\ 
   FW MHz& $P_e, kW$ & $P_i ,kW $ & $ I_{CD}, kA $&  $P_a $ \\ \hline
   60  &340/287& 660/773 & 22/16 & 95\% \\
   83  & 420 & 580 & 30  & 99\% \\
   117 & 395/212 & 605/788 & 30/18 & 99\% \\ \hline
  \end{tabular}
  \caption{Heating and Current drive results for DIII-D L and H
  mode cases. The second number for electron and ion absorbed power and
  current drive indicates results obtained using \emph{Transp}
  profiles. $P_a$ is the percent of injected power absorbed. For the
  L mode, shot results are quoted for 6  and 100 edge reflections (with
  the larger value of $P_a$ corresponding to 100 edge reflections and
  the smaller value to 6 reflections).}
  \label{t1a}
 \end{centering}
\end{table} 
