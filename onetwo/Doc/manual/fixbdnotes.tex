\subsection{Non Uniform Grid and  Fixed Boundary Calculations}
 Solving the Grad-Shafranov equation in cylindrical coordinates can be
done efficiently if the edge of the plasma is approximated by a stair
step type boundary.
 However this approximation is not always satisfadtory and we would
like to 
have a more accurate representation of te boundary. Inverse solvers
don not have this problem but typically need a direct solver to supply
information to other codes 
(e.g.the ray tracing codes curray and toray and the neutral beam code
freya).

In \ot the plasma boundary is accomodated in a direct solver by using
variable spacing in the $(R,Z) $ grid for points adjacent to the
plasma boundary. The boundary thus becomes part of the computational
grid in much the same way as it does for the rectangular boundary
case. In this section we document how this is accomplished
computationally and describe some routines that are useful when
dealing with equilibirum files (eqdsks) that were gnerated using this
method. Because the the number of radial  grid points inside the plasma will
vary as a funtion of height Z ( and vice versa) we loose the  regular
structure that allows application of fast Poisson solvers (eq cyclic
reduction). This can be compensated by the relative ease with which 
applicable solvers, such as succesive over relaxation, can be run in
parallel however.  

Consider the standard $(R,Z)$ grid with nw radial and nh
verical grid points. The object of a fixed bounday direct solver is to find
$\psi(R,z) $ by solving the G.S. equation with the plasma boundary
fixed and known apriori. Hence this becomes a simple, elliptic,
boundary value problem. For computational purposes we define an index,
$k= (i-1)*nh + j $, that will identify any grid point $(R_i,Z_j), i \in 1..nw,
j \in 1,,nh $ within the computational domain. The first step is
to define an indicator array, called wzero in \ot, such that
wzero(k) =1 if point k is inside the plasma and wzero(k) = 0 if point
k is outside the plasma or on the plasma boundary. The determination 
of wzero is a standard
application of concepts from compuational geometry and is not
discussed here.

 All points in the computational domain but not on the
rectagualr boundary of the (R,Z) grid will be surrounded by four
neighbooring points. For grid point l these neighbooring points are
designated by lb,ll,lt,and lr for bottom,left,top, and right
respectively.  in terms of our single indexing scheme point ll has
index l-nh, point lt has index l+1,point lr has index l+nh and poin lb
has index l-1. The computation thus proceeds as follows. For each
point k with wzero(k) =1 we check the four neighbooring points. Any of
the neighbooring points that are outside(or on)  the plasma will have
wzero =0. this indicates that the grid spacing for that point,
relative to the central point l, will have to be adjusted. There are
in fact 16 possibilities ( the number of combinations of 4 points
taken  1,2,3 and 4 at a time) that need to be recognized. 

\tiny

\begin{verbatim}

 using the 5 point diamond difference we have
c                             lt
c
c               ll             l          lr
c
c                             lb
c any combination of the four points (ll,lt,lr,lb) can be on the
c plasma boundary.
c  there are 16 cases to be accounted for:
c  ll  lt  lr  lb
c   1   1   1   1       case 15 all points inside
c   1   1   1   0            14 lb outside
c   1   1   0   1            13 lr outside
c   1   1   0   0            12 lr,lb outside
c   1   0   1   1            11 lt outside
c   1   0   1   0            10 lt,lb outside
c   1   0   0   1             9 lt,lr outside
c   1   0   0   0             8 lt,lr,lb outside
c   0   1   1   1             7 ll outside
c   0   1   1   0             6 ll,lb outside
c   0   1   0   1             5 ll,lr outside
c   0   1   0   0             4 ll,lr,lb ouside
c   0   0   1   1             3 ll,lt outside
c   0   0   1   0             2 ll,lt,lb outside
c   0   0   0   1             1 ll,lt,lr outside
c   0   0   0   0             0 ,ll,lt,lr,lb outside
c
c  case 10 has no vertical extent and is pathological
c        5        horizontal
c        0        vertical and no horizontal 

\end{verbatim}
 \normalsize

 We find the intersection of the plasma boundary contour with the $ R,Z $ grid
 lines using computational geometry methods . From these intersections we define
 the four parameters
 ${\lambda_{lb},\lambda_{ll},\lambda_{lt},\lambda_{lr}}$
 associated with each grid point l. These four parameters represent
 the fraction of the uniform $ \Delta R $ and $ \Delta Z $ grid spacing
 associated with each of the points surrounding point l. For example,
 if all four neighbooring points of point l are inside the plasma then
 all the $\lambda $ are 1.0. If point ll is outside the plasma (l
 itself is inside of course) then the boundary contour crosses the
 radial line segement between point ll and point l and hence  the
 actuall distance to the boundary will be less than $ \Delta R $ .
 $\lambda_{ll}$ then gives the actual fraction of $\Delta R$ that the
 boundary lies from point l in the direction of point ll. Similar
 definitions hold for the other three values of $\lambda $.

 Given the distances that the neighbooring grid points are from point
 l we can define a conssitent set of first and second derivativs of
 $\Psi $ by expanding $\Psi $ in a Talor series about the point l:
 A Taylor series of $\Psi $ about point l can now be expressed as 

         \beq
             \Psi(R_i+r,Z_j+z) = \Psi(R_i,Z_j) + \pdiff{\psi}{R}r+
             \pdiff{\psi}{Z}z  + 0.5 \pdiffn{\Psi}{R}{2}r^2 \\
            +  0.5 \pdiffn{\Psi}{Z}{2}z^2 +\ddiff{R}\ddiff{Z}\Psi \ rz
          \eeq


   The coordinates $ (r,z) $ are to be taken as corresponding to the
   four neighbooring points relative to the
   central point l:
   \beq
   \Psi_{ll}((-\lambda_{ll}\Delta R,0))  \\
   \Psi_{lt}((0,\lambda_{lt}\Delta Z))         \\
  \Psi_{lr}((\lambda_{lr}\Delta R,0))  \\ 
  \Psi_{lb}((0,-\lambda_{lb}\Delta z))  \\
   \eeq
 Applied at the four neighbooring points ${lb,ll,lt,lr}$ the mixed
 second derivative contribution  in the Taylor series  is thus  zero since either $r$
 or $z$  is zero. The remaining derivatives
 in the Taylor series expansion (all evaluated at grid point l) are  the (4) unknowns. Hence we need
 to solve the 4 by 4 system

  \beq
   -\pdiff{\Psi}{R} \lambda_{ll} \Delta R +\frac{1}{2}\pdiffn{\Psi}{R}{2}(\lambda_{ll}
   \Delta R)^2 = \Psi_{ll} - \Psi_l \\
   \pdiff{\Psi}{Z} \lambda_{lt} \Delta Z +\frac{1}{2}\pdiffn{\Psi}{Z}{2}(\lambda_{lt}
   \Delta Z)^2 = \Psi_{lt} - \Psi_l \\
   \pdiff{\Psi}{R} \lambda_{lr} \Delta R +\frac{1}{2}\pdiffn{\Psi}{R}{2}(\lambda_{lr}
   \Delta R)^2 = \Psi_{lr} - \Psi_l \\
  -\pdiff{\Psi}{Z} \lambda_{lb} \Delta Z +\frac{1}{2}\pdiffn{\Psi}{Z}{2}(\lambda_{lb}
   \Delta Z)^2 = \Psi_{lb} - \Psi_l \\
  \eeq

  for the derivatives ${\pdiff{\Psi}{R},\pdiff{\Psi}{Z},\pdiffn{\Psi}{R}{2},\pdiffn{\Psi}{Z}{2}} $
  for each grid point l that is in the vicinity of the plasma
  boundary. For points where all $\lambda = 1$ this is not
  necessary since the standard diamond difference scheme then
 supplies the values for the first and second derivatives.
 The required solution is
\beq
  \pdiff{\Psi}{R}\pmb{\bigg \vert_l}&  = &
    -\frac{-\Psi_{lr}\lambda_{ll}^2 + \Psi_l\left ( \lambda_{ll}^2
        -\lambda_{lr}^2\right
      )+\lambda_{lr}^2\Psi_{ll}}{\lambda_R\Delta R} \\
  \pdiff{\Psi}{Z}\pmb{\bigg \vert_l} &  = &
    \frac{\Psi_{lt}\lambda_{lb}^2 + \Psi_l\left ( \lambda_{lt}^2
        -\lambda_{lb}^2\right )
      -\lambda_{lt}^2\Psi_{lb}}{\lambda_Z \Delta z}\\
  \pdiffn{\Psi}{Z}{2}\pmb{\bigg \vert_l}&  = & -2
  \frac{-\Psi_{lt}\lambda_{lb} + \Psi_l\left (\lambda_{lt}+\lambda{lb}
    \right ) -\lambda{lt}\Psi_{lb}}{\lambda_Z (\Delta Z)^2} \\
  \pdiffn{\Psi}{R}{2}\pmb{\bigg \vert_l}&  = & -2
  \frac{-\Psi_{lr}\lambda_{ll} + \Psi_l\left (\lambda_{ll}+\lambda{lr}
    \right ) -\lambda{lr}\Psi_{ll}}{\lambda_R (\Delta R)^2} \\
  \lambda_R &  = & \lambda_{ll}\lambda{_lr}(\lambda_{ll}+\lambda{lr}) \\
  \lambda_Z &  = & \lambda_{lb}\lambda{_lt}(\lambda_{lb}+\lambda{lt})
 \eeq

 Note that these derivatives reduce to the standard, second order,
 finite diamond difference scheme when  all the $ \lambda $ are equal
 to 1.  Hence we have a simple method for incorporating the boundary 
 but retaining the cylindrical $(R,Z)$ grid. 



One additional complication is that the points inside the
plasma are renumbered so that we generate equations only for those points that we
actually will  find a value of psi for. This array, called map(m) ,m
=1,..nw*nh, in
\ot is defined so that map(k) =j means that the grid point numbered k
in the rectangular computational domain is actuall grid point number j
when we solve the discretized GS equation.
 The solution of the resulting set of equations can be done most
 easily with an iterative technique such as SOR.

Since most codes that \ot uses (eq Freya,Toray,Curray,etc.)  routinely require the solution $\Psi $
in the entire rectangular domain we also need to find the solution
outside the plasma. Thus in the region outside the plasma, but inside
the rectangular domain, we solve the eqaution ${\bigtriangledown}^{*} \Psi =0.0 $
with the known value of Psi on the plasma boundary and some known
values of $\Psi $ on the rectangular $(R,Z)$ boundary. Typically these
later boundary points are assumed to be time independent and given by
the soltuion of the complete free boundary value problem (as
determined by Efit for example).


 \begin{figure}[hbt] %note: figure environment not available in slides
 \centering 
\begin{narrow}{-.50in}{0in}   
 \mbox{\epsfig{file=resid_example.eps,height=3in,width=6in}}
\\[20pt]
 \mbox{\epsfig{file=resid_example.eps,height=3in,width=6in}}
\end{narrow}
 \caption{(a,b): Residuals} 
  \label{f1}
 \end{figure}

 The user can verify that the discretized GS eqaution now takes the
 form
\beq
\Psi_{ll}\alpha_{ll} +\Psi_l\alpha_l +\Psi_{lr} \alpha_{lr} +
\Psi_{lt} \alpha_{lt} + \Psi_{lb}\alpha_{lb} = S_l(\delta R)^2 \\
S_l = -\mu_0R^2\pdiff{P}{\Psi}\pmb{\bigg \vert_l} -f\pdiff{f}{\Psi}\pmb{\bigg \vert_l} \\
\alpha_{ll} =2\frac{\lambda_{lr}}{\lambda_r} + \Delta R
\frac{\lambda_{lr}^2}{\lambda_r R_l}\\
\alpha_{lr} = 2\frac{\lambda_{ll}}{\lambda_r} - \Delta R
\frac{\lambda_{ll}^2}{\lambda_r R_l}\\
\alpha_{lt} = 2\frac{\lambda_{lb}}{\lambda_z}\left ({\frac\Delta R }{} \right)^2 \\
\alpha_{lb} = 2\frac{\lambda_{lt}}{\lambda_z}\left ({\frac\Delta R }{} \right)^2 \\
\alpha_l = -2\frac{\lambda_{ll}+\lambda_{lr}}{\lambda_r} + \Delta R
\frac{\lambda_{ll}^2-\lambda_{lr}^2}{\lambda_r R_l} 
 -2\frac{\lambda_{lt}+\lambda_{lb}}{\lambda_Z}\left( \frac{\Delta
     R}{\Delta Z} \right)^2 
\eeq

The non linearity in the source term,$S_l$ can be handled using Picard
(also called successive subsitution) iteration and tthis option exists
in \ot. However it is possible o apply a globally convergent Newton
based  approach to these equations as follows.