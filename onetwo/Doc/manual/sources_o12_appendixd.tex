\normalsize
\section{Generating Profiles Consistent with MHD Equilibrium
Pressure Profiles}
Quite often the situation arises where we have an equilibrium that we would like
to use together with profiles of densities and tempertures that are scaled to
yield satisfactory performance. For example, scaling up of DIII-D AT scenarios
to ITER discharges presents us with this situation.

There are some basic tools available to accomplish this task. Depending on what
one chooses as known profiles there are  some variations in the method. The
basic idea is to consider the equations for charge neutrality, zeff, and eqdsk
pressure together with sufficient auxiliary information to allow a unique
determination of profiles that are then self- consistent. The basic set of
equations we work with are 
%\begin{gather}
\begin{align} \label {zeff-ref}
 n_{e} -Z_{p1}n_{p1}  -Z_{p2}n_{p2}  -Z_{imp1}n_{imp1}
 -Z_{imp2}n_{imp2} &= Z_bn_b +Z_{\alpha}n_{\alpha}\\
 Z_{eff} n_{e} -\left<Z_{p1}^2\right>n_{p1}-\left<Z_{p2}^2\right>n_{p2} \notag
 \\ 
 -\left<Z_{imp1}^2\right>n_{imp1}-\left<Z_{imp2}^2\right>n_{imp2} &=
 Z_b^2n_b +Z_{\alpha}^2n_{\alpha} \\
 n_{e}C_eT_e +n_{p1}C_iT_i+ n_{p2}C_iT_i \notag \\
 +n_{imp1}C_iT_i
 +n_{imp2}C_iT_i &= P - \frac{2}{3}(w_{beam}+w_{\alpha})\\
 Z_{frac}n_{p1} - n_{p2} &= 0 \\
 Z_{impfrac}n_{p1} - n_{imp2} &=0 
\end{align}
Here we assume two primary (i.e. hydrogenic) ions and two impurities with the
last two equations specifying the amounts of the second species in each case. We
assume that the beam and $\alpha$ densitites and stored energy densitites are
given. This implies that an iterative process is required to solve the linear
set of equations since beam deposition and fusion rates depend on the unknown
densitites and temperatures. $P$ is the known pressure profile from the
equilibrium calculations. Typically we assume that the electron density profile
is known, which eliminates the first of the equations \Eqref{zeff-ref}. The
above set of eqautions applies at each value of the minor radius grid $\rho $. 
The parameters multiplying the electron and ion temperatures, $C_e$, $C_i$ are
set to unity if \Eqref{zeff-ref} is included or they are automatically adjusted
at  each radius in such a way that the error in \Eqref{zeff-ref} is minimized if
that equation is not  included. Typically we want to fix the electron density at
a specified value so the later case is generally the one that arises.  The
general prodeedure is a follows:
\begin{enumerate}
 \item Start with an appropriate inone and eqdsk file. The inone file will
 contain densitites and temperatures that are not consistent with the eqdsk
 pressure profile. Run onetwo (snapshot mode perhaps) to generate the qikone 
 file.
 \item Use the (Python) script \emph{gnuplot\_qikone.py} to read the  relevant
 information out of qikone. \emph{gnuplot\_qikone.py} also genrates a page of
 graphs if possible and most importantly, the file fixbd.in is written. Fixbd.in
 contains the informaion required by the fixbdry code so that the above set of
 equations can be solved.
 \item Edit fixbd.in to fill in the fields that are marked with *****. These are
 user prefrences and determine, for example, if 4 or 5 equations are solved.
 \item Run the fixbdry code: \\
 \texttt{fixbdry129x129  fixbd.in} \\
 The fixbdry code generates an output file profiles.dat which are suitable for
 inclusion in the inone file. Python program \emph{gnuplot\_profiles.py} also
 reads this file and generates some graphics output.
 \item After pasting the appropriate inforamtion from profiles.dat into the
 inone file you return to step 1 for the next iteration. Typically 2-3
 iterations are sufficient to stabilize the solution.
\end{enumerate}
The graphical  output from \emph{gnuplot\_qikone.py}  and
\emph{gnuplot\_profiles.py} for a typical case is shown in Fig.~\ref{profiles}.
\begin{figure}
 \includegraphics[angle=-90,width=\textwidth]{zeff_fig1.eps}
 \caption{Output from one of the .py scripts.\label{profiles}}
\end{figure}
