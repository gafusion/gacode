   \documentclass[12pt]{article}
   \pagestyle{myheadings}\markboth{ /u/stjohn/tex/o12ref.dir/fusion_rate.tex}{ /u/stjohn/tex/o12ref.dir}
   %% LaTeX2e file `abbrev.tex'
%% generated by the `filecontents' environment
%% from source `sources_o12' on 2010/05/19.
%%

   \newcommand{\vp}{ V^{\prime}}
   \newcommand{\ot}{Onetwo\xspace}
   \newcommand{\ct}{Curray\xspace}
   \newcommand{\pdiff}[2]{\frac{\partial{ #1}}{\partial{ #2}}}
   %useage \pdiff{#1}{#2}
   \newcommand{\ddiff}[1]{\frac{\partial}{\partial #1}}
   \newcommand{\strt}{\begin{eqnarray}}
   \newcommand{\trts}{\end{eqnarray}}
   \newcommand{\beq}{\begin{eqnarray}}
   \newcommand{\eeq}{\end{eqnarray}}
   \newcommand{\pdiffn}[3]{\frac{\partial^{#3}{#1}}{\partial{ #2}^{#3}}}
   \newcommand{\pdiffz}[3]{\frac{\partial{#1}}{\partial{ #2}}\bigg \vert_{#3}}
    \newcommand{\myint}{\int\limits_{\rho_{j-\frac{1}{2}}}^{\rho_{j+\frac{1}{2}}
}d\rho H \rho}
    \newcommand{\myind}{\int\limits_{\rho_{j-\frac{1}{2}}}^{\rho_{j+\frac{1}{2}}
}d\rho}
    \newcommand{\onehalf}{\frac{1}{2}}
    %\newcommand{\eqref}[1]{(\ref{#1})}
    \newcommand{\Eqref}[1]{Eq.~\eqref{#1}}
    \newcommand{\Eqrefs}[2]{Eqs.~\eqref{#1}-\eqref{#2}}

 \textheight8.5in
\newenvironment{narrow}[2]{%
   \begin{list}{}{%
   \setlength{\topsep}{0pt}%
   \setlength{\leftmargin}{#1}%
   \setlength{\rightmargin}{#2}%
   \setlength{\listparindent}{\parindent}%
   \setlength{\itemindent}{\parindent}%
   \setlength{\parsep}{\parskip}}%
\item[]}{\end{list}}

 % can keep my definitions in file abbrev.tex
   \usepackage{epsfig}
   \usepackage{amsmath}
   \begin{document}


   \title{EVALUATION OF FUSION RATES IN ONETWO}
   \author{H.ST.JOHN}
   \date{\today}
   \maketitle
       \boldmath\	
      \section{Fast Evaluation of Integrals}
        \begin{description}
  	\item{sbcx $ [\frac{\#}{cm^3sec}]$} A source of fast ions
           (and also thermal neutrals with energy $\frac{3}{2}T_i $)
           due to charge exchange of beam  neutrals with thermal
           ions. sbcx is derived
	 either from hbirz (no prompt orbit averaging) or hdepz
	(with prompt orbit averageing) 
	\begin{eqnarray}
            sb_{cx}(\rho,i)=
        \end{eqnarray}
	\item{scx $[\frac{\#}{cm^3sec}]$} charge exchange between thermal
		ions and thermal neutrals. If two neutral species,corresponding
	to two primary ion species are present then scx will be a sink for
	one species and a source for the other. That is,ion species a charge
	exchanges with neutral species b producing a neutral of type a and
	an ion of type b. The charge exchange rate,$ cx12r(\rho) $ is based
	on the Freeman-Jones cross sections. For ion density $n_i $ and 
	neutral density $n_{nj} $ where $nj$ is the other species we have:
	\begin{eqnarray}
	     scx(\rho,i)= n_i(\rho)*n_{nj}(\rho)*cx12r(\rho)
	\end{eqnarray}
	If only one neutral species is present then there is no particle
	source, $ scx=0$ . (There is an energy source however because the ion
	and neutral temperatures are not assumed equal).

 	\end{description}

	\setlength{\unitlength}{1.0cm}
	\begin{picture}(16,16)
	\put(0,0){\line(1,0){16}}
	\put(16,0){\line(0,1){16}}
	\put(16,16){\line(-1,0){16}}
	\put(0,16){\line(0,-1){16}}

	\thicklines
	\put(1.5,1.2){\makebox(2,1)[s]{Test}}
	\put(8,4){\vector(1,0){2}} \put(1,0){$y$}
	\put(8,4){\vector(0,1){2}} \put(0,1){$z$}
	\put(8,4){\vector(-1,-1){1.5}}
	\put(8,4){\vector(1,2){1}}
	\put(8,4){\vector(2,-1){2}}
	\put(8,4){\vector(-1,4){0.5}}
	\end{picture}

	\footnote{/u/stjohn/tex/o12ref.dir}
    \end{document}

