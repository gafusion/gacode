


%\renewcommand{\textfraction}{0.10}
%\renewcommand{\topfraction}{0.40}  % not greater than 1- textfraction
%\renewcommand{\bottomfraction}{0.65} % not greater than 1- textfraction
%\renewcommand{\floatpagefraction}{0.10}

\section{Running  Nubeam in Onetwo} \label{intro}

 N.B. : This appendix was part of a submission to the NTCC.
 Consequently some observations/remarks/details are not intended for
 the users of the module, rather they were  directed at developers.


 Funding provided by the National Transport Code Collaboratory 
(NTCC) enabled us to install and test  the Monte Carlo fast ion 
 physics package, Nubeam \cite{ref1}, at General Atomics. The project consisted
 of building an interface to the Nubeam  code in \ot and subsequently
 testing the Nubeam  code with DIII-D and ITER discharges. We found
 that the Nubeam code gave results for all physical parameters tested
 that were supported by more primitive analytic fast ion slowing down
 calculations. Please note
 that it was not our job to actually perform a formal review of the
 Nubeam code. Consequently a reviewers sheet is not attached to this
 report. 

  The most recent version of Nubeam uses C++ features that are not
  currently supported by our local computational facilities and 
  consequently we  are forced to remain with the older version of
  the modules dated Sept. 2003. We note that locally encountered 
  difficulties in building the Nubeam code have generally been caused
  by the coupling of the C++ code with Fortran and we consequently urge the
  developers of the  code to rewrite the Preact module in Fortran.
  Even the older (Sept. 2003)  version of the code will not build on
  our local HP architecture which further restricts our use of this
  module. At this time we satisfactorily run Onetwo coupled with
  Nubeam only on Linux based X86 machines. 

   Our original attempt at interfacing the Nubeam module with Onetwo 
   by linking  the Nubeam libraries directly with Onetwo indicated
   that, at least in our implementation, a memory leak of unknown origin
   was responsible for our inability to get successful runs. This
   difficulty was ultimately resolved by changing the interface to
   externally call Nubeam at desired times and shutting the Nubeam
   program down in between these calls. This latter method also gave
   us the ability, through judicious use of save files, to run Nubeam
   in a stand alone fashion on any subset of \ot generated input
   files.  

 \subsection{Onetwo Interface}\label{interface}
   The interface to \ot  is described in some detail in this
   section to act as  a guide for other institutions
   contemplating a similar project as well as to provide written
   documentation of the use of the module in \ot. As remarked above we now run the
   \ot-Nubeam combination separately, which requires that \ot
   generate suitable input files for Nubeam any time a call to the
   fast ion physics package is made. Since equilibrium evolution
   is also part of the general transport scheme, appropriate time
   dependent equilibrium information must also be 
   available. Finally, since the Nubeam package solves the
   instantaneous  deposition as
   well as fast ion slowing down problem (including fusion originated
   alpha particles), some form of time step control for Nubeam calls
   must be provided. The time stepping itself must be aware of
   possibly arbitrarily complicated beam pulse waveforms [see
   Fig.~\ref{Fig5}(a) for example].

   The Nubeam driver called by \ot is a modified version of the
   recent nbdrive program  made available by PPPL. (I have not
   yet produced a patch file that will generate these changes automatically.)

   Interfacing to Nubeam is complicated by the fact that as many as
   three separate files must be read in order to collect the
   information required to setup a suitable Onetwo/Nubeam coupled run.
   The first  file is the standard \ot input file called inone. For
   \ot runs that do not use Nubeam the inone file contains all of the
   necessary beam information. That case is not considered further in
   this paper.  If a Nubeam type Onetwo run is selected
   in inone by an appropriate choice of input parameters, then
   additional information taken from one or two other input  files is required.
   The arbitrary names of these additional input files, beam\_data\_namelist and
   beam\_data\_ufile, are specified in inone. The file pointed to by
   beam\_data\_ufile is optional (currently not used)  and is intended to provide an
   automatic, MDS+ driven interface to the beam pulse
   data. This file is  part of the
   interface work still required to couple Onetwo to Nubeam when time
   dependent waveforms are to be used.
 The last file, pointed to
   by beam\_data\_namelist, is the primary Nubeam input file and is
   required whenever Nubeam is run. This is the file traditionally 
   written by nblist.for (a Transp routine  that creates the
   nbdrive\_naml namelist).  It has in it the beam 
   geometry and the order of the beam data in the ufile, as well as
   the the beam on and off times, etc.  Onetwo will attempt to read
   this file as a standard namelist input file (which means generally
   that the file must first be edited by hand because nblist does not
   properly terminate namelists ??).

   The actual  driver code  called by  \ot when  Nubeam is started  is
   called Nubeam\_driver. (This is also the name of the code to execute
   if Nubeam is run in stand alone mode.)
   When Nubeam\_driver   is started it expects to read a
   single file that contains all necessary information to run Nubeam.
   Nubeam\_driver is set up to take the first part of this name from
   the  command line used to start it. Thus for example the Nubeam
   package is started by executing the line 
   \begin{center}
   \texttt{nubeam\_driver nubeam\_12}  
   \end{center}
   where the complete input file name is actually
   nubeam\_12\_namelist.dat.
   Onetwo dynamically constructs a command line which includes the
   name of the input file that Nubeam\_driver will read.
   nubeam\_12\_namelist.dat is created  by the \ot code before
   Nubeam\_driver  is called. Once this file exists Nubeam can be
   run  in stand alone mode if desired. But the normal
   usage is to have \ot repeatedly run Nubeam  through Nubeam\_driver
   as  a sub process
   with automatically generated nubeam\_12\_namelist.dat  files that
   reflect the plasma state at the current time. 



   Nubeam\_driver   creates output files based on the first part of the
   input file name given on the  command line. Thus for example
   the files, nubeam\_12\_details.dat and nubeam\_12\_summary.dat
   are created. The file nubeam\_12\_summary.dat is intended for
   visual perusal and is not used  further  by \ot. The other file,
   nubeam\_12\_details.dat, contains all of the  fast ion deposition and
   slowing down data that \ot will process. This file is parsed by \ot
   but I  do not recommend that approach (instead the rather lengthy
   output should be reformatted explicitly for machine readability
   and an appropriate read routine should be created - I have not done
   this at this time). Nubeam\_driver
   will also write two netcdf formated restart
   files, nubeam\_12\_xplasma\_state.cdf and
   nubeam\_12\_nubeam\_state.cdf. These files are read by Nubeam on the
   next call to Nubeam and serve as initial conditions for the next
   time step. \ot does not read these netcdf files at all. Obviously on
   the first call to Nubeam these restart files will not exist. This
   forces us into the situation that we must start the analysis before
   the beams (or other fast ions ) exist. However once restart files exist
   they will be used to restart the analysis at suitable times so this
   restriction is not very severe. A complication is the fact that \ot
   profiles are also changing as a function of time and hence another
   file, nubeam\_12\_restart\_profs.txt, is required to complete a
   restart problem specification. All of these files are overwritten
   each time Nubeam is executed! The user must make sure that old
   *.cdf files are removed form the working directory before starting
   a new case. Otherwise Nubeam will pick up those old files, most
   likely resulting in unwanted results.  Optionally the
    restart files for Nubeam can be saved at a particular time
   using the inone parameter wrt\_restart\_file\_time.
      
   Since Nubeam does both the  instantaneous deposition of neutral beam
   ions as well as the slowing  down  of fusion products and beam ions,
   a finite time step, $\Delta t_{nb}$ (greater than about $10^{-5} sec $) must be
   supplied to the Nubeam code. The fast ion distribution related
   quantities are then evolved from $t$ to $ t+\Delta t_{nb}$.
   At the final time, $t+\Delta t_{nb}$, the physical quantities are
   dumped to file nubeam\_12\_details.dat. But Nubeam was called
   with profiles that exist in the transport code at time $t$.
   Since Nubeam does not evolve (thermal) transport related quantities such as the
   thermal electron and ion  densities and temperatures these and
   other profiles  are held constant  during the Nubeam time step $\Delta
   t_{nb}$. Thus there is an implied iteration necessary to fully bring
   the fast and thermal distributions into agreement. Given the current
   processing power available to us, such iteration is however not
   feasible and hence no  attempt to accommodate it currently exists in
   \ot. The minimum time step allowed by Nubeam ($ \approx
   10^{-5} sec$) is much smaller than what is practically possible.
   Although stiff confinement models may force \ot to take sub millisecond
   time steps in the thermal transport modeling, typically we use 5 to
   20 percent of the fast ion lifetime for $\Delta t_{nb}$.
   To accommodate this disparity in time steps linear interpolation is
   used. That is, once Nubeam returns with the fast ion related
   quantities at time  $t+\Delta t_{nb}$ we return to the thermal
   transport model in \ot at time $t$ and evolve the thermal
   quantities from time $t$ to time   $t+\Delta t_{nb}$ using linear
   interpolation of the nubeam quantities which are now known both at
   the beginning and at the end of the time interval $\Delta t_{nb}$.
   I believe  that such a non-iterative approach is benign in this
   application, but no attempt to justify it has been made.


   \subsection{Using Nubeam with Onetwo in restart mode}
   As explained above, the \ot/Nubeam combination must be 
   initially set up to run
   from a start time where none of the beams are  on. Once such a run
   exists however it is possible to use the files generated 
   by Nubeam and \ot as restart files.
    The three required restart files provide
   fast ion information such as beam density, slowing rates, etc., at
   the time the restart files were  written. The restart files supplement
   the information in the file nubeam\_data\_namelist and all four files
   are required in a restart run.
    The restart information can be used
   as initial conditions for a new run, eliminating the requirement
   that the beam power is zero at the start time of the Onetwo run.
   It is the users responsibility to make sure that the information in
   the restart files is compatible with the inone file that will be
   used to run \ot. The restart file, nubeam\_12\_restart\_profs.txt
   (or similar name---see above), has the time  of creation of the
   file in it on the first line, \texttt{nubeam\_restart\_time}. The second line contains the number
   of beams that are on at this time, \texttt{nbeams}, the following 4*nbeams
   lines contain the beam power, energy, full and half energy
   fractions for each beamline. Recall that \texttt{time0} is the time given in
   inone at which the transport simulation is run. Suppose that we set
   things up so that \texttt{time0} $<$ \texttt{nubeam\_restart\_time}. Then as \ot
   steps forward in time we  eventually encounter the
   \texttt{nubeam\_restart\_time} and there  would be a sudden jump in such
   quantities as the fast ion density when the information in the
   restart file was added to the current information in \ot.
   This is non physical and hence we do not allow 
   \texttt{nubeam\_restart\_time} $>$ \texttt{time0} as input (the code will check for this
   and exit if it is found). Hence we must have \texttt{time0} $\ge $
   \texttt{numbeam\_restart\_time}. 
   The code applies the initial conditions present in the  
   nubeam\_12\_restart\_profs.txt at \texttt{time0}, even if the
   \texttt{nubeam\_restart\_time} in that file is less than \texttt{time0}!
   The times at which the individual beams are turned on and off
   (\texttt{tbona} and \texttt{tboffa} in file beam\_data\_namelist) must be
   consistent with this (as long as the user does not modify
    nubeam\_12\_namelist.dat from what it was when the restart files
    were created this will be the case). The problem that can arise
    here is that an individual beam, say beam number 2, is off at
    time \texttt{nubeam\_restart\_time} but the user sets \texttt{time0} in inone such
    that beam 2 is on at \texttt{time0}.  This is an inconsistency  since the
    restart file does not contain any information about beam 2 but the
    value of \texttt{time0} in inone assumes that beam 2 is on at the beginning.
    If it is found that at \texttt{time0} beam 2 is on but beam 2 is off
    at time  \texttt{nubeam\_restart\_time} in file nubeam\_12\_restart\_profs.txt
    the code will exit.

    An obvious application of the restart method is to supply a
    consistent  electric field  at \texttt{time0} when beam driven current 
   is present. By setting up an initial run which assumes that the
   beam is turned on 3 slowing down times before \texttt{time0} and running
   from that time up to \texttt{time0} one can get an equilibrated beam at \texttt{time0}.



   \subsection{File Usage  Summary}
  As is evident from the discussion above the file structure
  associated  with running Nubeam is
  somewhat complex. A new set of these files is created each time
  Nubeam  is called. Normally the previous versions of these file is
  simply overwritten by the new  ones at the  current time. 
  The following summary  should help in keeping things in
  perspective:
  \begin{description}
    \item[inone] The primary \ot input file. The Nubeam related input
      in this file is all given in the second namelist. It is highly
      recommended (to avoid confusion) that only the following beam
      related quantities are specified in inone when the Nubeam option
      is selected:
      \begin{description}
       \item[use\_nubeam]  Logical variable. If true indicates that
         Nubeam is to be used. Default is false.
        \item[beam\_data\_ufile]  Name of the ufile to be used to get
          some beam input quantities. Not well defined at this time
          (and hence not used).
        
        \item[nubeam\_restart] Integer, \\
        =1  use existing restart files on first time step,\\
                                    =0  restart files don't exist,
                                    code will create them. \\
         If \texttt{nubeam\_restart} = 1 supply the following: 
        \begin{description}
          \item[nubeam\_state\_path]  Fully qualified name of Nubeam state
           restart file.
          \item[nubeam\_xplasma\_path] Fully qualified name of xplasma
           state restart file .
          \item[profile restart file] This item is NOT input in
            inone. It is here as a reminder that the file
            ***\_restart\_profs.txt is also  required. Here *** is the
            value of nubeam\_state\_path. In other words the profile
            restart file is assumed to reside in the same place as the
            state restart file.
      \end{description}
      \item[wrt\_restart\_file\_time ] A single time at which the restart
        files are to be saved. If left blank then only the last set of
        restart files at the final time will be available.


       \item[save\_nubeam\_input] Integer, defaulted to 0.      
                   Set to 1 to save all input files to nubeam (one such
                   file is created each time nubeam is called).
                   Note that Nubeam must be called on a regular basis 
                   even after the beam  is off
                   since fast ion slowing down is part of the 
                   Nubeam calculations. Hence use this option cautiously.


            \item[beam\_data\_namelist]  Character variable.
                     Default is \\
                     \textsf{'}nubeam2\_namelist.dat\textsf{'}.  It is the name
                     of the beam input data file used by Nubeam.
                     If \texttt{use\_nubeam=.false.} and \texttt{beam\_data\_namelist}
                     points to a valid file, that file
                     will be used to generate beam input for the
                     standard Onetwo Nfreya package.
      \end{description}
   \end{description}


  
   A detailed input description to run Nubeam as a sub process
   of \ot is given in file cray102.f.



   \subsection{Nubeam - \ot Comparison}\label{compare} 
    In Onetwo the initial  fast ion deposition is also done using 
    Monte Carlo methods. This aspect of the problem  is thus expected
    to be identical in the two codes and it is only the subsequent
    orbiting (or lack thereof) during the fast ions lifetime that
    drives differences between the codes.

    In Fig.~\ref{Fig1}, we show a simple time dependent example of the 
    type normally treated by \ot. 
    
 \begin{figure} %note: figure environment not available in slides
 \centering 
 \begin{narrow}{-.50in}{0in}   
   \mbox{ \rotatebox{90.} {\epsfig{file=beamon.eps,height=3in,width=3in}}}
   \mbox{ \rotatebox{90.}{\epsfig{file=totfi.eps,height=3in,width=2.9in}}}
\end{narrow}
 \caption{(a) A simple rectangular beam pulse used to drive Nubeam
   and the \ot fast ion calculations. (b) The resulting stored fast
   ion beam density showing the asymptotic approach to steady state.}
  \label{Fig1}
 \end{figure}
    A single beam pulse is turned on at
    1800 msec and remains on for the duration of the 200 msec transport
    time. The total number of fast ions in the plasma approaches a steady
    state value as indicated in part (b) of the figure. In this and
    all subsequent figures we show the analytic and Nfreya based
    results generated by \ot in red and the Nubeam results for the
    identical case in blue. For Nubeam two curves are shown, the
    choppier, dashed curve corresponds to using 1000 ``ions'' in
    Nubeam. The smoother, solid, blue curves refer to Nubeam results
    using 10000 Monte Carlo particles. As indicated in
    Fig.~\ref{Fig1}(b) the assumed analytic slowing down distribution
    in \ot leads to a larger number of fast ions, due in part to
    fewer charge exchange losses.  
    The actual equilibrium fast ion density is shown Fig.~\ref{Fig2}(a),
    where we see that the higher, co-injected, fast ion density in \ot is on the
    outer half of the plasma. 
 \begin{figure} %note: figure environment not available in slides
 \centering 
 \begin{narrow}{-.50in}{0in}   
   \mbox{
     \rotatebox{90.}{\epsfig{file=beamben.eps,height=3in,width=3.0in}}}
   \mbox{ \rotatebox{90.} {\epsfig{file=neutron2.eps,height=3in,width=3in}}}
\end{narrow}
 \caption{(a) The fast ion density at 2000 msec across the plasma. (b) The neutron rates
   as a function of time}
  \label{Fig2}
 \end{figure} 
    \ot uses a prompt orbit model, based on
    conservation of canonical angular momentum, to spread the fast
    ions out over an orbit width at the time of birth. Obviously such
    a prompt model can only approximate the spatial diffusion of the
    fast ions during the slowing down process.



    In Fig.~\ref{Fig2}(b),  we show   the
    various combinations of fast and thermal $ D(D,n)He^3 $ neutron
    rates as a function of time. The thermal neutron rate determined
    by Nubeam is not distinguishable from  the rate determined by \ot
    in the figure. The
    slight drop in the thermal  rate as a function of time is due to the
    buildup of fast ions. Since zeff and the electron density are held
     constant during the simulation, the thermal ion density decreases
     slightly as the steady state fast ion density approaches equilibrium.
    However the beam-thermal rate is significantly less in \ot. The beam-thermal
    neutron rate in \ot is calculated by using a Maxwellian
    distribution
for the thermal ions and a classical slowing down distribution for the
fast ions. Consistent with this the beam-beam rate in \ot  has the
value of $2.71 \times 10^{13}/sec $ compared to $ 3.11 \times 10^{13}/sec$ for Nubeam.
    Even though the beam density is higher in \ot the beam associated
    neutron rates are less. Assuming that the Monte Carlo determined
    rates in Nubeam are more applicable this indicates that the assumed slowing
    down distribution in \ot can lead to beam related neutron
    production rates that are in error by $ \approx 30\%$.

   The preferential heating of the thermal ions after 200 msec of beam
   evolution is
   shown in Fig.~\ref{Fig3}. 
 \begin{figure} %note: figure environment not available in slides
 \centering 
 \begin{narrow}{-.50in}{0in}   
   \mbox{
     \rotatebox{90.}{\epsfig{file=qbeam.eps,height=3in,width=3.0in}}}
   \mbox{ \rotatebox{90.} {\epsfig{file=pwrintg.eps,height=3in,width=3in}}}
\end{narrow}
 \caption{(a)The electron and ion thermal heating profiles. (b) The integrated
   thermal heating profiles}
  \label{Fig3}
 \end{figure}   
   Here the the Nubeam determined spatial distribution of the
   neutral beam heating profile is somewhat noisy, even  when 10000
   Monte Carlo particles  are used in the simulation. Near the
   magnetic axis a simulation with 1000 particles can be quite far off due
   to the lack of sufficient sampling  in the small volume near the
   magnetic axis. Such a result would tend to cause havoc in a
   transport simulation (at the next time step the void area might be
   filled in, etc.), and hence we expect that significantly more than
   1000 ions should be used as a matter of routine. Overall tracking
   with the \ot results for both the spatial power density
   [Fig.~\ref{Fig3}(a)] and integrated power [Fig.~\ref{Fig3}(b)] is
   reasonable.

   Two other features which are of paramount importance to the
   analysis and simulation of thermal plasmas are the beam driven
   current and the beam supplied torque density that drives the
   toroidal rotation of the thermal ion species. These quantities are
   compared with \ot in Fig.~\ref{Fig4}(a,b), respectively. 
 \begin{figure} %note: figure environment not available in slides
 \centering 
 \begin{narrow}{-.50in}{0in}   
   \mbox{
     \rotatebox{90.}{\epsfig{file=bcur.eps,height=3in,width=3.0in}}}
   \mbox{  {\epsfig{file=torque.eps,height=3in,width=3in}}}
\end{narrow}
 \caption{(a) The beam driven current. (b) The torque density acting to
   spin up the plasma.}
  \label{Fig4}
 \end{figure}
   For \ot the
   total (shielded) beam driven current is 75 kA, while for Nubeam 
   the number is about 90 kA. The current profiles track reasonably
   well and the difference can be ascribed to the analytic formulation
   of the beam driven current used in in \ot (see Ref. \cite{ref2}) versus the Monte Carlo
   simulation of the collision operators used in Nubeam (see Ref. \cite{ref1}).

   The torque density from Nubeam, Fig.~\ref{Fig4}(b), is quite noisy
   even with 10000 ions (the 1000 ion result is not  shown because it is
   too noisy). Because the torque density enters only as a source term
   in the toroidal momentum equation, the effect of this noise is
   probably not severe. However we should keep in mind that the
   actual torque density used at any given time and at any given
   radial grid point will be a linear combination of two profiles of
   the  type shown, separated in time by $\delta_{nb}$ as explained
   above. Since this time interval can be large we might expect some
   discrepancies due to this effect, particularly with stiff
   confinement models that depend on the rotational shear.

   We do not explicitly discuss the final set of
   figures, Figs.~\ref{Fig5} and \ref{Fig6}. They are included here to
   demonstrate that arbitrarily complex beam pulse waveforms can be
   handled by Nubeam. 
 \begin{figure} %note: figure environment not available in slides
 \centering 
 \begin{narrow}{-.50in}{0in}   
   \mbox{
     \rotatebox{90.}{\epsfig{file=bpow.eps,height=3in,width=3.0in}}}
   \mbox{ \rotatebox{90.} {\epsfig{file=fastionstot.eps,height=3in,width=3in}}}
\end{narrow}
 \caption{(a) A complicated series of beam pulses. (b) The resulting
   total number of fast ions in the plasma.}
  \label{Fig5}
 \end{figure}
  %\FloatBarrier  %this command prevents part of teh bibliography from 
                 %appearing mixed in with the graphs.
 \begin{figure} %note: figure environment not available in slides
 \centering 
 \begin{narrow}{-.50in}{0in}   
   \mbox{
     \rotatebox{90.}{\epsfig{file=totneut.eps,height=3in,width=3.0in}}}
   \mbox{\rotatebox{90.}  {\epsfig{file=bcur2.eps,height=3in,width=3in}}}
\end{narrow}
 \caption{(a) The Nubeam calculated neutron rates,  and (b) the Nubeam
   determined beam driven current due to the pulsed power shown in Fig.~\ref{Fig5}(a). }
  \label{Fig6}
 \end{figure}

