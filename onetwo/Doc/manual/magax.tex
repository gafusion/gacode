When the Newton method is used to resolve the nonlinearity in the GS
equationwe have the following situation.




The fiducial point $k_a \equiv (i_a_1)*nh +ja $ is found as the lower
left point of the rectangle { ka,ka+1,ka+1+nh,ka+nh} This rectangle is
is found by searching for two sucessive R grid points where
$\pdiff{\Psi}{R} $ changes sign and two succesive z points where
$\pdiff{\Psi}{Z} $ changes sign. These derivatives are given by
Eq[\ref{??}] and hence, in terms of $\Psi$ the twelve grid points
{ka-nh,....ka+1+2*nh} become involved. Given this information  it is easy to find the line
$\overbar{de}$  which represents the locus of point on which
$\pdiff{\Psi}{Z} =0 $ . Similarly the line \overbar{bc} represents
the set pf points where$\pdiff{\Psi}{R} =0 $.( Althouh shown as tilted
in the figure, mostlikely these lines will turn out to be vertcal and
horizontal for most tokamak equilibria). The intersection point a
represent the estimated location of the magnetic axis. Since $\Psi $
at the magnetic axis is thus shown to depend on the 12 given points it
is straightforward to incorporate that dependency into the Jacobian.
