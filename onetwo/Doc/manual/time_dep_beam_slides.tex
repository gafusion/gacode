

\begin{filecontents}{abbrev.tex}
   \newcommand{\clr} {\color{red}}
   \newcommand{\clb} {\color{blue}}
   \newcommand{\vp}{ V^{\prime}}
   \newcommand{\ot}{\emph{ONETWO }}
   \newcommand{\pdiff}[2]{\frac{\partial{ #1}}{\partial{ #2}}}
   %useage \pdiff{#1}{#2}
   \newcommand{\ddiff}[1]{\frac{\partial}{\partial #1}}
   \newcommand{\strt}{\begin{eqnarray}}
   \newcommand{\trts}{\end{eqnarray}}
   \newcommand{\beq}{\begin{eqnarray}}
   \newcommand{\eeq}{\end{eqnarray}}
   \newcommand{\pdiffn}[3]{\frac{\partial^{#3}{#1}}{\partial{ #2}^{#3}}}
   \newcommand{\pdiffz}[3]{\frac{\partial{#1}}{\partial{ #2}}\bigg \vert_{#3}}
   \textheight8.5in
   \endinput
\end{filecontents}
\documentclass{slides}  
  %note: the landscape option above simply tells tex to
  %  use the long dimension of the page in calculating its layout
  % (if this is all that is done the the lines would run off the paper
  % when printed)
  %also need to use landscape option in dvips, which actually rotates
  %everything so that it fits across the long side of the page.
  %(if landscape option is used in dvips but not in tex document
  % then the text will be too short for the long side of the page)
 \usepackage{amsmath} 
 %\usepackage{fancyheadings,graphics,color}
 \usepackage{fancyheadings,graphics}
 \usepackage[dvips]{color}
   %\usepackage{epsfig}
  %\usepackage[usenames]{color} seems logical but doesn't work
   %% LaTeX2e file `abbrev.tex'
%% generated by the `filecontents' environment
%% from source `sources_o12' on 2010/05/19.
%%

   \newcommand{\vp}{ V^{\prime}}
   \newcommand{\ot}{Onetwo\xspace}
   \newcommand{\ct}{Curray\xspace}
   \newcommand{\pdiff}[2]{\frac{\partial{ #1}}{\partial{ #2}}}
   %useage \pdiff{#1}{#2}
   \newcommand{\ddiff}[1]{\frac{\partial}{\partial #1}}
   \newcommand{\strt}{\begin{eqnarray}}
   \newcommand{\trts}{\end{eqnarray}}
   \newcommand{\beq}{\begin{eqnarray}}
   \newcommand{\eeq}{\end{eqnarray}}
   \newcommand{\pdiffn}[3]{\frac{\partial^{#3}{#1}}{\partial{ #2}^{#3}}}
   \newcommand{\pdiffz}[3]{\frac{\partial{#1}}{\partial{ #2}}\bigg \vert_{#3}}
    \newcommand{\myint}{\int\limits_{\rho_{j-\frac{1}{2}}}^{\rho_{j+\frac{1}{2}}
}d\rho H \rho}
    \newcommand{\myind}{\int\limits_{\rho_{j-\frac{1}{2}}}^{\rho_{j+\frac{1}{2}}
}d\rho}
    \newcommand{\onehalf}{\frac{1}{2}}
    %\newcommand{\eqref}[1]{(\ref{#1})}
    \newcommand{\Eqref}[1]{Eq.~\eqref{#1}}
    \newcommand{\Eqrefs}[2]{Eqs.~\eqref{#1}-\eqref{#2}}

 \textheight8.5in
\newenvironment{narrow}[2]{%
   \begin{list}{}{%
   \setlength{\topsep}{0pt}%
   \setlength{\leftmargin}{#1}%
   \setlength{\rightmargin}{#2}%
   \setlength{\listparindent}{\parindent}%
   \setlength{\itemindent}{\parindent}%
   \setlength{\parsep}{\parskip}}%
\item[]}{\end{list}}




   %\onlyslides{6}


%----------------------------------------------------------------------------
      \color{blue} %\pagestyle{fancy}
  \begin{document}        
       \begin{center}
           \Large\bfseries Wedn. Meeting\\ % 1
            Time Dependent Beam and GLF23 Modeling
           March 27,'02 \\
           H.S.J.

        \end{center}
    

%------------------------------------------------------------------
 
     \begin{slide}          \setlength{\topmargin}{-0.5in}
       \begin{center}
           \Large\bfseries The Fast Ion Slowing Down Problem 
        \end{center}
        \normalsize  

        \bigskip
        \hrule height4pt

         \begin{itemize} \bfseries \tiny
           \item
         The simplest form of fast ion slowing down on a 
         background plasma is given by the Fokker-Plank equation:
           \begin{eqnarray}
             \tau_s \pdiff{f_b}{t}(v,\zeta,t) = \frac{1}{v^2} 
             \ddiff{v}{(v^3+v_b^3)} f_b(v,\zeta,t) +\frac{Z_2v_c^3}{2 v^3}
             \ddiff{\zeta}{(1-\zeta^2)}  \pdiff{f_b}{\zeta}(v,\zeta,t)
            \nonumber \\  + \frac{\tau_s}{\tau_{cx}}f_b(v,\zeta,t) +
            \frac{\tau_s}{\tau_{fus}}f_b(v,\zeta,t) +
            \frac{\tau_s}{\tau_{terhm}}f_b(v,\zeta,t) +
            \tau_s S^i(v,\zeta,t)
            \label{eq:wav2}
           \end{eqnarray}
          \item
           We are interested in the solution of this equation when the 
           source term has a (repetitive) pulsed time
           dependance of the form
          \begin{equation} \label{eq:wav1}
          S^i(v,\zeta,t)= \frac{\dot S_0^i}{v^2}\delta(v-v_0) \delta(\zeta
          -\zeta_0) \biggl( H(t-t_0^i) - H(t-t_1^i) \biggr)
                    \end{equation}
           which is turned on and off at times  $t_0^i ,t_1^i $ (H is
           the Heavyside step function).
           \item 
           The resulting fast ion distribution function due to this
           source is
          \begin{eqnarray}  %use eeqnarray for multiple line formulas
           f_b^i(v,\zeta)= \frac{\dot S_0^i \tau_s}{v^3+v_c^3} P_{tot}(v)
               \sum_{l=0}^{\infty}\frac{2l+1}{2}P_l(\zeta)P_l(\zeta_b)
               \left[\frac{v^3}{v_b^3} \left(\frac{v_b^3+v_c^3}{v^3+v_c^3}
                \right)\right]^{\frac{1}{6}l(l+1)Z_2} \nonumber \\  
            \bigg[
                   H\Bigl(t-t_0^i - \tau_0(v_b)+\tau_0(v) \Bigr)
                   -H\Bigl(t-t_1^i - \tau_0(v_b)+\tau_0(v) \Bigr)
            \bigg]
              \label{eq:wav3a}
          \end{eqnarray}
            \item  $P_{tot}(v) $ is the total loss rate( charge
            exchange,fusion ,orbits,thermalization). We require
            linearity to proceed:
            \begin{equation}
               f_b^{tot}(v,\zeta) = \sum_{i}f_b^i(v,\zeta)
              \label{eq:wav4a}
            \end{equation} 
    \end{itemize}
    \end{slide}



     \begin{slide}          \setlength{\topmargin}{-0.5in}
       \begin{center}
           \Large\bfseries The Fast Ion Slowing Down Problem 
        \end{center}
        \normalsize  

        \bigskip
        \hrule height4pt

         \begin{itemize} \bfseries \tiny
           \item
            In order to use these results in a thermal transport code
            we form various moments of the distribution function with 
            the collision operators. For each individual pulse we
            require moments such as  \\
            The fast ion density :
         \begin{equation}
             n_b^i(t) \equiv \int_{v_{min}^i(t)}^{v_{max}^i(t)}
                       \int_{-1}^{1} 
                     f_b^i(v,\zeta)
             v^2dvd\zeta   
        \label{eq:wav5a}
        \end{equation}
           The fast ion stored energy density :
         \begin{equation}
             E_b^i(t) \equiv \frac{1}{2}m_b\int_{v_{min}^i(t)}^{v_{max}^i(t)}
                       \int_{-1}^{1} 
                      v^2 f_b^i(v,\zeta)
             v^2dvd\zeta   
        \label{eq:wav5b}
        \end{equation}
           The power desnity  delivered to the electrons:
         \begin{equation}
             Q_e^i(t) \equiv \frac{1}{2}m_b\int_{v_{min}^i(t)}^{v_{max}^i(t)}
                       \int_{-1}^{1} 
                   (\ddiff{v}{v^3 f_b^i(v,\zeta)})
             v^2dvd\zeta   
        \label{eq:wav5c}
        \end{equation}
           The energy delivered to the ions:
         \begin{equation}
             Q_i^i(t) 
                    \equiv \frac{1}{2}m_b \int_{v_{min}^i(t)}^{v_{max}^i(t)}
                       \int_{-1}^{1} 
                    (\ddiff{v}{v_c^3 f_b^i(v,\zeta)})
             v^2dvd\zeta   
        \label{eq:wav5d}
        \end{equation}
           The fast ion momentum transfer to electrons:
         \begin{equation}
             M_e \equiv \frac{m_b}{\tau_s}\int_{v_{min}^i(t)}^{v_{max}^i(t)}
                       \int_{-1}^{1} 
                    \frac{\zeta}{v}(\ddiff{v} f_b^i(v,\zeta))
             v^2dvd\zeta   
        \label{eq:wav5e}
        \end{equation}
    \end{itemize}
    \end{slide}





     \begin{slide}          \setlength{\topmargin}{-0.5in}
       \begin{center}
           \Large\bfseries The Fast Ion Slowing Down Problem 
        \end{center}
        \normalsize  

        \bigskip
        \hrule height4pt

         \begin{itemize} \bfseries \tiny
           \item
       In these equations  both the upper and lower limits of
       integration are functions of time due to the transient
       nature of the source:
       \begin{align}
          v_{max}(t) & = v_b^i &  t_0^i\le t \le t_1^i \\
                     & = [((v_b^i)^3
                     +v_c^3)\exp(\frac{-3(t-t_1^i)}{\tau_s})
                    -v_c^3]^{\frac{1}{3}} &t_1^i \le t \le
                    t_1^i+\tau_0(v_b) \\
                    & = v_t & t \ge t_1^i+\tau_0(v_b) \\
          v_{min}(t) & =  [((v_b^i)^3
                     +v_c^3)\exp(\frac{-3(t-t_0^i)}{\tau_s})
                     -v_c^3]^{\frac{1}{3}}  &t_0^i \le t \le
                    t_0^i+\tau_0(v_b) \\
                     & = v_t & t \ge t_0^i+\tau_0(v_b)
       \end{align}

       The upper limit
       of integration remains fixed at the fast ion birth speed
       $v_b^i$
       until the source $ \dot S_0^i $ is shut off.  Thereafter this
       upper limit decreases  since ions of speeds greater than $
       v_{max} $ are not replenished by the source. 
       The lower limit
       starts at the same value as the upper limit and decreases as
       time goes on until it reaches an arbitrarily defined thermal
       cutoff value $ v_t $.
       If the source is on long enough ($ t_1^i -t_0^i \ge
       \tau_0(v_b)$ ) the moments 
       become time independent when the lower limit of 
       integration  reaches the value $ v_t $ at time $t_0^i +\tau_0(v_b)$
       (typically $v_t$ is taken as 0.0, but in \ot it is user
       specifiable).     
    \end{itemize}
    \end{slide}


    \begin{slide}          \setlength{\topmargin}{-0.5 in}
       \begin{center}
           \Large\bfseries Example Interpretation of Results
        \bigskip
        \end{center}
        \hrule height4pt
         \begin{itemize} \bfseries \tiny
          \item The relationship between the beam slowing down time and
          the on/off switching frequency of the beam sources, coupled
          with the superposition of sources from different beams and
          energy components will produce complex
          waveforms:

       \rotatebox{0.}{\resizebox{10 cm}{12 cm}{
               \includegraphics{plot1a.eps}}}

        Fast ion density.
                 The square wave,composed pulses a,b,c,etc., are the
                 beam switching times inyes
                 arbitray units.
    \end{itemize}
    \end{slide} 
 
    \begin{slide}          \setlength{\topmargin}{0. in}
       \begin{center}
           \Large\bfseries Example Delayed Particle Source
        \bigskip
        \end{center}
        \hrule height4pt
         \begin{itemize} \bfseries \tiny
          \item

       \rotatebox{0.}{\resizebox{10 cm}{12 cm}{
               \includegraphics{thermal_source.eps}}}

         Thermal  particle source due to a
          single beam pulse that starts at 1 msec and ends at 41
          msec.The delayed sources are due to the different slowing
          down times of the three beam energy components.
    \end{itemize}
    \end{slide}


     \begin{slide}          \setlength{\topmargin}{-0.5in}
       \begin{center}
           \Large\bfseries Combined Analytic Monte Carlo Approach
        \end{center}
        \normalsize  

        \bigskip
        \hrule height4pt

         \begin{itemize} \bfseries \tiny
           \item Should use ADAS cross section 
           \item Common Pulsed wavetrain interface for
                 analytic and M.C. codes
           \item Will use PPPL M.C. Beam package. But intend
                 to modify for fast ion instabilities
                 to account for anomalous fast ion diffusion???
           \item Expect that most of the time we will use 
             analytic method for speed. This is workable in
             classes of discharges as was denonstrated by
             Azumi et. al. for JT60U.

    \end{itemize}
    \end{slide}


    \begin{slide}          \setlength{\topmargin}{0. in}
       \begin{center}
           \Large\bfseries GLF23 Status in Onetwo
        \bigskip
        \end{center}  
        \hrule height4pt
         \begin{itemize} \bfseries \tiny
          \item New renormalized mode introduced into Onetwo
            \item Sparse Jacobian with Broyden factorization
                  under development.
          \item Application to Iter-Feat standard and AT cases underway
          \item steady state and timie evolution (up to 10 sec) cases
            have been run

      \setlength{\leftmargin}{10. in}
       \rotatebox{90.}{\resizebox{10 cm}{15 cm}{
               \includegraphics{teti_etor_iter_feat.eps}}}

    \end{itemize}
    \end{slide}
\end{document}   
        \rotatebox{90.}{\resizebox{10 cm}{10 cm}{
               \includegraphics{etor_iter_feat.eps}}}
       \rotatebox{90.}{\resizebox{10 cm}{10 cm}{
               \includegraphics{teti_iter_feat.eps}}}