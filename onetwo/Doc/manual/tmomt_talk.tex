


\begin{filecontents}{abbrev.tex}
   \newcommand{\clr} {\color{red}}
   \newcommand{\clb} {\color{blue}}
   \newcommand{\vp}{ V^{\prime}}
   \newcommand{\ot}{\emph{ONETWO }}
   \newcommand{\pdiff}[2]{\frac{\partial{ #1}}{\partial{ #2}}}
   %useage \pdiff{#1}{#2}
   \newcommand{\ddiff}[1]{\frac{\partial}{\partial #1}}
   \newcommand{\strt}{\begin{eqnarray}}
   \newcommand{\trts}{\end{eqnarray}}
   \newcommand{\beq}{\begin{eqnarray}}
   \newcommand{\eeq}{\end{eqnarray}}
   \newcommand{\pdiffn}[3]{\frac{\partial^{#3}{#1}}{\partial{ #2}^{#3}}}
   \newcommand{\pdiffz}[3]{\frac{\partial{#1}}{\partial{ #2}}\bigg \vert_{#3}}
   \textheight8.5in
   \endinput
\end{filecontents}
\documentclass{slides}  
  %note: the landscape option above simply tells tex to
  %  use the long dimension of the page in calculating its layout
  % (if this is all that is done the the lines would run off the paper
  % when printed)
  %also need to use landscape option in dvips, which actually rotates
  %everything so that it fits across the long side of the page.
  %(if landscape option is used in dvips but not in tex document
  % then the text will be too short for the long side of the page)
 \usepackage{amsmath} 
 %\usepackage{fancyheadings,graphics,color}
 \usepackage{fancyheadings,graphics}
 \usepackage[dvips]{color}
   %\usepackage{epsfig}
  %\usepackage[usenames]{color} seems logical but doesn't work
   %% LaTeX2e file `abbrev.tex'
%% generated by the `filecontents' environment
%% from source `sources_o12' on 2010/05/19.
%%

   \newcommand{\vp}{ V^{\prime}}
   \newcommand{\ot}{Onetwo\xspace}
   \newcommand{\ct}{Curray\xspace}
   \newcommand{\pdiff}[2]{\frac{\partial{ #1}}{\partial{ #2}}}
   %useage \pdiff{#1}{#2}
   \newcommand{\ddiff}[1]{\frac{\partial}{\partial #1}}
   \newcommand{\strt}{\begin{eqnarray}}
   \newcommand{\trts}{\end{eqnarray}}
   \newcommand{\beq}{\begin{eqnarray}}
   \newcommand{\eeq}{\end{eqnarray}}
   \newcommand{\pdiffn}[3]{\frac{\partial^{#3}{#1}}{\partial{ #2}^{#3}}}
   \newcommand{\pdiffz}[3]{\frac{\partial{#1}}{\partial{ #2}}\bigg \vert_{#3}}
    \newcommand{\myint}{\int\limits_{\rho_{j-\frac{1}{2}}}^{\rho_{j+\frac{1}{2}}
}d\rho H \rho}
    \newcommand{\myind}{\int\limits_{\rho_{j-\frac{1}{2}}}^{\rho_{j+\frac{1}{2}}
}d\rho}
    \newcommand{\onehalf}{\frac{1}{2}}
    %\newcommand{\eqref}[1]{(\ref{#1})}
    \newcommand{\Eqref}[1]{Eq.~\eqref{#1}}
    \newcommand{\Eqrefs}[2]{Eqs.~\eqref{#1}-\eqref{#2}}

 \textheight8.5in
\newenvironment{narrow}[2]{%
   \begin{list}{}{%
   \setlength{\topsep}{0pt}%
   \setlength{\leftmargin}{#1}%
   \setlength{\rightmargin}{#2}%
   \setlength{\listparindent}{\parindent}%
   \setlength{\itemindent}{\parindent}%
   \setlength{\parsep}{\parskip}}%
\item[]}{\end{list}}




   %\onlyslides{5}


%----------------------------------------------------------------------------
      \color{blue} %\pagestyle{fancy}
  \begin{document}        
       \begin{center}
           \Large\bfseries Wedn. Meeting\\ % 1
           July 13,'00 \\
           H.S.J.

        \end{center}
    

%------------------------------------------------------------------
 
     \begin{slide}          \setlength{\topmargin}{-0.5in}
       \begin{center}
           \Large\bfseries Toroidal Momentum Equation % 1

        \end{center}
        \normalsize  

        \bigskip
        \hrule height4pt
        %\bigskip
         % \bf
         \begin{itemize} \bfseries \tiny
           \item
         The equation for toroidal momentum and rotation used in
        Onetwo assumes that all of the momentum and energy is carried by
        the ions. All ions have the same temperature and rotation speed,
        the associated momentum of each ion fluid depends on the mass
        of the ion however.
        The actual equation solved by Onetwo is
          \begin{multline}
             \sum_{i=1}^{nprim}m_i n_i<R^2>\pdiff{\omega}{t}
             +\omega\sum_{i=1}^{nprim}m_i<R^2>\pdiff{n_i}{t} \\
             +\frac{1}{ H \rho}\ddiff{\rho}\left(H\rho
                 \Gamma_\omega\right) 
           = S_\omega  +S_\omega ^{2D}  \label{eq:omega}
          \end{multline}
        The flux surface average toroidal angular momentum density
         for ion species i 
        is given by
        \beq
            m_in_i\omega<R^2>
         \eeq
        The total momentum flux($\frac{g}{sec^2}$) is made up of two parts as usual
          \beq
            \Gamma_\omega=\Gamma_\omega^{cond}+\Gamma_\omega^{conv}
                               \label{eq:omgam}
           \eeq
          The ``convective'' flux is
          \beq
             \Gamma_\omega^{conv}=\sum_{i=1}^{nprim}m_i<R^2>\omega\Gamma_i
          \eeq
          The ``conductive'' momentum flux is 
        \beq
             \Gamma_\omega^{cond} \equiv \Pi =\sum_{k=1} \Pi_k
               \label{eq:gwcond}
         \eeq
        With each $\Pi_k $ defined as
         \beq
                   \Pi_k=-d_k(\rho)\ddiff{\rho} u_k \label{eq:dcoef}
          \eeq
        At the present time \color{red} $d_k (\frac{g*cm}{sec}) $ is assumed to be \emph{diagonal} with a contribution
        only form $k=\omega $ \color{blue}
        Several simple models for $d_\omega $ exist in the code.

        All of the above equations can be run in analysis or simulation
        mode. Analysis mode means that the appropriate dependent variable
        is known a priori as a function of space (and possibly time). 
        Consequently the associated diffusion equation can be inverted
        to yield information on the transport coefficients. The success of 
        this inverse method depends on how well the sources are known.
        For 
        example if the
        momentum equation is run in analysis mode, Eq.[\ref{eq:omega}] is
        first solved for $\Gamma_\omega $ (assuming all other terms in
         the equation
        are known a priori):
          \begin{multline}
             \Gamma_\omega =\frac{1}{H\rho}\int_0^\rho H\rho\bigg(
                             S_\omega +S_\omega^{2D}
                         -\omega\sum_{i=1}^{nprim}m_i<R^2>\pdiff{n_i}{t} \\
             - \pdiff{\omega}{t}\sum_{i=1}^{nprim}m_i n_i<R^2> \bigg)d\rho
          \end{multline}
           and then   $\Gamma_\omega^{cond} $ is obtained from 
            Eq.[\ref{eq:omgam}] ( the convective part is known at this point
            from particle balance,Eq.[\ref{eq:ni}]). The diffusion 
            coefficient can then be obtained using Eq.[\ref{eq:dcoef}].

        The equations that are run in simulation must have 
        transport coefficients
        and appropriate initial and boundary conditions specified a priori.
        The profiles are then evolved from the initial condition profile as
        usual. 
    \end{itemize}
    \end{slide}

%-----------------------------------------------------------------------------
 
     \begin{slide}          \setlength{\topmargin}{-0.5in}
       \begin{center}
           \Large\bfseries Toroidal Momentum Sources % 1

        \end{center}
        \normalsize  

        \bigskip
        \hrule height4pt
        %\bigskip
         % \bf
         \begin{itemize} \bfseries \tiny
     \item
       All source terms below are in units of \color{red}  $
       [\frac{g}{cm \cdot sec^2}]$ \color{blue}
         \color{red} The electrons are assumed to have negligible
         momentum  \color{blue} (ie no separate
        equation for the electron toroidal momentum is introduced. However
        some momentum sources associated with the fast ion electron interactions
        are included below as ion terms). At the present time sources and sinks
associate with \emph{ion impact ionization} are neglected.
        \begin{description}
        \item[\color{red} $S_\omega $ \color{blue}] is defined as
          \color{red}  
        \begin{multline}
         S_\omega=Sprbeame+Sprbeami+Sprcxl+Spreimpt+\\
             Sprcx+Sprcxre 
        \end{multline}
          \color{blue}
        where the definition of each of the individual terms follows.
         \begin{description}  %begin 1b
        \item[\color{red} sprbeame \color{blue} ] is the (delayed ) source of angular momentum 
                transferred from the (beam) fast ions  to the electrons
                during the slowing down process.This angular momentum is
                rapidly shared with the ions and is thus a source term for the
                thermal ion toroidal momentum equation:
         \beq \color{red}
             sprbeame(\rho)=bke(\rho,e_b,b_j)*spbr(\rho,e_b,b_j)
         \eeq \color{blue}
        \item[ \color{red} sprbeami \color{blue}] is the (delayed ) source of angular momentum
        transferred from the (beam) fast ions to the thermal ion
        fluid.
         \color{red}
         \begin{multline} 
             sprbeami(\rho)=bki(\rho,e_b,b_j)*spbr(\rho,e_b,b_j)+ \\
                            fbth(\rho,e_b,b_j)*sb(\rho,e_b,b_j)*atw_b \\
                        *m_p*v_z(\rho)\frac{<R^2>}{<R>}
         \end{multline} \color{blue}
        \item[ \color{red} ssprcxl  \color{blue}] represents the gain of  angular momentum due to charge
                    exchange of a fast ion with a thermal neutral (the
                    thermal neutral adds its momentum to the thermal
                    ion distribution) \color{red}
        \begin{multline} 
                ssprcxl(\rho)=fprscxl*spbr(\rho,e_b,b_j) \\
                  +fscxl*sb(\rho,e_b,b_j)*atw_b \\
                        *m_p*v_z(\rho)*\frac{<R^2>}{<R>}
        \end{multline}  \color{blue}
        \item[ \color{red} spreimpt  \color{blue}] represents gain of momentum due to electron impact
        ionization of thermal neutrals \color{red}
                \begin{multline} 
                spreimpt(\rho)=\sum_{i=1}^{nprim}eirate(\rho)*enn_i(\rho)
                    m_i*vneut_i(\rho)*<R> \label{eq:spreimpt}
        \end{multline}  \color{blue}
        \item[ \color{red} sprcx  \color{blue}] represents the source/sink of momentum due to charge
        exchange of thermal neutral with thermal ion. For two ion and
        neutral species we have \color{red}
                \begin{multline} 
                sprcx(\rho)=\sum_{i=1}^{nprim}enn_i(\rho)*cexr_i(\rho)\\
                    en_i(\rho)*atw_i*m_p*<R>*(vneut_i(\rho)-vionz(\rho))\\
                   +cxmix*(atw_k*vneut_k(\rho)-atw_i*v_z(\rho))
        \end{multline}  \color{blue}
        \item[ \color{red} sprcxre  \color{blue}] represents the sink of thermal angular momentum due to
        charge exchange of thermal ion with a fast neutral and also
        includes radiative recombination of thermal ions.  \color{red}
        \beq 
                sprcxre(\rho)=\sum_{i=1}^{nprim}sbcx_i(\rho)*<R^2>
                    \omega m_i  \label{eq:sprcxre}
        \eeq  \color{blue}
         \end{description} %end 1b
        \item{  \color{red} $S_\omega^{2D}$  \color{blue} } is the source term due to evolution of
             the mhd equilibrium. It is given by  \color{red}
           \begin{multline} 
            S_\omega^{2D}=-\omega\sum_{i=1}^{nprim}\bigg( m_in_i\ddiff{t}<R^2>
                     -m_in_i<R^2>\omega\ddiff{t}\ln H \\
                  +\frac{d}{H}\ddiff{\rho}H\omega m_in_i<R^2> \bigg )
           \end{multline} 
         \color{blue}
        spr2d  is the name of this term in the code output. At present the
        individual contributions are not broken out. In the code this term 
         is multiplied by an input factor
        $angrm2d(4)$,which is defaulted to 1.0
        but the user can assign any value (eq. 0.0) to gauge the effect of
        this term
        \end{description}

   \end{itemize}
   \end{slide}
\end{document}  