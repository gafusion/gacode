\begin{filecontents}{abbrev.tex}
   \newcommand{\vp}{ V^{\prime}}
   \newcommand{\ot}{\emph{ONETWO }}
   \newcommand{\pdiff}[2]{\frac{\partial{ #1}}{\partial{ #2}}}
   %useage \pdiff{#1}{#2}
   \newcommand{\ddiff}[1]{\frac{\partial}{\partial #1}}
   \newcommand{\strt}{\begin{eqnarray}}
   \newcommand{\trts}{\end{eqnarray}}
   \newcommand{\beq}{\begin{eqnarray}}
   \newcommand{\eeq}{\end{eqnarray}}
 \textheight8.5in
\end{filecontents}






   \documentclass[12pt]{article}
   %\pagestyle{myheadings}\markboth{ /u/stjohn/ONETWOPAGES/code\_descr/source\_o12.ps}{ /u/stjohn/ONETWOPAGES/code\_descr/source\_o12.ps}
   %{ /u/stjohn/tex/o12ref.dir/source\_o12.tex}
   %% LaTeX2e file `abbrev.tex'
%% generated by the `filecontents' environment
%% from source `sources_o12' on 2010/05/19.
%%

   \newcommand{\vp}{ V^{\prime}}
   \newcommand{\ot}{Onetwo\xspace}
   \newcommand{\ct}{Curray\xspace}
   \newcommand{\pdiff}[2]{\frac{\partial{ #1}}{\partial{ #2}}}
   %useage \pdiff{#1}{#2}
   \newcommand{\ddiff}[1]{\frac{\partial}{\partial #1}}
   \newcommand{\strt}{\begin{eqnarray}}
   \newcommand{\trts}{\end{eqnarray}}
   \newcommand{\beq}{\begin{eqnarray}}
   \newcommand{\eeq}{\end{eqnarray}}
   \newcommand{\pdiffn}[3]{\frac{\partial^{#3}{#1}}{\partial{ #2}^{#3}}}
   \newcommand{\pdiffz}[3]{\frac{\partial{#1}}{\partial{ #2}}\bigg \vert_{#3}}
    \newcommand{\myint}{\int\limits_{\rho_{j-\frac{1}{2}}}^{\rho_{j+\frac{1}{2}}
}d\rho H \rho}
    \newcommand{\myind}{\int\limits_{\rho_{j-\frac{1}{2}}}^{\rho_{j+\frac{1}{2}}
}d\rho}
    \newcommand{\onehalf}{\frac{1}{2}}
    %\newcommand{\eqref}[1]{(\ref{#1})}
    \newcommand{\Eqref}[1]{Eq.~\eqref{#1}}
    \newcommand{\Eqrefs}[2]{Eqs.~\eqref{#1}-\eqref{#2}}

 \textheight8.5in
\newenvironment{narrow}[2]{%
   \begin{list}{}{%
   \setlength{\topsep}{0pt}%
   \setlength{\leftmargin}{#1}%
   \setlength{\rightmargin}{#2}%
   \setlength{\listparindent}{\parindent}%
   \setlength{\itemindent}{\parindent}%
   \setlength{\parsep}{\parskip}}%
\item[]}{\end{list}}

 % can keep my definitions in file abbrev.tex
   \usepackage{epsfig,color}
   \usepackage{amsmath} 
   %\usepackage{amstex} %need this package for Sb environment ??
   \listfiles

   %\includeonly{sources_o12_bib}\ref{}

\begin{document}
       \section {Summary}
       \begin{itemize}
         \item An AT type ITER-FEAT discharge scaled up from DIII-D
           shot 106029 was analyzed for RWM stability.
         \item The scaled up discharge has density $\approx n_{GW} $
               With $\beta_NH(89p) = 8.46 $, $QDT =  20.3$, with 33Mw of
               1Mev negative  neutral beam injection.
         \item Primary and impurity ion densities and (equal)
           electron and ion temperature profiles were determined in
           such a way as to be
           consistent with the mhd equilibrium pressure and the fast
           alpha and neutral beam stored energy densities.
         \item It is shown that 33Mw of neutral beam 
               injection is sufficient to maintain the central
               toroidal rotation speed at 2.e4 rad/sec using
               ion power balance diffusivity for the momentum 
               diffusivity. According to ref{2} the required central
               rotation speed for RWM stability is about 1.5e4
               rad/sec.
     \end{itemize}
         \newpage
       \section{ Generating Profiles Consistent with Mhd Equilibrium
         Pressure Profiles}
       As an application of the Iter-Feat concept DIII-D H mode shot
       106029 was used to determine an equilibrium which could
       be used as a basis for further studies. This work is described
       in ref[1].
       
       For transport analysis we would like to maintain the pressure
       profile in this scaled up equilibrium,identified as
       $'iter\_wrs\_d3d\_4\_129.gfile'$. Thus we have to find
       densities and temperatures which are consistent with the
       assumed equilibrium pressure profile and which yield
       sufficiently attractive performance parameters. In the present
       work this was accomplished by assuming an electron density
       profile,see fig 1,at approximately 1.0 times the Greenwald
       limit. To determine the primary ion and impurity densities and 
       the electron and ion temperatures (assumed equal for this
       study) we consider the equations for charge neutrality, zeff,
       and equilibrium pressure together with  auxiliary 
       information necessary to solve the following set of equations:

          \begin{align} \label {zeff-ref}
             n_{e} -Z_{p1}n_{p1}  -Z_{p2}n_{p2}  -Z_{imp1}n_{imp1}
             -Z_{imp2}n_{imp2} &= Z_bn_b +Z_{\alpha}n_{\alpha}\\
             Z_{eff} n_{e} -<Z_{p1}^2>n_{p1}-<Z_{p2}^2>n_{p2} \notag
             \\ 
              -<Z_{imp1}^2>n_{imp1}-<Z_{imp2}^2>n_{imp2} &=
              Z_b^2n_b +Z_{\alpha}^2n_{\alpha} \\
             n_{e}C_eT_e +n_{p1}C_iT_i+ n_{p2}C_iT_i \notag \\
             +n_{imp1}C_iT_i
             +n_{imp2}C_iT_i &= P - \frac{2}{3}(w_{beam}+w_{\alpha})\\
           Z_{frac}n_{p1} - n_{p2} &= 0 \\
           Z_{impfrac}n_{p1} - n_{imp2} &=0 
         \end{align} 

         Here we assume two primary (ie hydrogenic) ions and two
         impurities with the last two equations specifying the amounts
         of the second species in each case. We further assume that the beam
         and $ \alpha $ densities and stored energy densities are
         given. This implies that an iterative process is required to
         solve the linear set of equations since beam deposition and
         fusion rates depend on the unknown densities and
         temperatures. P is the known pressure
         profile from the equilibrium calculations. We
         take the electron density profile is known which
         eliminates the first of the equations EQ[\ref{zeff-ref}].
         The above set of equations applies at each value of the minor
         radius grid $\rho $. 
         The parameters multiplying the electron
         and ion temperatures, $C_e,C_i$ are set to unity if
         the electron density is solved for as part of the above
         equation set. Otherwise  $C_e,C_i$ are automatically adjusted at 
         each radius in such a way that the error in
         Eq[\ref{zeff-ref}] is minimized. Typically we want to fix
         the electron density at a specified value so the later case 
         is generally the one that arises.
         
         
         \begin{center}
         %\begin{minipage}[b]{.46\linewidth} 
         \rotatebox{-90.}{\resizebox{6.0 in}{6.0 in}{
               \includegraphics{zeff_fig1.eps}}}
        %\end{minipage}\hfill
          \end{center}
        \pagebreak



       \section{Neutral beam Injection}
       Due to the large densities involved sufficient penetration for
       neutral beam heating and current drive requires a high energy
       beam. We have assumed a 1Mev(negative ion)  DT beam with injected power of
       33Mw. The deposition profile for the beam is given in Fig. 2. A 
       visual indication of the fast ion birth points in the plasma
       in cross section and top views is given in Fig 3 and 4
       respectively. For the tangential injection shown shinethrough
       is essentially zero. 
           \\
          \\
           \\
           \\
        
         \begin{center}
         %\begin{minipage}[b]{.46\linewidth} 
         \rotatebox{0.}{\resizebox{4.5 in}{4.5 in}{
               \includegraphics{cgmplot_0_12457.eps}}}
        %\end{minipage}\hfill
        \pagebreak
 
         \rotatebox{0.}{\resizebox{4.5 in}{7.5 in}{
               \includegraphics{cgmplot_0_12470.eps}}}
          \end{center}
        \pagebreak






       \section{Transport Analysis}
       The primary question to be addressed here is the stabilization
       of resistive wall modes. To accomplish this stabilization we
       need to ensure that the plasma is spinning sufficiently
       rapidly. Analyses done by M.Chu,ref [2], indicated that a value of 15\% 
       of the observed central toroidal rotation speed value in DIII-D
       shot 106029 was sufficient to stabilize the largest mode
       amplitude which occurred at the magnetic axis. The DIII-D rotation speed 
       profile was scaled up by maintaining the same fractional
       central Alfven 
       speed in the scaled up discharge. This leads to a toroidal
       rotation speed profile shown  as the solid line in Fig.(5).

       To determine the balance between neutral beam torque input and
       drag and momentum diffusion we have to solve the toroidal
       momentum balance equation. This equation requires that we
       specify an momentum diffusivity. Since the determination of 
       the plasma kinematic viscosity from theoretical principles
       does not match experimental results we are forced to use
       alternative means to determine the momentum diffusivity. An
       often used criterion is to assume that the toroidal momentum
       diffusivity is equal to the ion diffusivity. We have adopted
       this method as well and further simplified the problem by using 
       a power balance diffusivity for the ions. With the electron and 
       ion temperatures determined consistently with the equilibrium
       pressure as described in section 1,we obtain the power  balance
       diffusivity given in Fig (6). For orientation the ion
       neoclassical diffusivity is also indicated.

       With the given (fixed in time and space) power balance
       diffusivity we find that the toroidal rotation speed profile
       can be maintained at a level higher than required for resistive 
       wall mode stabilization which ,in accordance with ref [2],is
       assumed to be 15\% of  the central value.  In Fig(5) we show
       the rotation speed profile for a  neutral 
       beam injected power of  33Mw.  The rotation speed profile 
       is in steady state at the time shown. The central
       value of$ 2.0x10^4 $ is higher than the required value of
       about $15\% (1.0x10^5) = 1.5x10^4 \frac{rad}{sec}$ for
       stability.
        \begin{center}
         %\begin{minipage}[b]{.46\linewidth} 
         \rotatebox{90.}{\resizebox{4. in}{4. in}{
               \includegraphics{rotationss.eps}}}
        %\end{minipage}\hfill
          \end{center}
        \begin{center}
         %\begin{minipage}[b]{.46\linewidth} 
         \rotatebox{90.}{\resizebox{4. in}{4. in}{
               \includegraphics{chiei.eps}}}
        %\end{minipage}\hfill
          \end{center}
        \pagebreak
        
        Table I represents a summary of the transport results.
        It should be noted that beam-beam fusion reactions were not
        included in the analysis.
        \begin{verbatim}
                               Table I             
  Transport analysis summary:  Iter-Feat H mode (106029 extrapolation)
 
  Minor radius a (cm):   187.0         b/a:                     1.98
  Nominal Rmajor (cm):   620.0         R at geom. cent. (cm): 634.1
  R at mag. axis (cm):   672.9         Z at mag. axis (cm):    51.9
  Volume (cm**3):          7.88E+08    Pol. circum. (cm):    1804.9
  surface area (cm**2):    6.89E+06    cross. sect area   :  2.04E+05

  Bt (G):                  5.30E+04    Ip  (A):                 1.00E+07
  Bt at Rgeom (G):         5.18E+04    r(q = 1)/a:                0.00
  Line-avg den (1/cm**3):  9.09E+13    Tau-particle-dt (s):     0.200
  nGW = 1.0e14

      profiles             ucenter     uedge     ucen/uav
  elec. den. (1/cm**3):    1.06E+14    2.11E+13    1.25
  elec. temp. (keV):      23.53        2.34        2.20
  ion   temp. (keV):      23.53        2.34        2.20
  current  (A/cm**2):     52.08       51.34        1.05
  Zeff:                    1.05        2.16        0.58
  q:                       2.52        9.34
  q* at edge:                          4.92
  ang. speed (1/sec):     1.98E+04    4.30E+03    1.66E+00
                            exper.          code
  Surface voltage  :       0.00            0.00      Volts
  Average voltage  :       0.00            0.01      Volts
  Ohmic power:             0.00E+00        9.27E+04  Watts
  Beam power-torus:        3.30E+07        3.30E+07  Watts
  Neutron rate:            0.00E+00        7.05E+17  #/s
 
      computed quantities
  Beam power elec. (W):    2.00E+07    ke at a/2 (1/cm-s):      1.15E+18
  Beam power ions  (W):    1.27E+07    ki at a/2 (1/cm-s):      5.66E+17
  Beam power cx loss (W):  3.80E+03    ki/kineo at a/2:        20.72
  Shinethrough (%):        0.02        chi electrons at a/2:    1.27E+04
  RF power absorbed:       0.00E+00    chi ions at a/2:         7.07E+03
  Radiated power   (W):    2.00E+07    r*/a: Te = (Te(0)+Te(a))/2 0.56
  Poloidal B field (G):    6.96E+03    Beta-poloidal:           1.868
  beam torque (nt-m)       4.45E+01    total torque (nt-m):    -3.25E+02
  stored ang mtm (kg*m2/s):  3.18E+02  momt inertia (kg*m**2):  1.10E-02
 
                           electrons     ions       thermal      total
  Stored Energy (J)        1.837E+08   1.644E+08   3.480E+08   4.080E+08
  dE/dt (W):               1.238E+04   3.576E-06   1.238E+04   1.238E+04
  Input power (W):         1.178E+08               1.661E+08   1.661E+08
  Energy conf. time (s):   1.5589                  2.0950      2.4561 
  angular momentum confinement time (sec)          2.4356
 
      Beta-toroidal        volume-avg       center
  electrons:               1.451E-02       3.742E-02
  ions:                    1.299E-02       3.498E-02
  beam:                    1.101E-03       1.789E-02
  alphas:                  3.638E-03       1.866E-02
  total:                   3.224E-02       1.089E-01 H(89p) =    2.81
  beta_N*H(89p)            8.46
  total power input (Mw) =  3.3000E+07  time =  1.2000E+00
  Itot =  1.00E+07  Iohm =  3.19E+06  Iboot =  4.88E+06  Ibeam =   1.93E+06 
  Irf =   0.00E+00
  QDD =   0.023998 QDT =  20.320179 QTT =   0.037929
  BEAM-BEAM FUSION RATE NEGLECTED!
  














 \end{verbatim}

  
\noindent  REFERENCES \\
 Leur,J.A.,L.L.Lao DIII-D Engineering Physics
 Memo,D3DJAL010918b,10/05/01  \\
  Chu,M., private communication,11/28/01

\end{document}
