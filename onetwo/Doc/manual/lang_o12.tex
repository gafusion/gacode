   \documentclass[12pt]{article}
   %% LaTeX2e file `abbrev.tex'
%% generated by the `filecontents' environment
%% from source `sources_o12' on 2010/05/19.
%%

   \newcommand{\vp}{ V^{\prime}}
   \newcommand{\ot}{Onetwo\xspace}
   \newcommand{\ct}{Curray\xspace}
   \newcommand{\pdiff}[2]{\frac{\partial{ #1}}{\partial{ #2}}}
   %useage \pdiff{#1}{#2}
   \newcommand{\ddiff}[1]{\frac{\partial}{\partial #1}}
   \newcommand{\strt}{\begin{eqnarray}}
   \newcommand{\trts}{\end{eqnarray}}
   \newcommand{\beq}{\begin{eqnarray}}
   \newcommand{\eeq}{\end{eqnarray}}
   \newcommand{\pdiffn}[3]{\frac{\partial^{#3}{#1}}{\partial{ #2}^{#3}}}
   \newcommand{\pdiffz}[3]{\frac{\partial{#1}}{\partial{ #2}}\bigg \vert_{#3}}
    \newcommand{\myint}{\int\limits_{\rho_{j-\frac{1}{2}}}^{\rho_{j+\frac{1}{2}}
}d\rho H \rho}
    \newcommand{\myind}{\int\limits_{\rho_{j-\frac{1}{2}}}^{\rho_{j+\frac{1}{2}}
}d\rho}
    \newcommand{\onehalf}{\frac{1}{2}}
    %\newcommand{\eqref}[1]{(\ref{#1})}
    \newcommand{\Eqref}[1]{Eq.~\eqref{#1}}
    \newcommand{\Eqrefs}[2]{Eqs.~\eqref{#1}-\eqref{#2}}

 \textheight8.5in
\newenvironment{narrow}[2]{%
   \begin{list}{}{%
   \setlength{\topsep}{0pt}%
   \setlength{\leftmargin}{#1}%
   \setlength{\rightmargin}{#2}%
   \setlength{\listparindent}{\parindent}%
   \setlength{\itemindent}{\parindent}%
   \setlength{\parsep}{\parskip}}%
\item[]}{\end{list}}

 % can keep my definitions in file abbrev.tex
   \usepackage{amsmath} 
   \begin{document}


   \title{BRIEF SYNOPSIS OF ONETWO}
   \author{H.ST.JOHN}
   \date{   }   %make the date just some blanks
   \maketitle


     % \Large
       \boldmath\


      \section{Description}
    The Onetwo transport code solves a user selectable,arbitrary
 combination,of flux surface average parabolic equations representing
 the radial diffusion and convection of 1 or 2 thermal ion species and
 up to 2 neutral species,the electron energy,the ion energy,the poloidal
 magnetic field,and the toroidal momentum.  The magnetic geometry may be
 self consistently maintained by solving the elliptic mhd equilibrium
 equation simultaneously with the diffusion equations.  Both free and
 fixed plasma boundary equilibrium models are supported. Auxilliary
 heating models include Monte Carlo neutral beam deposition and electron
 and ion cyclotron and fast wave r.f. heating.  Models for ohmic,bootstrap,
 beam and r.f. driven currents are included.  Energy confinement may be
 simulated by selecting theoretical or empirical electron and/or ion
 thermal conductivity models,including Rebut-Lallia,Shay,and Waltz-
 Dominguez.  A unique feature of the code is its ability to substitute
 measured profiles for any of the dependent variables and solving the
 diffusion equations only for the remaining unkown profiles. A combination
 of Crank-Nicholson,predictor and iterated corrector methods are used to
 converge the non linear diffusion equations. The elliptic equations are
 solved using variable finite difference schemes with single cyclic
 reduction. 
    \ot is written in Fortran 77,it does not at the present time use
 any Fortran 90 extensions. The code is routinely compiled and runs on
 Cray2,C90,Cray YMP Cray J90 (all of which use the Unicos operating
 system),Hp 9000 series 700,800 (Hp Ux operating system),Digital
 Alpha (unix),silicon graphics (IRIS operating system) and Sun (Solaris
 operating system). \ot actually consists of a suite of codes,onetwo,
 gafit,toray,curray,preplt,verdi,trplot,eqplot,nubplt,xplot,ploteq,psiplt,
 and fixbdry. 
 

\section{Input}
  All of the input to the \ot code is done through the fortran
  namelist construct.
  Basic run parameters:
  \begin{itemize}
  \item $time0,timmax$ the start and end times of the analysis
  \item specification of which of the dependent variables is
         run in analysis mode and which are run in simulation mode,
         $itenp,itte,itti,itxj,itangrot =0,1$
  \item the primary impurityion  and neutral species,input as 
        ASCII characters.
  \item the particle confinement time,$taupin $ in sec,used to set
	the neutral density.
  \item the $5 x 5 $ neoclassical transport matrix
 \item the bootstrap current selection model switch
 \item $inenez $ indicator switch,selects option of inputting primary
	io densities or calculating them self consistently from given
	zeff and electron density profiles. 
  \item a number of integer switches used to control presentation
        of the output and plots.
  \item integer switches to select sawtooth, balloning mode and
	confinement  models.
  \end{itemize}
  The code requires that profiles (ie initial conditions) be given.
  Profile input  is done in terms of a normalized variable
 $ \rho $ which is equal to the square root of the toroidal flux
 inside a given flux surface divided by the total toroidal flux inside the
 plasma.  Measured kinetic data is
 parameterized in terms of a least squares cubic spline fit as a function
 of $\rho $.  The input profiles required are:
 \begin{itemize}
  \item $n_e$,the electron density,$\frac{1}{cm^3}$
  \item $T_e$,the electron temperature, kev
  \item $T$,the ion temerature, kev
  \item $\omega $ the toroidal rotation speed profile,$\frac{rad}{sec}$
  \item $zeff $  the effective charge number,usually measured
   \item $qrad$ the radiated power density 
 \end{itemize}
 The above profiles are input as function values at specific $\rho$ locations
 commonly refered to as ``knots''. Each profile may be time dependent so
 that for example the input for the ion temperature T takes the form
 \beq
          T(i,j) i=1,...no.\  knots,\ j=1... no.\  time intervals
 \eeq
	with the actual times specified in  $bctime(j) $ .
 The boundary conditions used with these profiles are zero gradient
 at the magnetic axis and derivative of the profile at the plasma edge.

 Neutral beam  input.
 \begin{itemize}
  \item the number of beam lines
  \item the beam line aiming parameters
  \item the energy of each beam
  \item the power of each beam
  \item the cross section set (ADAS or Boley)
  \item integer switches to control frequency of beam-beam,beam-thermal
	neutron rate calculations and beam density iteration/convergence.
 \end{itemize}
 

 Basic mhd data input is in terms of the ``eqdsk'' data file format
 described under EFIT.
 The basic input consists of
 \begin{itemize}
  \item a description of the cylindrical R,Z grid
  \item the number of flux surfaces
  \item $ \pdiff {P}{\psi} $ the pressure gradient
  \item $ \pdiff {F}{\psi} $ the diamagnetic flux function
  \item $ q $ the safety factor profile
  \item $ \psi(R,z) $ an array of psi values 
  \item $\langle J \frac{R}{R_0} \rangle$ the flux surface average
	toroidal current density (used the  initial condition for
        evolution of the poloidal magnetic field)
  \item $r(i),z(i),i=1..n$ the R,Z coordinates of the plasma surface.
  \item $r_{lim}(i),z_{lim}(i),i=1..n_{lim}$ the R,Z coordinates of the limiter.
  Other mhd input includes integer switches to select whether or not
  mhd calculations will be coupled with transport simulations,selection
 of fixed or free boundary equilibrium models,and startup options.
 \end{itemize}

\section{NetCdf Data Format}
 Onetwo is currently being converted to use the network common data format
 for its output. Many commercial packages,including IDL, support the netCDF
 data format. In addition netCDF files written by C,Fortran or Perl programs
 are interchangeable amongst themselves (so that a netCDF file created
 in Fortran can be read by a C program for example).
 \nocite{*}


     \bibliographystyle{unsrt}   % list citations in order of appearance
     \bibliography{/u/stjohn/tex/tpx_project.dir/bibliography.dir/tokamak,
     /u/stjohn/tex/tpx_project.dir/bibliography.dir/comp_in_phys}  
    % database file is tokamak.bib,documentstyle-article prints
     % the bibliography with a `REFERENCES' header whereas
     % documentstyle-report prints 'BIBLIOGRAPHY' header


\end{document}
