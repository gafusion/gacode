%\large    %type size ,must be given after begin{document}
\boldmath

\section{Some Mathematical Details}

\label{appendix1}
\bigskip
\hrule height2pt
\bigskip
The element of  area on the surface of the toroidal plasma is $ 2 \pi R dl $
where $dl$ is an element of length along a given (cross sectional) psi contour.
The total plasma surface area is thus
\begin{equation}
 S(\rho) =\oint{2 \pi R dl}=2 \pi \oint{\frac{B_{p}Rdl}{B_{p}}}  
 \label{eq:area}.
\end{equation}
The flux surface average operation is defined,for an arbitrary function A, as
\begin{equation}
 \langle A\rangle \equiv \frac{\oint{ \frac{A dl}{B_p}}}{\oint{ \frac{dl}{B_p}}}
 = \frac{\oint{ \frac{A dl}{B_p}}}    {\frac{1}{2 \pi}
 \frac{\partial V}{\partial  \psi} }
 \label{eq:flavg},
\end{equation}
where the rate of change of plasma volume with respect to flux surface label $
\psi$, $\frac{\partial V}{\partial \psi}$, has been introduced. Hence we can
write \Eqref{eq:area} as
\begin{equation}
 S(\rho) =\left( \frac{\partial V}{\partial \rho} \right )
 \left( \frac{\partial \rho}{\partial \psi}\right) \langle B_pR\rangle
 =4{\pi}^2R_0H\rho\frac{\partial \rho}{\partial \psi}
 \langle B_pR\rangle    \label{eq:sdef},
\end{equation}
where $ H $ is defined as 
\begin{equation}
 H \equiv \frac{\frac{\partial V}{\partial \rho}}{4\pi^2R_0\rho}
 \label{eq:hdef}.
\end{equation}
The flux surface average poloidal $ B $ field is defined as
\begin{equation}
 B_{p0} \equiv     \frac{1}{R_0} \frac{\partial \psi}{\partial \rho}
 \label{eq:bpdef}
\end{equation}
so \Eqref{eq:sdef} becomes
\begin{equation}
 S(\rho) = 4\pi^2R_0H\rho\left \langle\frac{RB_{P}}{R_0B_{P0}}\right\rangle
 \label{eq:sdef1}.
\end{equation}
Since $\nabla\rho = \frac{\partial \rho}{\partial \psi} \nabla \psi $ and 
$B_P=\frac{\mid\nabla\psi\mid}{R}$ we can also write the surface area  as
\begin{equation}
 S(\rho)=4\pi^2R_0H\rho\langle\mid\nabla\rho\mid\rangle
 \label{eq:sdef3a}.
\end{equation}
By the definition of $ H $ this can also be expressed as
\beq
 S(\rho)=  \frac{\partial V }{\partial \rho} 
 \langle\mid\nabla\rho\mid\rangle
 \label{eq:sdef3}.
\eeq
Note that $ S(\rho) $ represents the true physical surface area of the  nested
flux surfaces. 

Quantities that have to be integrated over the cross sectional area of the flux
tubes, such as the toroidal  current density $ J_\phi $, are developed as
follows:
\beq
 \int{J_\phi dA} = \int {J_\phi(R,Z)dRdZ }
 = \int{J_\phi\frac{d\Psi dl}{R B_p}},
\eeq
where we have used $ dRdZ = \frac{d\Psi dl}{R B_p} $. This result can be
manipulated into the flux surface average expression
\beq
 \int{J_\phi dA} = 2\pi  \int {\left\langle\frac{J_\phi(R,Z)R_0}{R}
 \right\rangle H \rho d \rho }
\eeq
by using $\frac{dV}{d\Psi}= 2\pi \int \frac{dl}{B_p} $.

\subsection{Flux Surface Averaging of Diffusion Equations}

As an example of how flux surface averaging proceeds consider  the electron
energy balance equation. It  can be written in the form
\beq
 \frac{3}{2} \pdiff{P_e}{t}+\nabla\cdot\left( Q_e +\frac{5}{2}
 P_eV_e\right)=S+\vec{J}\cdotp\vec{E}-Q_{\delta}+ V_i\cdotp\nabla P_i.
\eeq
If we apply the flux surface averaging operation to this equation and multiply
by $ V^{\prime} $ we get
\beq
 \frac{3}{2}\vp\left<\pdiff{P_e}{t}\right>+\vp\left\langle
 \nabla \cdotp\left (Q_e+\frac{5}{2}P_eV_e\right)\right\rangle
 =\vp\left\langle rhs\right\rangle
 \label{eq:eng}.
\eeq
An easily established property of the flux surface averaging is that 
\beq
 \left\langle\nabla \cdotp \vec{A}\right\rangle=\frac{1}{\vp}
 \ddiff{\rho} \left[ \vp \left\langle A\cdotp \nabla \rho
 \right\rangle\right]
 \label{eq:divg}.
\eeq
Applying this result to the divergence term in \Eqref{eq:eng} we obtain
\beq
 \frac{3}{2}\vp\left<\pdiff{P_e}{t}\right>+\ddiff{\rho}
 \left[\vp\left<\left(Q_e+\frac{5}{2}P_eV_e\right) \cdotp
 \nabla \rho \right> \right] = \vp\left<rhs\right>
 \label{eq:eng1}.
\eeq
Now use
\beq
 \vp\left<\left.\pdiff{A}{t}\right|_{R,Z}\right> = \left.\ddiff{\rho}\right|_t
 \left(\vp\left<A\vec{u_\rho} \cdotp \nabla \rho \right>\right)
\eeq
to write \Eqref{eq:eng1} in the form
\beq
 \begin{array}{c}
 \displaystyle
 \frac{3}{2}\left.\ddiff{t}\right|_\rho \left(\vp P_e\right)
 -\frac{3}{2}\left.\ddiff{t}\right|_{t} \left(\vp\left<P_e\vec{u_\rho}
 \cdotp \nabla \rho \right> \right)\\  
 \displaystyle
 +\left.\ddiff{\rho}\right|_t \left(\vp \left<\vec{Q_e} \cdotp
 \nabla \rho \right> + \frac{5}{2} \left<
 \vp P_eV_e \cdotp \nabla \rho \right> \right)
 \end{array}
 = \vp\left<rhs\right>
 \label{eq:eng2}.
\eeq
The important definition is that of the flux surface average energy flux, which
as seen from the above equation should be defined as
\beq
 q_e\equiv \left<\vec{Q}_e \cdotp \nabla \rho \right>
 \label{eq:qedef}.
\eeq
The conductivity, $K_e $, can depend on more than just one space dimension. It
is defined as the constant of proportionallity relating the flow to the
gradient:
\beq
 \vec{Q}_e=-K_e \nabla T_e .
\eeq 
If we assume that $T_e $  is a flux
surface functions then we can write
\beq
 -K_e \nabla T_e = -K_e \pdiff{T_e}{\rho}\nabla\rho .
\eeq
Hence
\beq
 q_e=-\pdiff{T_e}{\rho} \left<K_e \mid \nabla \rho\mid^{2} \right>.
\label{eq:qedef2}
\eeq
The actual amount of energy flowing out of any given surface is
\beq
 \iint\vec{Q}_e\cdotp d\vec{A}=\pdiff{V}{\rho}q_e
 \label{eq:qe}
\eeq
The flow of energy out of any flux surface, as given by \Eqref{eq:qe}, is the
quantity that has physical significance and must be conserved. Note however that
the RHS of \Eqref{eq:qe} is not expressed in the customary form of the  surface
area times the  energy flux flowing through that area. We can change 
\Eqref{eq:qe} into this form  by multiplying and  dividing by the factor
$\left<\mid\nabla\rho\mid\right>$ :
\begin{eqnarray}
 \iint\vec{Q}_e\cdotp d\vec{A}& = 
 & \left<\mid\nabla\rho\mid\right>\pdiff{V}{\rho}
 \left( \frac{q_e}{\left<\mid\nabla\rho\mid\right>} \right) \\
 & \equiv & S(\rho) q_e^*
 \label{eq:qe2}.
\end{eqnarray}
Here $S(\rho)$ is the true physical area of the flux surface [see
\Eqref{eq:sdef3}] and $q_e^*$ is the average flux through this surface. Some
codes use this definition with the result that the divergence part of the
diffusion equation looks like 
\beq
 \frac{1}{\vp} \left.\ddiff{\rho}\right|_t 
 \left(\vp \left<\mid\nabla\rho\mid\right>q_e^* +\cdots  \right ) 
 \label{eq:qe3}   .  
\eeq

As is obvious from examination of the equations in Section \ref{sec:transport},
\ot does not use the convention given by \Eqref{eq:qe3}. Instead \ot cancels the
common factor of $\left<\mid\nabla\rho\mid\right> $ from $S(\rho)$ and $q_e^*$
and uses $q_e$ as an effective flux. This flux must then be multiplied by an
effective surface area, $\pdiff{V}{\rho} $, in order to maintain the proper
energy flow as given by \Eqref{eq:qe}. Internally the code calculates the
conductive flux from the expression
\beq
 q_e = -\kappa_e \pdiff{T_e}{\rho} \label{eq:qe4}.
\eeq
Hence for the anomalous transport models the user must supply a definition of
$\kappa$  adjusted so that the physical flow of energy is given by
\Eqref{eq:qe}.

As an example we have the IFS anomalous conductivity model \cite{IFS} which
states that the conduction of energy out of any given  flux surface is the LHS
of the following equation:
\beq
 \left(-n\chi_{IFS}\pdiff{T_e}{\rho}\right)\vp\left<\mid\nabla\rho\mid\right>
 =\pdiff{V}{\rho}q_e .
\eeq
The RHS of this equation follows from the physical condition, \Eqref{eq:qe} and
gives us the definition of $\kappa $, \Eqref{eq:qe4} (or $\chi$ ), that should
be used in \ot:
\beq
 \chi_{\ot} =  \chi_{IFS}\left<\mid\nabla\rho\mid\right>.
\eeq 
The effective flux put out by \ot will be defined so that it must be multiplied
by $\vp $ (and not the true surface area $\vp\left<\mid\nabla\rho\mid\right>$)
to get the true flow as illustrated with the following sequence of equations:
\begin{eqnarray}
 q_{e\ot} & = & -n_e\chi_e\pdiff{T_e}{\rho} \\
 & = & -n_e\chi_{IFS}\left<\mid\nabla\rho\mid\right>\pdiff{T_e}{\rho} \\
 & = &  q_{eIFS}\left<\mid\nabla\rho\mid\right> \\
 q_{e\ot}\vp & = &   q_{eIFS}\left<\mid\nabla\rho\mid\right> \vp \\
 & = &  q_{eIFS} S(\rho).
\end{eqnarray}
