


\begin{filecontents}{abbrev.tex}
   \newcommand{\clr} {\color{red}}
   \newcommand{\clb} {\color{blue}}
   \newcommand{\vp}{ V^{\prime}}
   \newcommand{\ot}{\emph{ONETWO }}
   \newcommand{\pdiff}[2]{\frac{\partial{ #1}}{\partial{ #2}}}
   %useage \pdiff{#1}{#2}
   \newcommand{\ddiff}[1]{\frac{\partial}{\partial #1}}
   \newcommand{\strt}{\begin{eqnarray}}
   \newcommand{\trts}{\end{eqnarray}}
   \newcommand{\beq}{\begin{eqnarray}}
   \newcommand{\eeq}{\end{eqnarray}}
   \newcommand{\pdiffn}[3]{\frac{\partial^{#3}{#1}}{\partial{ #2}^{#3}}}
   \newcommand{\pdiffz}[3]{\frac{\partial{#1}}{\partial{ #2}}\bigg \vert_{#3}}
   \textheight8.5in
   \endinput
\end{filecontents}
\documentclass{slides}  
  %note: the landscape option above simply tells tex to
  %  use the long dimension of the page in calculating its layout
  % (if this is all that is done the the lines would run off the paper
  % when printed)
  %also need to use landscape option in dvips, which actually rotates
  %everything so that it fits across the long side of the page.
  %(if landscape option is used in dvips but not in tex document
  % then the text will be too short for the long side of the page)
 \usepackage{amsmath} 
 %\usepackage{fancyheadings,graphics,color}
 \usepackage{fancyheadings,graphics}
 \usepackage[dvips]{color}
   %\usepackage{epsfig}
  %\usepackage[usenames]{color} seems logical but doesn't work
   %% LaTeX2e file `abbrev.tex'
%% generated by the `filecontents' environment
%% from source `sources_o12' on 2010/05/19.
%%

   \newcommand{\vp}{ V^{\prime}}
   \newcommand{\ot}{Onetwo\xspace}
   \newcommand{\ct}{Curray\xspace}
   \newcommand{\pdiff}[2]{\frac{\partial{ #1}}{\partial{ #2}}}
   %useage \pdiff{#1}{#2}
   \newcommand{\ddiff}[1]{\frac{\partial}{\partial #1}}
   \newcommand{\strt}{\begin{eqnarray}}
   \newcommand{\trts}{\end{eqnarray}}
   \newcommand{\beq}{\begin{eqnarray}}
   \newcommand{\eeq}{\end{eqnarray}}
   \newcommand{\pdiffn}[3]{\frac{\partial^{#3}{#1}}{\partial{ #2}^{#3}}}
   \newcommand{\pdiffz}[3]{\frac{\partial{#1}}{\partial{ #2}}\bigg \vert_{#3}}
    \newcommand{\myint}{\int\limits_{\rho_{j-\frac{1}{2}}}^{\rho_{j+\frac{1}{2}}
}d\rho H \rho}
    \newcommand{\myind}{\int\limits_{\rho_{j-\frac{1}{2}}}^{\rho_{j+\frac{1}{2}}
}d\rho}
    \newcommand{\onehalf}{\frac{1}{2}}
    %\newcommand{\eqref}[1]{(\ref{#1})}
    \newcommand{\Eqref}[1]{Eq.~\eqref{#1}}
    \newcommand{\Eqrefs}[2]{Eqs.~\eqref{#1}-\eqref{#2}}

 \textheight8.5in
\newenvironment{narrow}[2]{%
   \begin{list}{}{%
   \setlength{\topsep}{0pt}%
   \setlength{\leftmargin}{#1}%
   \setlength{\rightmargin}{#2}%
   \setlength{\listparindent}{\parindent}%
   \setlength{\itemindent}{\parindent}%
   \setlength{\parsep}{\parskip}}%
\item[]}{\end{list}}




   %\onlyslides{5}


%----------------------------------------------------------------------------
      \color{blue} %\pagestyle{fancy}
  \begin{document}        
       \begin{center}
           \Large\bfseries Wedn. Meeting\\ % 1
           July 13,'00 \\
           H.S.J.

        \end{center}
    

%------------------------------------------------------------------
 
     \begin{slide}          \setlength{\topmargin}{-0.5in}
       \begin{center}
           \Large\bfseries Solution Methods% 1

        \end{center}
        \normalsize  

        \bigskip
        \hrule height4pt
        %\bigskip
         % \bf
        \begin{description}
          \bfseries \tiny
           \item {Pred. Corrector} Fast but not suitable for stiff
           models
           \item[Adaptive Method Lines] very slow,for stiff
             problems time step
             controlled by Runge Kutta accuracy requirements.
             \item[Globally Convergent Newton Based methods]
               Combines 
               steepest descent,with Newton line search and
               or trust reqion strategies to get solution within
               radius of convergence of pure Newton method. (Tries
               pure Newton first at each step). Hookstep variation
               uses an approximation to Hessian to introduce second
               orde curvature effects.
               \begin{enumerate}
                 \item fast for non stif, slow for stiff problems
                 \item ability to run time independent cases by
                   finding solution of pdes without time derivative
               \end{enumerate}
             \item{two chi methods}
               \begin{enumerate}
                 \item pure glf23
                 \item DV method 
                \end{enumerate}
    \end{description}
    \end{slide}

%-----------------------------------------------------------------------------
 
\end{document}  