\title{ EQUATIONS AND ASSOCIATED DEFINITIONS USED IN  ONETWO}
\author{H.~ST.~JOHN}
\date{   }   %make the date just some blanks
\maketitle


% \Large
\boldmath\
At this time this document is incomplete and may have errors in
content, typos, etc. It is a preliminary working version only!
This document was last edited (by S. Smith) on \today.
This document describes features of the code current with
the May '96 version of \ot. Some items described here
may not be available in earlier versions. This file is
called  source-o12.*  where the extension is ps for the postscript file, 
tex for the latex file and dvi for 
the dvi file. If you use tex you may get a copy
of the tex file and hack out parts that you might want 
to include in your own documents.

\section{Brief Description}
The Onetwo transport code solves a user selectable, arbitrary
combination of flux surface average parabolic equations representing
the radial diffusion and convection of:
\begin{itemize}
\item 1 or 2 thermal ion species and
up to 2 neutral species, 
\item the electron energy, 
\item the ion energy, 
\item the poloidal magnetic field, and 
\item the toroidal momentum.  
\end{itemize}
The magnetic geometry may be
self consistently maintained by solving the elliptic mhd equilibrium
equation simultaneously with the diffusion equations.  Both free and
fixed plasma boundary equilibrium models are supported. Auxiliary
heating models include Monte Carlo neutral beam deposition and electron
and ion cyclotron and fast wave r.f. heating.  Models for ohmic, bootstrap,
beam, and r.f. driven currents are included.  Energy confinement may be
simulated by selecting theoretical or empirical electron and/or ion
thermal conductivity models,including Rebut-Lallia, Shay, and Waltz-Dominguez.  A unique feature of the code is its ability to substitute
measured profiles for any of the dependent variables, solving the
diffusion equations only for the remaining unknown profiles. A combination
of Crank-Nicholson, predictor, and iterated corrector methods are used to
converge the non linear diffusion equations. The elliptic equations are
solved using variable finite difference schemes with single cyclic
reduction.


In the next section the actual equations solved by Onetwo are
given explicitly. The equations are similar to the original
ones presented in GA-A16178. However a new equation describing
the evolution of the toroidal momentum has been added as have 
bootstrap models and other source terms. Thus the original document
is badly out of date. It is my intent, through online documentation,
to fill in the gaps which have developed by making available
``ONETWO PAGES'' to the users. The equations appearing below have
terms separated in a manner which makes them suitable for the matrix
finite difference representation used  in the code, to facilitate
comparison with actual code fragments. Perhaps  more importantly,
the terms are given  in such a manner that they can be associated
with actual output quantities produced by the code. A consequence
is that simple rearrangement of terms  may be necessary in
order to see the physical conservation process being described.
Another consequence is that I use mixed math/fortran style in
some places. This is justifiable since the main purpose of this
document is to connect the output of \ot with the definitions
presented here. 
The equations in \ot were not cast in non dimensional form
so we specify the units of input and output terms as
appropriate. 
The dependent variables which are either evolved in time
(simulation mode) or 
are known a priori (analysis mode) are the ion densities,
$n_i(\frac{1}{cm^3})$,
the electron temperature $T_e(keV)$,the ion temperature
$T(keV)$,the poloidal magnetic field taken as the compound
quantity $FGH\rho B_{p0}(Gauss-cm) $ and the angular rotation speed
$\omega (\frac{1}{sec})$. In order to write some equations more
concisely we use the notation $u^i$ to refer to these dependent
variables with the understanding that i=1 is a collective index that 
ranges over all ion densities, i=2 implies $T_e$, i=3 implies $T$,
i=4 implies $FGH\rho B_{P0}$, and i=5 implies $\omega$.
