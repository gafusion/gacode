\documentstyle{article}
\begin{document}

\begin{thebibliography}{99}
\bibitem{hahm95a}
HAHM, T.~S. and BURRELL, K.~H.,
\newblock Physics of Plasmas {\bf 2} (1995) 1648.

and the excellent review article by

\bibitem{burr97a}
BURRELL, K.~H.,
\newblock Physics of Plasmas {\bf 4} (1997) 1499.
\end{thebibliography}

        When we take derivative with respect to the minor radius,
we take derivatives with respect to the flux surface half-width.

        The following \LaTeX fragment on the flow shear rate
is from my review article being prepared for publication:

The flow shear rate was derived by Hahm and Burrell \cite{hahm95a}
for general toroidal geometry
\begin{equation}
\omega_{E \times B} = \frac{ (R B_{\rm pol})^2}{B}
  \left( \frac{\partial}{\partial \psi} \right)
  \frac{E_r}{R B_{\rm pol}}
\end{equation}
where the radial electric field can be computed from ion force balance
\begin{equation}
E_r = \frac{ \partial p_i / \partial r }{ Z_i e n_i }
  - v_{{\rm pol,}i } B_{\rm tor} + v_{{\rm tor,}i } B_{\rm pol},
\end{equation}
where the three separate terms represent the effect of diamagnetic drift,
poloidal ion velocity, and toroidal ion velocity.
The toroidal velocity can be driven by unbalanced neutral beam injection
\cite{bath97a,erba99b} 
or, surprisingly, by radio frequency heating 
\cite{chang98a,coppi98a}, which applies torque in the direction of the
plasma current.
The poloidal velocity is currently taken to be given by the neoclassical
expression \cite{staebler97a}
\begin{equation}
v_{\rm pol} = \frac{0.8839 f_c}{0.3477 + 0.4058 f_c}
\frac{B_{\rm tor}}{Z_i e B^2} \frac{\partial T_i}{\partial r}
\end{equation}
where
\[ f_c = 1 - 1.46 \sqrt{\epsilon} + 0.46 \epsilon \sqrt{\epsilon}, \]
or from more sophisticated neoclassical computations
\cite{www-nclass,zhu99a}.

        As for momentum transport, as far as I know, the only anomalous 
transport model that currently computes the momentum transport flux is 
the GLF23 model.  Usually, researchers estimate the poloidal velocity
using only neoclassical momentum transport.

\end{document}

Ron Cohen wrote:
> 
> Trying to get toroidal rot. eq. wired up properly.
> 
> In mmm95, is the exb shearing rate just
> d v_ExB/d rminor or something more complicated?  If the former
> I assume I can take it as d <v_ExB>/drho * flux surf av. of |grad rho| ?
> 
> What would you suggest for a momentum diffusivity (and in what
> eq. to use it?)
> 
> -Ron-

-- 
Glenn Bateman (Research Physicist)     Lehigh University
Physics Department, 16 Memorial Drive East, Bethlehem, PA 18015 USA
Phone: (610) 758 5733 (Lehigh office)  (610) 758 5730 (Lehigh FAX)
http://www.lehigh.edu/~glb4      bateman@fusion.physics.lehigh.edu

