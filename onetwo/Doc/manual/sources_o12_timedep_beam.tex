\section{Time Dependent Beam Input}

\subsection{The Fast Ion Slowing Down Problem}

The simplest form of fast ion slowing down on a background plasma is given by
the Fokker-Planck equation:
\begin{eqnarray}
 \tau_s \pdiff{f_b}{t}(v,\zeta,t) = \frac{1}{v^2} 
 \ddiff{v}{(v^3+v_b^3)} f_b(v,\zeta,t) +\frac{Z_2v_c^3}{2 v^3}
 \ddiff{\zeta}{(1-\zeta^2)}  \pdiff{f_b}{\zeta}(v,\zeta,t)\nonumber \\  
 + \frac{\tau_s}{\tau_{cx}}f_b(v,\zeta,t) +
 \frac{\tau_s}{\tau_{fus}}f_b(v,\zeta,t) +
 \frac{\tau_s}{\tau_{terhm}}f_b(v,\zeta,t) +
 \tau_s S^i(v,\zeta,t)
 \label{eq:wav2}
\end{eqnarray}
where the azimutahl dependence of the fast ion velocity has been integrated out
due to assumed symmetry in that variable. Typically the charge exchange loss
rate detrmined by $(\tau_{cx})$, the loss rate due to fast ion fusion
$(\tau_{fus})$, and especially the explicit loss rate due to thermalization
$(\tau_{therm})$ are neglected Ref[]. We are interested in the solution of this
equation when the  source term has a (possibly repetitive) pulsed time
dependence of the form
\begin{equation} \label{eq:wav1}
 S^i(v,\zeta,t)= \frac{\dot S_0^i}{v^2}\delta(v-v_0) \delta(\zeta-\zeta_0) 
 \biggl( H(t-t_0^i) - H(t-t_1^i) \biggr)
\end{equation}
which is turned on and off at times  $t_0^i ,t_1^i $ (H is
the Heavyside step function).
The resulting fast ion distribution function due to this
source is
\begin{eqnarray}  %use eeqnarray for multiple line formulas
 f_b^i(v,\zeta)= \frac{\dot S_0^i \tau_s}{v^3+v_c^3} P_{tot}(v)
 \sum_{l=0}^{\infty}\frac{2l+1}{2}P_l(\zeta)P_l(\zeta_b)
 \left[\frac{v^3}{v_b^3} \left(\frac{v_b^3+v_c^3}{v^3+v_c^3}
 \right)\right]^{\frac{1}{6}l(l+1)Z_2} \nonumber \\  
 \bigg[
 H\Bigl(t-t_0^i - \tau_0(v_b)+\tau_0(v) \Bigr)
 -H\Bigl(t-t_1^i - \tau_0(v_b)+\tau_0(v) \Bigr)
 \bigg]
 \label{eq:wav3a}.
\end{eqnarray}
Here $\tau_0(v)$ is the time required for a fast ion of speed $v$ to thermalize
(i.e. to reach speed $v_t$) 
\begin{equation} 
 \tau_0(v)=
 \frac{1}{3}\tau_s\log\biggl(\frac{v^3+v_c^3}{v_t^3+v_c^3}\biggr) 
 \label{eq:wav4}
\end{equation}
and $P_{tot}(v) $ is the total loss rate due to charge exchange, fast ion
fusion, and thermalization. In what follows, we assume that fast ion
interactions are negligible (however beam-beam fusion is ??)   so that we may
use the linear superpostion of solutions from various sources to get the total
fast ion distribution function at any given time:
\begin{equation}
 f_b^{tot}(v,\zeta) = \sum_{i}f_b^i(v,\zeta).
 \label{eq:wav4a}
\end{equation} 
In order to use these results in a thermal transport code
we form various moments of the distribution function with 
the collision operators. For each individual pulse we are
interested in the following moments. \\
The fast ion density :
\begin{equation}
 n_b^i(t) \equiv \int_{v_{min}^i(t)}^{v_{max}^i(t)}
 \int_{-1}^{1} 
 f_b^i(v,\zeta)
 v^2dvd\zeta   
 \label{eq:wav5a}
\end{equation}
The fast ion stored energy density :
\begin{equation}
 E_b^i(t) \equiv \frac{1}{2}m_b\int_{v_{min}^i(t)}^{v_{max}^i(t)}
 \int_{-1}^{1} 
 v^2 f_b^i(v,\zeta)
 v^2dvd\zeta   
 \label{eq:wav5b}
\end{equation}
The power desnity delivered to the electrons:
\begin{equation}
 Q_e^i(t) \equiv \frac{1}{2}m_b\int_{v_{min}^i(t)}^{v_{max}^i(t)}
 \int_{-1}^{1} 
 \left[\ddiff{v}{v^3 f_b^i(v,\zeta)}\right]
 v^2dvd\zeta   
 \label{eq:wav5c}
\end{equation}
The energy delivered to the ions:
\begin{equation}
 Q_i^i(t) 
 \equiv \frac{1}{2}m_b \int_{v_{min}^i(t)}^{v_{max}^i(t)}
 \int_{-1}^{1} 
 \left[\ddiff{v}{v_c^3 f_b^i(v,\zeta)}\right]
 v^2dvd\zeta   
 \label{eq:wav5d}
\end{equation}
%SPS Commenting out the following wrong equations
The fast ion momentum transfer to electrons:\\
%\begin{equation}
% n_b^i(t) \equiv \int_{v_{min}^i(t)}^{v_{max}^i(t)}
% \int_{-1}^{1} 
% f_b^i(v,\zeta)
% v^2dvd\zeta   
% \label{eq:wav5e}
%\end{equation}
The fast ion momentum transfer to ions:\\
%\begin{equation}
% n_b^i(t) \equiv \int_{v_{min}^i(t)}^{v_{max}^i(t)}
% \int_{-1}^{1} 
% f_b^i(v,\zeta)
% v^2dvd\zeta   
% \label{eq:wav5f}
%\end{equation}
The fast ion fusion rates:\\
%\begin{equation}
% n_b^i(t) \equiv \int_{v_{min}^i(t)}^{v_{max}^i(t)}
% \int_{-1}^{1} 
% f_b^i(v,\zeta)
% v^2dvd\zeta   
% \label{eq:wav5g}
%\end{equation}
The fast ion charge exchange rate:\\
%\begin{equation}
% n_b^i(t) \equiv \int_{v_{min}^i(t)}^{v_{max}^i(t)}
% \int_{-1}^{1} 
% f_b^i(v,\zeta)
% v^2dvd\zeta   
% \label{eq:wav5h}
%\end{equation}
The fast ion  thermalization rate:\\
\begin{equation}
 \dot S_{th}^i = \dot S^i - \dot L^i -\pdiff{n_b^i}{t}
 \label{eq:wav5i}
\end{equation}
Here $\dot{L}^i $ is the loss rate of fast ions due to all factors except
thermalization:
\begin{equation}
 L^i(t)  =  \int_{v_{min}^i(t)}^{v_{max}^i(t)}
 \int_{-1}^{1} 
 f_b^i(v,\zeta) P(v)
 v^2dvd\zeta   
 \label{eq:wav5hh}
\end{equation}
The reason \Eqref{eq:wav5i} is written in this particular form rather than in
terms of a more fundamental definition such as
\begin{equation}
 \dot S_{th}^i = \lim_{\Delta t \to 0}
 \frac{\int_{v_{th}(t)}^{v(t+\Delta t)}
 \int_{-1}^{1} 
 f_b^i(v,\zeta)v^2dvd\zeta}{\Delta t}
 \label{eq:wav5j}
\end{equation}
is that the formulation of the Fokker-Planck equation \eqref{eq:wav2} is not
valid near thermal energies and hence the form of $f_b $ is not adequate to
evaluate \Eqref{eq:wav5j}. On the  other hand, \Eqref{eq:wav5j} consists of
quantities that depend on the slowing down distribution at higher energies and
hence yields a reasonable approximation. We note that the time delay associated
with the appearance of thermalized fast ions is properly given by this approach.
For example, during the linear (in time) rise of the stored fast ion density
that occurs when the loss rate is zero, the time  derivative term cancels the
source term initially, leading to a thermalization rate of zero until the
derivative term changes.

In these equations  both the upper and lower limits of integration are functions
of time due to the transient nature of the source:
\begin{align}
 v_{max}(t) & = v_b^i &  t_0^i\le t \le t_1^i \\
 & = [((v_b^i)^3
 +v_c^3)\exp(\frac{-3(t-t_1^i)}{\tau_s})
 -v_c^3]^{\frac{1}{3}} &t_1^i \le t \le
 t_1^i+\tau_0(v_b) \\
 & = v_t & t \ge t_1^i+\tau_0(v_b) \\
 v_{min}(t) & =  [((v_b^i)^3
 +v_c^3)\exp(\frac{-3(t-t_0^i)}{\tau_s})
 -v_c^3]^{\frac{1}{3}}  &t_0^i \le t \le
 t_0^i+\tau_0(v_b) \\
 & = v_t & t \ge t_0^i+\tau_0(v_b).
\end{align}

The upper limit of integration remains fixed at the fast ion birth speed $v_b^i$
until the source $ \dot S_0^i $ is shut off.  Thereafter this upper limit
decreases  since ions of speeds greater than $ v_{max} $ are not replenished by
the source.  The lower limit starts at the same value as the upper limit and
decreases as time goes on until it reaches an arbitrarily defined thermal cutoff
value $ v_t $. If the source is on long enough ($ t_1^i-t_0^i \ge \tau_0(v_b)$) 
the moments  become time independent when the lower limit of  integration 
reaches the value $ v_t $ at time $t_0^i +\tau_0(v_b)$ (typically $v_t$ is taken
as 0.0, but in \ot it is user specifiable). Additionally, since we are
neglecting losses during the slowing down process, $n_b^i $ may also display a 
time independent plateau when the pulse length, $ t_1^i -t_0^i $, is shorter
than $ \tau_0(v_b)$. In this case both $v_{max}(t) $ and $v_{min}(t) $ are
decreasing at the same constant rate so that the density doesn't start to change
until $v_{min}(t) = v_t $. Thereafter the contribution to the density made by
the integral, \Eqref{eq:wav5}  decreases steadily as $v_{max}(t) $ approaches
$v_t$.

Toroidal plasma roation is taken into account by 


\subsection{Interpretation of results}
By neglecting charge exchange, fusion, and explicit thermalization  losses, the
fast ion  density has a simple solution useful for demonstration purposes:
\begin{equation}
 n_b^i(t) \equiv \int_{v_{min}^i(t)}^{v_{max}^i(t)}
 \int_{-1}^{1} 
 f_b^i(v,\zeta)
 v^2dvd\zeta   
 \label{eq:wav5} \qquad (?)
\end{equation}
The relationship between the beam slowing down time and the on/off switching
frequency of the beam sources, coupled with the superposition of sources from
different beams and energy components can produce complex waveforms. Furthermore
the fundamental time step $ dt $ that  the transport equations are solved with
is typically dynamically adjusted and could lead to innapropriate sampling times
for the neutral beams, even if the user is careful to pick consistent times
initially.  A final complication is that the starting time of the transport
simulation is typically not the same as the initial startup of the beams. Thus
an initial condition for neutral beam related quantities must be generated from
the user input. These issues are addressed in this section and some simple
examples are presented.
\begin{figure}[hbt] %note: figure environment not available in slides
 \centering
 \mbox{\epsfig{file=plot1a.eps,height=10cm,width=12cm}}
 \caption{Example waveform from neutral beam deposition. The square wave
 composed pulses a, b, c, etc., are the beam switching times in arbitrary units.
 With the given slowing down time  the resulting fast ion density waveform
 (total) is  due to the summation of up to three individual components, aa, bb,
 cc due to the pulses.}
 \label{Figbwav1}
\end{figure}

The situation described above is depicted in
Fig.~\ref{Figbwav1}. Here we show a number of individual
pulses
of duration 60$ms$, spaced 10$ms$ apart. The beam slowing
down time is approximately $\tau_0(v_b) = 90 ms $. Due to the assumption
of linearity the total fast ion density in the figure is the
linear superpostion of the individual responses to each beam
pulse. As shown, up to three individual pulses combine to form
the resultant fast ion density for this hypothetical
case. Since the pulse length is less than the slowing down
time, individual responses  never reach the
steady state value \Eqref{eq:wav5} ($ 90 ms \cdot 1\times 10^{20}
\frac{1}{cm^3s} = 9\times 10^{18}\frac{1}{cm^3}$).  As a consequence, the individual responses
(aa,bb,cc,...) have a flat top (plateau)  value because the
density is not changing as the group of fast ions in the range
$(v_{max}(t),v_{min}(t)) $ slow down as was explained above. The
individual responses such as the one labeled aa in the figure
are the result of evaluating the integral in
\Eqref{eq:wav5} in accordance with the limits of
integration given above:
\begin{align}
 n_b^i(t) &= \dot S_0^i(t-t_0^i) & t_0^i \le t \le
 min(t_1^i,\tau_0(v_b)) \\
 & =  \dot S_0^i  (\tau_0(v_b) - t_0^i) &  \\
 & =  \dot S_0^i (t_1^i -  t_0^i)  &  \\
 & =  \dot S_0^i (t_1^i -  t_0^i) & \\
\end{align}

\begin{figure}[hbt] %note: figure environment not available in slides
 \centering
 \mbox{\epsfig{file=plot2.eps,height=10cm,width=12cm}}
 \caption{Example waveform from neutral beam deposition. The square wave (with
 pulses a,b,c)  is the beam switching times in arbitrary units. The two
 resulting fast ion density waveforms are due to different time steps used in
 the calculations.}
 \label{Figbwav2}
\end{figure}

When the pulse length is increased so that the fast ion density reaches its
fully developed state before the pulse is turned off we have the situation shown
in fig ??.

It should be noted that time step control must be asserted by these
computations. Unfortunately it is not enough to simply enforce the rule that
each source switching time be observed exactly by the time stepping routines
that advance the transport equations. Two additional time values must be
explicitly enforced. The first, $ t_2^i $, is when $vmin(t_2^i) = v_t$. The
second, $t_3^i$, is when $n_b^i(t_3^i) $ reaches zero. Unlike the beam switching
times  $(t_0^i,t_1^i) $ the set of values $( t_2^i,t_3^i)$ is not amenable to
prediction on an a priori basis. Instead these times are determined as the
calculations proceed.

Finally we remark that the computations are designed to accept an arbitrary time
step dt, where the signficance of dt is that the state of the system is to be
updated from time t to time $t_{new} = t+dt $. However, on return from the
modules that perform the computations described above, it is not guaranteed that
$t_{new} =  t+dt$. In fact the new time will be given by
\begin{equation}
 t_{new} = t + min(t_j^i -t ,dt)
 \label{eq:wav10},
\end{equation}
where only those values for which $t_j^i -t >  0.0 $ are to be considered, 
$j=0,1,2,3 $, and i ranges over all pulses of each source of each beam. Since
there is no a priori assumed relationship between the phasing of various beam
lines and sources it may become quite tedious to advance the transport equations
if the time step is controlled by \Eqref{eq:wav10} rather than by other physics.
To avoid this situation times $t_3^i $ and/or $ t_2^i $ may optionally be
ignored. This will result in some details of the waveforms being incorrect.
However transport results are not necessarily affected since the generated
waveform will be correct at the times that output is requested. For example,
consider Fig.~\ref{Figbwav2}.  This case is similar to the previous one but has
a longer pulse length $(110 ms)$ and hence the fast ion density reaches its
saturated value before the beam turns off. In the figure we show the exact
solution as well as two approximate solutions where $t_3^i $ and where both
$t_3^i and t_2^i$ are ignored.

Special waveforms are possible since there is no assumption about the 
relationship of sources for a  given beam or between beams. For example, in
Fig.~\ref{Figbwav3},
\begin{figure}[hbt] %note: figure environment not available in slides
 \centering
 \mbox{\epsfig{file=plot3.eps,height=10cm,width=12cm}}
 \caption{Example waveform from neutral beam deposition.
 Two independent sources produce wavetrains with pulses a
 and b respectively. The resulting fast ion density does
 not have a plateau value (aa-bb) due to the different
 intensities of the sources.}
 \label{Figbwav3}
\end{figure}
we have a single beam line with two sources. The first source  produces pulse
train a and the second source pulse train b. The second source has 1.5 times the
intensity of the first source. The resulting fast ion density has a rising
region (aa-bb) instead of a plateau from (aa-bb) due to the difference in source
strengths.


\begin{figure}[hbt] %note: figure environment not available in slides
 \centering     
 \mbox{\epsfig{file=thermal_source.eps,height=10cm,width=12cm}}
 \caption{Illustration of the delayed particle source due to a single beam pulse
 that starts at 1 msec and ends at 41 msec. The delayed sources are due to the
 different slowing down times of the three beam energy components.}
 \label{Figbwav3c}
\end{figure}

\subsection{Beam Initial Conditions}
Suppose that for the situation shown in Fig.~\ref{Figbwav3}
we wish to start our analysis at time $t = 0.5 sec$. Considering 
only the fast ion density we see that it is in fact possible to 
assume that the beam was not on prior to $ t = 0.5 sec $
because its influemce has decayed away. Unfortunately, the
general situation is not so simple. Suppose instead that we
want to start the transport analysis at time $ t = 0.28 sec $
in Fig.~\ref{Figbwav3}. In this case the beam history must be
taken into account for the time prior to the start of the
analysis. Furthermore automatic corrections in the
MHD/transport coupling of \ot  can result in arbitrary time
rollbacks to previous solution points. Thus we are forced to 
generate and periodically update some sort of beam history file
as part of the solution scheme. The proper way to generate an
intial history file depends on how the code is run
however. For example in TDEM mode the user will expect
that the TDEM information is used in creating the file. In a
non-TDEM run other options have to be pursued. 
