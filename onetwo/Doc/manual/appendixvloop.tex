\section{Vloop  Boundary Condition} 

A boundary condition that allows specification  of the plasma
loop voltage as a function of time instead of the total current
was introduced into \ot starting with version 3.4 . The
relevant input parameters are input in the first namelist of inone.
The options are:
\begin{description}
 \item[vloop\_bc(1\ldots kbctim)] in volts: Is an array of loop voltages (at the
 plasma boundary). If only one value is given then it is assumed that
 \texttt{vloop\_bc} is constant in time. \texttt{Vloop\_bc} takes precedence
 over \texttt{totcur}!!  Note that as the resistivity changes vloop will drive
 different amounts of ohmic current and hence the total current will float. To
 use this option specify at least one value of \texttt{vloop\_bc} in the first
 namelist of inone.  
 \item[vloop\_bc\_time(1\ldots kbctim)] in sec: \texttt{vloop\_bc} can be given
 in \texttt{bctime}(1\ldots kbctim)  or it can be given on a separate time base,
 \texttt{vloop\_bc\_time}. \texttt{vloop\_bc\_time} has the same maximum length
 as \texttt{bctime} (i.e. \texttt{kbctim}). The code will detect how many valid
 entries there are in  \texttt{vloop\_bc\_time}. The \texttt{vloop\_bc\_time}
 array must be in 1:1 correspondence with the  \texttt{vloop\_bc} array but the
 two arrays do not have to be in any special order. (Onetwo will time sort  both
 of the arrays according to monotonically increasing values in
 \texttt{vloop\_bc\_time}.) If \texttt{vloop\_bc\_time} is not used then
 \texttt{vloop\_bc} must correspond to the elements in \texttt{bctime} and
 \texttt{bctime} itself must be ordered monotonically increasing in time
 (\texttt{bctime} is not time sorted). Obviously \texttt{vloop\_bc\_time} must
 be a superset of the start and end times of the analysis  (e.g. \texttt{time0,
 timmax}).
 \item[vloopvb] Set to 1 to get special monitoring output to file
 vloop\_monitor.txt. This is used primarily as a debugging
 aid. (This file could get to be quite lengthy for a long
 time run.)  The file can be read/plotted with readvloop.py.
 But note that many of the local machines do not have the
 Biggles plot package that readvloop.py uses.
 You can do one of three things:
 \begin{enumerate}
  \item Install Biggles (very easy if Python is available)
  \item Rewrite the plot calls in readvloop.py to your graphics package
  \item Just delete the lower half of vloop.py and use it only to read the data
  (and then pass it into IDL for example)
 \end{enumerate}
\end{description}

The method used to introduce \texttt{vloop\_bc} into the boundary conditions is
as follows. This method was used because it is consistent with other
requirements in the code where the boundary is not necessarily at $\rho = 1$ .
At the initial time the ohmic current (or parallel electric field)   must be
known throughout the plasma. In \ot we continue to obtain this profile in the
standard manner. That is, the total current, usually taken from the eqdsk, is
used as the boundary condition for the startup guess to generate an ohmic
current profile consistent with the current drive models that are active at this
time. This supplies the  starting ohmic current profile for all grid points
except the point at the very edge of the plasma. To get \texttt{curohm} at the
edge we use the condition 
\beq 
 \label{ae1} V_l = 2\pi R_0 E_0 = \left.\eta H
 \left<\frac{\vec{J}_{ohmic}\cdot\vec{B}}{B_{T0}}\right>\right\vert_{rho=1,new}, 
\eeq 
where $V_l$ is the input supplied loop voltage (i.e. \texttt{vloop\_bc}) at 
this time. The total current is then adjusted to reflect  this new value of the
edge ohmic current by adjusting the total current density:
\begin{multline}  \label{ae2}
 \left.\left< \frac{J_\phi R_0 }{R}\right>\right\vert_{rho=1,new}= 
 \left.\left< \frac{J_\phi R_0 }{R}\right> \right\vert_{rho=1,prev }
 -\left.\left<\frac{\vec{J}_{ohmic} \cdot \vec{B}}{B_{T0}}\right>
  \right\vert_{\rho =1,prev}
 \\
 +\left.\left<\frac{\vec{J}_{ohmic} \cdot \vec{B}}{B_{T0}}\right>
  \right\vert_{rho=1,new}.
\end{multline}
The new current density, \Eqref{ae2}, (which differs from the old
current density only in the edge value) is then integrated to obtain a
new total current:
\beq \label{ae3}
 I = 2\pi R_0 \int_{0}^{1} \left< \frac{J_\phi R_0 }{R}\right> H \rho d\rho.
\eeq
Finally this value of I is used as the actual boundary condition for
the $\rho FGH B_{P0} $ profile to advance Farady's law in time.
Some iteration is required to achieve consistency. In the above
formulae  we have used $\rho =1 $ symbolically to indicate the plasma
edge ($\rho $ actually has units of cm and ranges from 0 to some
maximum value depending on the enclosed toroidal flux).

An example of the output is given in Fig.~\ref{aefig1}. In part a of the figure
the total current is plotted as a function of time. The current increases even
though the edge current density, part b, follows the input loop voltage (part
d). The edge ohmic current density [as given by \Eqref{ae1}] is shown in part c.
\begin{figure}[hbt] %note: figure environment not available in slides
 \centering
 \mbox{\epsfig{file=fig2raw.eps,height=15cm,width=15cm}}
 \caption{Example of using the vloop boundary condition for an ITER
 AT type discharge.}
 \label{aefig1}
\end{figure}
